\section{Processi di Supporto}
\subsection{Documentazione}
Ogni processo\glo e attività significativi volti allo sviluppo del progetto sono documentati. Lo scopo di questa sezione è definire gli standard che riguardano i documenti prodotti durante l'intero ciclo di vita del software. I documenti sono consultabili nelle relative sezioni della repository\glo: \url{https://github.com/teamafkSWE/docs}. 		

\subsection{Descrizione}
Questo capitolo contiene le decisioni e le norme che sono state scelte per la scrittura, verifica e approvazione della documentazione ufficiale. L'insieme di tali norme garantisce consistenza ed omogeneità nella stesura di testi.

\subsection{Ciclo di vita di un documento}
Ogni documento segue le seguenti fasi di ciclo di vita:
\begin{itemize}
\item \textbf{Sviluppo}: creazione del documento, definizione della struttura e prima stesura di tutte le parti che lo compongono;
\item \textbf{Verifica}: un documento entra in fase di verifica successivamente al suo completamento. \'E dovere del \textit{Responsabile} assegnare tale compito ad almeno un verificatore. Quest'ultimo deve applicare le procedure di verifica e segnalare eventuali modifiche da apportare al documento;
\item \textbf{Approvazione}: il \textit{Responsabile} approva il documento, che sarà quindi ritenuto completo e pronto per il rilascio.
\item \textbf{Rivisitazione e ampliamento}: con l'avanzare del progetto si prevede di espandere ciascun documento, aggiungendo nuove sezioni o migliorando quanto scritto in precedenza. Sarà compito del \textit{Responsabile} istanziare una nuova fase di sviluppo per provvedere alla realizzazione di questi aggiornamenti. Al termine di essa, vengono eseguite nuovamente le fasi di verifica ed approvazione del documento.
\end{itemize}

\subsection{Template}
\'E stato creato un template \LaTeX{} per uniformare la struttura grafica e lo stile di formattazione dei documenti. Lo scopo dei template è quello di permettere, a colui che redige il documento, di adottare automaticamente le conformità previste dalle \textit{Norme di Progetto}. Nel caso quest'ultime cambiassero, essi permettono di agevolare la procedura di adeguamento alle nuove norme.

\subsection{Struttura dei documenti}
Un file ''nome\_file.tex'' (in cui ''nome\_file'' verrà sostituito dal nome del documento) raccoglie, tramite comandi di input, le sezioni di cui è composto il documento. Tra i file in input ci sono:
\begin{itemize}
\item ''AFKstyle.sty'', contenente i pacchetti necessari alla compilazione e i comandi relativi all'impostazione grafica;
\item ''copertina.tex'', che contiene i comandi \LaTeX{} per l'impostazione della prima pagina del documento.
\end{itemize}

\subsubsection{Prima pagina}
Il frontespizio è la prima pagina del documento ed è così strutturato:\begin{itemize}
\item \textbf{Logo del gruppo}: logo del \textit{TeamAFK} visibile come primo elemento centrato in alto;
\item \textbf{Titolo}: nome del documento, posizionato centralmente sotto il logo;
\item \textbf{Gruppo e progetto}: nome del gruppo e del progetto \textit{Predire con Grafana}, visibile centralmente sotto il titolo;
\item \textbf{Recapito}; indirizzo di posta elettronica del gruppo, posizione sotto il nome del gruppo e del progetto;
\item \textbf{Informazioni sul documento}: tabella posizionata al di sotto del recapito, contenente le seguenti informazioni: \begin{itemize}
\item \textbf{Versione}: versione del documento;
\item \textbf{Approvatore}: nome e cognome dei membri del gruppo incaricatoi dell'approvazione del documento;
\item \textbf{Redattori}: nome e cognome dei membri del gruppo incaricati della redazione del documento;
\item \textbf{Verificatori}: nome e cognome dei membri del gruppo incaricati della verifica del documento;
\item \textbf{Uso}: tipolo d'uso del documento, che può essere ''interno'' o ''esterno'';
\item \textbf{Distribuzione}: destinatari del documento.
\end{itemize}
\item \textbf{Descrizione}: descrizione sintetica del documento, posizionata centralmente in fondo alla pagina.
\end{itemize}

\subsubsection{Registro delle modifiche}
Ogni documento dispone di un \textit{Registro delle Modifiche}: una tabella posta a seguito della prima pagine, contenente le modifiche apportate al documento. In essa sono indicati: \begin{itemize}
\item versione del documento dopo la modifica;
\item data della modifica;
\item nominativo di chi ha modificato;
\item ruolo di chi ha modificato;
\item breve descrizione della modifica.
\end{itemize}

\subsubsection{Indice}
L'indice ha lo scopo di riepilogare e dare una visione macroscopica della struttura del documento, mostrando le parti gerarchiche di cui è composto. Ogni documento è corredato dall'indice dei contenuti, posizionato dopo il registro delle modifiche. Se sono presenti immagini o tabelle all'interno del documento, l'indice dei contenuti è seguito prima dalla lista delle immagini, poi dalla lista delle tabelle.

\subsubsection{Contenuto principale}
La struttura delle pagine di contenuto è così definita: \begin{itemize}
\item \textbf{logo}: presente in alto a sinistra;
\item \textbf{nome del documento}: presente in alto a destra;
\item \textbf{riga di separazione}: divide l'intestazione dal contenuto;
\item \textbf{contenuto della pagina}: posto tra l'intestazione e il piè di pagina;
\item \textbf{riga di separazione}: divide il contenuto dal piè di pagina;
\item \textbf{nome e versione del documento}: posto in basso a sinistra;
\item \textbf{numero della pagina}: presente in basso a destra, con il formato ''Pagina X di Y'', in cui la X indica il numero della pagina corrente e Y il numero totale delle pagine.
\end{itemize}

\subsubsection{Verbali}
I verbali vengono prodotti dal/i soggetto/i incaricato/i alla loro stesura in occasione di incontri tra i membri del team, con o senza la presenza di referenti esterni. \'E prevista la stesura di più verbali, uno per ogni incontro.
La struttura dei verbali è così definita: \begin{itemize}
\item \textbf{Luogo}: luogo di svolgimento dell'incontro;
\item \textbf{Data}: data dell'incontro, nel formato \texttt{YYYY-MM-DD};
\item \textbf{Ora di inizio}: l'orario di inizio dell'incontro;
\item \textbf{Ora di fine}: l'orario di fine dell'incontro;
\item \textbf{Partecipanti}: elenco dei membri del gruppo presenti all'incontro e, se presenti, i nominativi delle persone esterne che vi hanno partecipato;
\item \textbf{Topic}: argomenti affrontati durante l'incontro.
\end{itemize}
Ogni verbale dovrà essere denominato secondo il seguente formato: \\ \\
\centerline{\textbf{verbaleTipologia\_YYYYMMDD}} \\ \\
dove per ''Tipologia'' bisognerà indicare la tipologia d'uso, con la prima lettera maiuscola: \begin{itemize}
\item \textbf{Interno}: verbale ''interno'', concentrato sul riassunto dell'incontro tra i membri del team;
\item \textbf{Esterno}: verbale ''esterno'', concentrato sulla trattazione di argomenti con partecipanti esterni al gruppo, in particolare domande e risposte riguardanti il progetto in sè.
\end{itemize}

\subsubsection{Note a piè di pagina}
Eventuali note vanno indicate nella pagina corrente, in basso a sinistra. Ogni nota deve riportare un numero e una descrizione.

\subsection{Norme tipografiche}
\subsubsection{Convenzioni sui nomi dei file}
I nomi di file (estensione esclusa) e cartelle utilizzano la convenzione "Snake case\glo" e alcune regole aggiuntive elencate di seguito: \begin{enumerate}
\item i nomi dei file composti da più parole usano il carattere \textit{underscore} come carattere separatore;
\item i nomi sono scritti interamente in minuscolo;
\item le preposizione \textbf{vanno} messe.
\end{enumerate}
Alcuni esempi \textbf{corretti} sono: \begin{itemize}
\item studio\_di\_fattibilità;
\item analisi\_dei\_requisiti.
\end{itemize}
Alcuni esempi \textbf{non corretti} sono: \begin{itemize}
\item Norme\_di\_progetto (usa maiuscole);
\item norme-di-progetto (non utilizza underscore come separatore);
\item norme\_progetto (omette la preposizione "di").
\end{itemize}

\subsubsection{Glossario}
\begin{itemize}
\item ogni termine inserito nel \textit{Glossario} è marcato con una \textbf{G} maiuscola a pedice, solamente nella sua prima occorrenza;
\item non vengono segnate con la \textbf{G} a pedice le parole da \textit{Glossario} presenti nei titoli e nelle didascalie di immagini e tabelle;
\item se nel \textit{Glossario}\textit{\_}\textit{v1.0.0} un termine presenta una descrizione che utilizza termini da glossario, è necessario trattare questi termini come tali, segnando la \textbf{G} a pedice e aggiungendoli al documento con la relativa descrizione;
\item se nel \textit{Glossario}\textit{\_}\textit{v1.0.0} è presente un termine sinonimo (o tradotto il lingua inglese) di un altro già presente, bisognerà rimandare quella parola alla relativa definizione attraveso il comando \verb|\hyperref[par:"nome_paragrafo"]| e la relativa label \verb|\label{par:nome_paragrafo}|, posta sopra la prima occorrenza di definizione.
\end{itemize}
