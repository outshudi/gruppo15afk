\section{Processi Organizzativi}

%\subsection{Gestione di Progetto}

%\subsubsection{Ruoli di progetto}
%\paragraph{Responsabile di progetto}



\subsection{Processi di Coordinamento}
Di seguito vengono descritte le norme che regolano le comunicazioni e gli incontri del gruppo, che siano tra i membri o con committenti e proponenti
\subsubsection{Gestione Comunicazioni}
\paragraph{Comunicazioni Interne}
Le comunicazioni interne ai membri del gruppo vengono gestite  tramite 2 applicazioni:
\begin{itemize}
	\item Telegram\glo
	\item Discord\glo
\end{itemize}
\'E stato disposto un gruppo Telegram, sul quale si discute di tematiche generali o collettive, garantendo risposte rapide e ordinate in caso di decisioni per votazione grazie a funzioni particolari dell'applicazione. \\
Discord viene usato principalmente per le riunioni tra i membri del gruppo, ma l'applicazione mette anche a disposizione dei canali testuali. Tali canali sono stati suddivisi per tema e vengono usati per le comunicazioni specifiche per agevolare la stesura dei documenti. Tali canali sono:
\begin{itemize}
	\item \textbf{general}: Per discussioni riguardanti rotazioni dei ruoli e decisione degli argomenti da discutere nelle riunioni;
	\item \textbf{links}: Per tenere traccia di tutti i link utili al progetto;
	\item \textbf{analisi-requisiti}: Per discutere gli Use Case\glo e i requisiti necessari alla stesura dell'\textit{Analisi dei Requisiti};
	\item \textbf{norme}: Per discutere riguardo le regole del \textit{Way of Working} del gruppo, le norme da seguire e, di conseguenza, la stesura del documento \textit{Norme di progetto}\glo;
	\item \textbf{piano-progetto}: Per confrontarsi riguardo il monte ore dei vari ruoli e per facilitare la stesura del documento \textit{Piano di progetto}\glo;
	\item \textbf{piano-qualifica}: Per discutere strategie da attuare per garantire qualità attraverso verifica\glo e validazione\glo
\end{itemize}
\paragraph{Comunicazioni Esterne}
Le comunicazioni con soggetti esterni al gruppo sono di competenza del responsabile. Gli strumenti predefiniti sono la posta elettronica, dove viene utilizzato l'indirizzo \href{mailto:gruppoafk15@gmail.com}{gruppoafk15@gmail.com}.
Per comunicare con \textit{Zucchetti} si il servizio Skype\glo per le chiamate. Il responsabile ha il compito di tenere informati gli altri componenti
del gruppo in caso di assenza.

\subsubsection{Gestione Riunioni}


%\subsection{Strumenti}
