\section{Processi Organizzativi}

\subsection{Gestione di Progetto}

\subsubsection{Ruoli di progetto}
Ciascun membro del gruppo, a rotazione, deve ricoprire il ruolo che gli viene assegnato e che corrisponde all'omonima figura aziendale. Nel Piano di Progetto vengono organizzate e pianificate le attività assegnate agli specifici ruoli. I ruoli che ogni componente del gruppo è tenuto a rappresentare sono descritti di seguito.
\paragraph{Responsabile di progetto}\mbox{} \\ \mbox{} \\



\subsection{Processi di Coordinamento}
Di seguito vengono descritte le norme che regolano le comunicazioni e gli incontri del gruppo, che siano tra i membri o con committenti e proponenti
\subsubsection{Gestione Comunicazioni}
\paragraph{Comunicazioni Interne}\mbox{} \\ \mbox{} \\
Le comunicazioni interne ai membri del gruppo vengono gestite  tramite 2 applicazioni:
\begin{itemize}
	\item Telegram\glo;
	\item Discord\glo.
\end{itemize}
\'E stato disposto un gruppo Telegram, sul quale si discute di tematiche generali o collettive, garantendo risposte rapide e ordinate in caso di decisioni per votazione grazie a funzioni particolari dell'applicazione. \\
Discord viene usato principalmente per le riunioni tra i membri del gruppo, ma l'applicazione mette anche a disposizione dei canali testuali. Tali canali sono stati suddivisi per tema e vengono usati per le comunicazioni specifiche per agevolare la stesura dei documenti. Tali canali sono:
\begin{itemize}
	\item \textbf{general}: Per discussioni riguardanti rotazioni dei ruoli e decisione degli argomenti da discutere nelle riunioni;
	\item \textbf{links}: Per tenere traccia di tutti i link utili al progetto;
	\item \textbf{analisi-requisiti}: Per discutere gli Use Case\glo e i requisiti necessari alla stesura dell'\textit{Analisi dei Requisiti};
	\item \textbf{norme}: Per discutere riguardo le regole del \textit{Way of Working} del gruppo, le norme da seguire e, di conseguenza, la stesura del documento \textit{Norme di progetto}\glo;
	\item \textbf{piano-progetto}: Per confrontarsi riguardo il monte ore dei vari ruoli e per facilitare la stesura del documento \textit{Piano di progetto}\glo;
	\item \textbf{piano-qualifica}: Per discutere strategie da attuare per garantire qualità attraverso verifica\glo e validazione\glo.
\end{itemize}
\paragraph{Comunicazioni Esterne}\mbox{} \\ \mbox{} \\
Le comunicazioni con soggetti esterni al gruppo sono di competenza del responsabile. Gli strumenti predefiniti sono la posta elettronica, dove viene utilizzato l'indirizzo \href{mailto:gruppoafk15@gmail.com}{gruppoafk15@gmail.com}.
Per comunicare con \textit{Zucchetti} si il servizio Skype\glo per le chiamate. Il responsabile ha il compito di tenere informati gli altri componenti
del gruppo in caso di assenza.

\subsubsection{Gestione Riunioni}
Le riunioni possono essere interne o esterne. All'inizio di ogni riunione il \textit{Responsabile} nomina, tra i componenti del gruppo, un \textit{Segretario} che si occuperà di far rispettare l'ordine del giorno. Inoltre ha l'onere della stesura del \textit{Verbale di Riunione}\glo

\paragraph{Riunioni interne} \mbox{} \\ \mbox{} \\
E' compito del \textit{Responsabile} organizzare riunioni interne al gruppo. Ciò prevede, più nello specifico, la stesura dell'ordine del giorno e
stabilire data, orario e luogo di incontro, mediando se necessario con i membri per permettere la presenza di tutti. Per agevolare tutti, le riunioni sono tenute principalmente usando Discord, così da essere il più facilmente raggiungibili.
Il \textit{Responsabile} deve inoltre assicurarsi, attraverso la comunicazione
mediante i mezzi propri del gruppo, che ogni componente sia pienamente a conoscenza della riunione in tutti i suoi dettagli. \\
D'altro canto ogni membro del gruppo deve presentarsi puntuale agli appuntamenti,
e comunicare in anticipo eventuali ritardi o assenze adeguatamente giustificate.

\paragraph{Riunioni esterne} \mbox{} \\ \mbox{} \\
E' nuovamente compito del \textit{Responsabile} organizzare riunioni esterne.
Nello specifico egli deve preoccuparsi di contattare l'azienda proponente per fissare gli
incontri e qualora sia necessario, tenendo conto anche delle preferenze di date e orario
espresse dagli altri membri del gruppo. La partecipazione a tali riunioni deve essere,
a meno di casi eccezionali, unanime.
Ogni membro del gruppo può, inoltre, esprimere al Responsabile una richiesta, adeguatamente motivata, di fissare una riunione esterna. A questo punto sarà compito
dello stesso Responsabile giudicare come valida o meno la richiesta presentatagli ed
agire di conseguenza.

\paragraph{Verbale di riunione} \mbox{} \\ \mbox{} \\
Ad ogni riunione, interna o esterna, è compito del \textit{Segretario} designato redigere il \textit{Verbale di riunione} corrispondente, che deve essere poi approvato dal \textit{Responsabile}. La struttura del \textit{Verbale} è definita nella \textbf{((((sezione 1.5.5)))}


\subsection{Strumenti}
Il gruppo, nel corso del progetto, ha utilizzato o utilizzerà i seguenti strumenti:
\begin{itemize}
	\item \textbf{Telegram}: strumento di messaggistica usato per comunicazioni veloci tra i membri; 
	\item \textbf{Discord}: per comunicazioni specifiche o per riunioni interne;
	\item \textbf{Git}: sistema di controllo di versionamento;
	\item \textbf{Gitflow}: sistema per agevolare varie operazioni su Git;
	\item \textbf{GitHub}: per il versionamento e il salvataggio in remoto di tutti i file riguardanti il progetto;
	\item \textbf{GanttProject}: software OpenSource\glo usato per la realizzazione dei diagrammi di Gantt;
	\item \textbf{Draw.io}: applicazione web gratuita per il disegno di diagrammi;
	\item \textbf{Google Drive}: ptilizzato per il salvataggio in remoto dei file non sottoposti a versionamento, in modo da essere reperibili a tutti i membri;
	\item \textbf{Google Calendar}: per tenere traccia delle varie scadenze o riunioni fissate;
	\item \textbf{Skype}: servizio che offre possibilità di fare videoconferenze e chiamate VoIP, utilizzato per comunicare con il proponente;
	\item \textbf{Sistema Operativo}: i requisiti non indicano la necessità di usare un sistema operativo specifico, verranno quindi utilizzati Windows, Linux e Mac OS dai diversi membri del team.
\end{itemize}