\documentclass[a4paper, oneside, openany, dvipsnames, table, 12pt]{article}
\usepackage{../template/AFKstyle}
\usepackage{mathtools}
\usepackage{hyperref}
\newcommand{\Titolo}{Glossario}

\newcommand{\Gruppo}{AFK}

\newcommand{\Redattori}{x \newline y}

\newcommand{\Verificatori}{z}

\newcommand{\pathimg}{../Template/img/logoAFK.png}

\newcommand{\Approvatore}{a}

\newcommand{\Distribuzione}{Prof. Vardanega Tullio \newline Prof. Cardin Riccardo \newline Gruppo AFK}

\newcommand{\Uso}{Esterno}

\newcommand{\NomeProgetto}{Predire con Grafana}

\newcommand{\Mail}{gruppoafk15@gmail.com}

\newcommand{\Versionedoc}{1.0.0}

\newcommand{\DescrizioneDoc}{Raccolta e relativa spiegazione dei termini desueti o specialistici utilizzati nei vari documenti al fine di evitare ogni ambiguità}


\renewcommand\thesection{}

\begin{document}
\copertina{}

\newpage
\tableofcontents
\newpage

\newpage
\section{A}  
\textbf{Ambiente} \\
Struttura in cui un utente, computer o un applicativo opera.

\textbf{Android} \\
Sistema operativo per dispositivi mobile sviluppato da Google LLC e basato sul kernel Linux.

\textbf{API} \\
Le \texttt{API} (acronimo di Application Programming Interface, ovvero Interfaccia di programmazione delle applicazioni) sono set di definizioni e protocolli con i quali vengono realizzati e integrati software applicativi. 

\textbf{Applicazione mobile} \\
\'E un'applicazione software dedicata ai dispositivi di tipo mobile, quali smartphone o tablet, tipicamente progettata e realizzata in maniera più leggera in termini di risorse hardware utilizzate rispetto alle classiche applicazioni per PC, in linea con le restrizioni imposte dalla tipologia di dispositivo su cui è installata e/o eseguita. 

\label{par:appr_auto}
\textbf{Apprendimento automatico} \\ 
Indica un paradigma di erogazione di servizi offerti on demand da un fornitore ad un cliente finale attraverso la rete Internet come: l'archiviazione, l'elaborazione o la trasmissione dati, a partire da un insieme di risorse preesistenti, configurabili e disponibili in remoto.

\textbf{AWS} \\
\texttt{AWS} (acronimo di Amazon Web Services) sono un insieme di servizi di cloud computing sviluppati dall'azienda Amazon. Ne fanno parte: \begin{itemize}
\item ECS (Elastic Container Service ): servizio di orchestrazione dei container completamente gestito.
\item S3: Amazon Simple Storage Service (Amazon S3) è un servizio di storage di oggetti che offre scalabilità, disponibilità dei dati, sicurezza e prestazioni all'avanguardia. Viene utilizzato per archiviare e proteggere una qualsiasi quantità di dati per una vasta gamma di casi d'uso, ad esempio per siti Web, applicazioni per dispositivi mobili, backup e ripristino, archiviazione, applicazioni enterprise, dispositivi IoT e analisi di big data.
\item Rekognition: servizio che facilita l'inserimento di analisi di immagini e video utilizzando una tecnologia di deep learning verificata e altamente scalabile.
\item Sage maker: servizio completamente gestito che consente di creare, formare e distribuire in modo rapido modelli di machine learning (ML).
\item Elastic Transcoder: servizio per la transcodifica di contenuti multimediali nel cloud. Altamente scalabile, viene utilizzato per la conversione (o "transcodifica") di file multimediali dal loro formato originale in versioni differenti che possano essere riprodotte su dispositivi quali smartphone, tablet e PC.
\end{itemize}

\newpage
\section{B}
\label{par:blockC}
\textbf{Blockchain} \\
La blockchain è una sottofamiglia di tecnologie in cui il registro è strutturato come una catena di blocchi contenenti le transazioni e la cui validazione è affidata a un meccanismo di consenso. 
Permette inoltre la creazione e gestione di un grande database distribuito per la gestione di transazioni condivisibili tra più nodi di una rete. 

\textbf{Business logic} \\
\'E la logica di elaborazione, sotto forma di codice sorgente, che gestisce la comunicazione tra un'interfaccia utente e un database.
Essa contiene quindi le informazioni che definiscono o vincolano il modo in cui opera un'azienda.



\newpage
\section{C}
\textbf{Capitolato}\\	
Documento con lo scopo di facilitare e velocizzare la comprensione del contratto d'appalto. Esso descrive specifiche tecniche, caratteristiche generali e modalità di realizzazione degli intenti del committente.

\textbf{Cloud} \\
Indica un paradigma di erogazione di servizi offerti on demand da un fornitore ad un cliente finale attraverso la rete Internet come: l'archiviazione, l'elaborazione o la trasmissione dati, a partire da un insieme di risorse preesistenti, configurabili e disponibili in remoto.

\textbf{Cluster} \\
\'E un insieme di computer connessi tra loro tramite una rete telematica.

\textbf{Command Line Interface(CLI)} \\
\'E un tipo di interfaccia utente caratterizzata da un'interazione testuale tra utente ed elaboratore. L'utente impartisce comandi testuali in input mediante tastiera alfanumerica e riceve risposte testuali in output dall'elaboratore mediante display o stampante alfanumerici. 

\textbf{Consistenza} \\
La consistenza di una transazione, ossia una sequenza di operazioni, è la sua capacità di non violare i vincoli referenziali e di integrità della base di dati.

\label{par:container}
\textbf{Contenitore} \\
\'E una classe di oggetti che è preposta al contenimento di altri oggetti. \\ 
Questi oggetti usualmente possono essere di qualsiasi classe, e possono anche essere a loro volta dei contenitori.

\textbf{Continuous integration} \\
L'integrazione continua (CI) è una pratica che si applica in contesti in cui lo sviluppo del software avviene attraverso un sistema di controllo versione. Consiste nell'allineamento frequente (ovvero "molte volte al giorno") dagli ambienti di lavoro degli sviluppatori verso l'ambiente condiviso (mainline).
In particolare, si suppone generalmente che siano stati predisposti test automatici che gli sviluppatori possono eseguire immediatamente prima di rilasciare i loro contributi verso l'ambiente condiviso, in modo da garantire che le modifiche non introducano errori nel software esistente. Per questo motivo, il CI viene spesso applicato in ambienti in cui siano presenti sistemi di build automatico e/o esecuzione automatica di test.


\textbf{Container} \\
Traduzione in lingua inglese di \hyperref[par:container]{contenitore}.

\label{par:css3}
\textbf{CSS3} \\
\'E un linguaggio utilizzato per descrivere l’aspetto e la formattazione di un sito web. \\
Il CSS viene più comunemente usato nei documenti \texttt{HTML} e \texttt{XHTML}, ma applicabile anche ai documenti XML. 

\newpage
\section{D}
\label{par:db}
\textbf{Database} \\
Archivio di dati strutturato in modo da razionalizzare la gestione e l'aggiornamento delle informazioni e da permettere lo svolgimento di ricerche complesse. \\
I dati all'interno dei tipi più comuni di database attualmente in funzione vengono generalmente presentati in righe e colonne contenute in una serie di tabelle per garantire l'efficienza di elaborazione e query dei dati. Tali dati possono poi essere facilmente visualizzati, gestiti, modificati, aggiornati, controllati e organizzati.

\textbf{Deep Learning} \\
Per Deep Learning (o apprendimento automatico) si intende un insieme di tecniche basate su reti neurali artificiali organizzate in diversi strati, dove ogni strato calcola i valori per quello successivo affinché l'informazione venga elaborata in maniera sempre più completa.

\newpage
\section{E}
\textbf{Efficacia} \\
Capacità di raggiungere l'obiettivo prefissato.

\textbf{Efficienza} \\
Misura l'abilità di raggiungere l'obiettivo prefissato impiegando le risorse minime indispensabili.

\textbf{Ethereum} \\
\'E una piattaforma \hyperref[par:opens]{open source} globale per applicazioni decentralizzate. Lanciata nel 2015, Ethereum è la \hyperref[par:blockC]{blockchain} programmabile più utilizzata al mondo.

\newpage
\section{F}
\textbf{Framework} \\
Piattaforma che funge da strato intermedio tra un sistema operativo e il software che lo utilizza.

\textbf{Front-end} \\
Programma con il quale l’utente interagisce direttamente, che fa da interfaccia tra l’utente e un altro programma.

\newpage
\section{G}
\textbf{Grafana} \\
Piattaforma open-source che consente di monitorare i dati provenienti da diverse sorgenti, attraverso una loro rappresentazione grafica all'interno di una
dashboard.

\textbf{GUI} \\
Graphical User Interface – tradotto interfaccia utente, consente l’interazione uomo-macchina in modo visuale utilizzando rappresentazioni grafiche piuttosto che utilizzando una interfaccia a riga di comando.

\newpage
\section{H}
\textbf{HTML5}\\	
\`E linguaggio di markup nato per la formattazione e impaginazione di documenti ipertestuali disponibili nel web 1.0, oggi è utilizzato principalmente per il disaccoppiamento della struttura logica di una pagina web (definita appunto dal markup) e la sua rappresentazione, gestita tramite gli stili \hyperref[par:css3]{CSS} per adattarsi alle nuove esigenze di comunicazione e pubblicazione all'interno di Internet.

\newpage
\section{I}
\textbf{IOS} \\
Sistemo operativo per dispositivi mobile sviluppato da Apple.

\newpage
\section{J}
\textbf{JavaScript}\\	
Linguaggio di scripting orientato agli oggetti e agli eventi, comunemente utilizzato nella programmazione Web lato client per la creazione, in siti web e applicazioni web, di effetti dinamici interattivi tramite funzioni di script invocate da eventi innescati a loro volta in vari modi dall'utente sulla pagina web in uso.

\textbf{JSON} \\
JavaScript Object Notation, formato utilizzato per il trasporto dati, pensato
per una facile scrittura e lettura per un umano e una facile interpretazione per
le macchine che lo processano.

\newpage
\section{K}

\newpage
\section{L}
\textbf{Libreria} \\
Insieme di funzioni o strutture dati predefinite e predisposte per essere collegate ad un software. 

\textbf{Linguaggio di programmazione} \\
\'E un linguaggio formale che specifica un insieme di istruzioni che, a partire da un insieme di dati di input, produca un insieme di dati di output.

\textbf{Linguaggio naturale} \\
\'E il linguaggio usato nella comunicazione tra persone che condividono la stessa lingua.

\newpage
\section{M}
\textbf{Machine Learning} \\
Sinonimo di \hyperref[par:appr_auto]{apprendimento automatico}.

\textbf{Manutenibilità} \\
Capacità del software di essere modificato, includendo correzioni, miglioramenti o adattamenti (ISO/IEC 9126).

\newpage
\section{N}
\textbf{NodeJS}\\	
\`E una piattaforma Open source che permette di eseguire codice JavaScript Server-side, oltre che Client-side. Tale sistema si basa su un'architettura event-driven, in cui le operazioni sono subordinate alla ricezione di una notifica.

\textbf{Node Package Manager(NPM)} \\
\'E un gestore di pacchetti per il linguaggio di programmazione JavaScript.
\'E composto da un client da linea di comando e un database di pacchetti pubblici e privati, chiamato \textit{npm registry}. 

\newpage
\section{O}
\label{par:opens}
\textbf{Open source} \\
Software non protetto da copyright e liberamente modificabile dagli utenti.

\newpage
\section{P}
\textbf{Proof-of-concept} \\
Realizzazione incompleta o abbozzata di un determinato progetto o metodo, allo scopo di provarne la fattibilità o dimostrare la fondatezza di alcuni principi o concetti fondamentali.

\textbf{Python} \\
Python \'e un linguaggio di programmazione dinamico orientato agli oggetti utilizzabile per molti tipi di sviluppo software come ad esempio il web, il data analysis e il machine learning.

\newpage
\section{Q}

\newpage
\section{R}
\textbf{Real time} \\
Tempo effettivo durante il quale si verifica un processo o un evento. \\
In informatica, usato per descrivere il modo in cui un sistema informatico riceve i dati e li comunica o li rende immediatamente disponibili.

\textbf{Regressione lineare} \\
Rappresenta un metodo di stima del valore atteso condizionato di una variabile dipendente, dati i valori di altre variabili indipendenti. Si definisce retta di regressione la retta: $y = \alpha x + \beta $

\textbf{Requisito}\\	Particolare proprietà richiesta necessaria per conseguire uno scopo. Nello specifico, si categorizzano in:
\begin{itemize}
	\item \textbf{Requisito di fidatezza:} requisito di sistema che è incluso per aiutare a raggiungere un livello di fidatezza richiesto per il sistema;
	\item \textbf{Requisito funzionale:} definizione di una funzione o caratteristica che deve essere implementata in un sistema;
	\item \textbf{Requisito non funzionale:} vincolo o comportamento atteso che si applica a un sistema. Può riferirsi sia alle proprietà principali del software che sto sviluppando sia al processo di sviluppo.
\end{itemize}

\textbf{REST} \\
Representational State Transfer. Sistema di trasmissione di dati su HTTP privo di livelli e sessione. Viene impiegato come architettura software nei sistemi distribuiti.

\newpage
\section{S}
\textbf{Sistema distribuito} \\
Indica una tipologia di sistema informatico costituito da un insieme di processi interconnessi tra loro in cui le comunicazioni avvengono solo esclusivamente tramite lo scambio di opportuni messaggi.

\textbf{Smart Contract Technology} \\
Un programma che esegue inevitabilmente le condizioni postulate precedententemente
dagli sviluppatori. In pratica un contratto tradizionale i cui effetti sono
garantiti da un codice, nel caso degli smart contract di Ethereum, scritti nel
linguaggio Solidity.

\textbf{Stakeholder} \\
Ciascuno dei soggetti direttamente o indirettamente coinvolti in un progetto.

\textbf{Streaming} \\
Flusso di dati audio/video riprodotti in real-time, trasmessi da una sorgente a una o più destinazioni tramite una rete telematica.

\textbf{Suite} \\
Insieme di applicazioni svilippate dallo stesso produttore.

\textbf{Support Vector Machine (SVM)} \\
Inventata da V. Vapnik nel 1990, è un modello di apprendimento automatico supervisionato che utilizza algoritmi di classificazione per valutare specifici problemi. Questi algoritmi trovano impiego nell'ambito del machine learning.

\newpage
\section{T}
\textbf{Typescript} \\
Linguaggio di programmazione libero ed Open source sviluppato da Microsoft. Si tratta di un Super-set di JavaScript che basa le sue caratteristiche su ECMAScript6.

\newpage
\section{U}

\newpage
\section{W}
\textbf{Way of Working} \\
Metodo di lavoro che standardizza le attività di progetto in modo da renderele sistematiche, disciplinate e quantificabili.
\newpage
\section{X}

\newpage
\section{Y}

\newpage
\section{Z}



\end{document}