\usepackage{comment}
\section{Informazioni generali}
Nella stessa giornata si sono tenuti due incontri: il primo in mattinata e il secondo nel tardo pomeriggio.
\subsection{Informazioni primo incontro}
\begin{itemize}
\item \textbf{Luogo}: Discord\glo;
\item \textbf{Data}: 2020-03-18;
\item \textbf{Ora di inizio}: 10:30;
\item \textbf{Ora di fine}: 12:15;
\item \textbf{Partecipanti}: tutti i membri.
\end{itemize}

\subsubsection{Topic}

\begin{itemize}
\item Scelta logo;
\item Suddivisione Compiti;
\item Impostate baseline (scadenza norme, analisi, consegna);
\item Tecnologia di VCS (GitHub);
\item Tecnologia di scrittura documenti (LaTeX);
\item Programmazione futuri colloqui con proponente.
\end{itemize}

Come ultimo tema di discussione dell'incontro, sono state raccolte alcune domande, riguardanti diversi aspetti del prodotto, da porre al proponente in attesa del primo colloquio disponibile con lo stesso. 

\begin{comment}
\begin{itemize}
\item pianificazione dei task in vista del RR\glo ;
\item discussione dei ruoli ;
\item configurazione degli strumenti di condivisione:
\begin{itemize}
\item GitHub repository;
\item GitKraken;
\item documento condiviso Google Docs;
\end{itemize}
\item creazione template LaTex;
\item discussione struttura \textbf{Studio di Fattibilità};
\item stilato elenco di domande da porre al proponente ri guardo al prodotto;
\begin{itemize}
\item obiettivo;
\item funzioni;
\item caratteristiche;
\item macro architetture presenti;
\item vincoli;
\end{itemize}
\end{itemize}
\end{comment}

\pagebreak
