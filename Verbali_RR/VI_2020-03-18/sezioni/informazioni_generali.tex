\section{Informazioni generali}
Nella stessa giornata si sono tenuti due incontri: il primo in mattinata e il secondo nel tardo pomeriggio.
\subsection{Informazioni primo incontro}
\begin{itemize}
\item \textbf{Luogo}: Discord\glo;
\item \textbf{Data}: 2020-03-18;
\item \textbf{Ora di inizio}: 10:30;
\item \textbf{Ora di fine}: 12:15;
\item \textbf{Partecipanti}: tutti i membri.
\end{itemize}

\subsubsection{Topic}
\begin{itemize}
\item scelto il canale di comunicazione principale: Discord;
\item struttura dello \textit{Studio di Fattibilità}; ogni capitolato deve seguire la seguente struttura: \begin{enumerate}
\item Descrizione generale;
\item Obiettivi;
\item Tecnologie utilizzate;
\item Valutazione generale: descrizione aspetti positivi, negativi e valutazione finale;
\end{enumerate}
\item impostato il \textit{Glossario}.
\end{itemize}

L'ultima parte dell'incontro ha riguardato la scelta del capitolato da aggiudicarsi: ogni componente ha esposto la propria preferenza, indicandone gli aspetti positivi e negativi. Questo confronto ha indirizzato il gruppo verso il capitolato C4, ovvero \textit{Predire in Grafana}.
\pagebreak

\subsection{Informazioni secondo incontro}
\begin{itemize}
\item \textbf{Luogo}: Discord;
\item \textbf{Data}: 2020-03-18;
\item \textbf{Ora di inizio}: 16:30;
\item \textbf{Ora di fine}: 19:00;
\item \textbf{Partecipanti}: tutti i membri.
\end{itemize}

\subsubsection{Topic}
\begin{itemize}
\item bozza dello \textit{Studio di Fattibilità} dei capitolati 1-2-3;
\item inizio della stesura del \textit{Glossario}.
\end{itemize}