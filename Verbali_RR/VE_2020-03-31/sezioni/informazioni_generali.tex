\section{Informazioni generali}
\subsection{Informazioni incontro}
\begin{itemize}
\item \textbf{Luogo}: Skype;
\item \textbf{Data}: 2020-03-31;
\item \textbf{Ora di inizio}: 14:30;
\item \textbf{Ora di fine}: 15:00;
\item \textbf{Partecipanti}:
	\begin{itemize}
		\item tutti i membri;
		\item Gregorio Piccoli (proponente)
	\end{itemize}
\end{itemize}

\subsection{Topic}
Durante la stesura del documento \textit{Analisi dei Requisiti} sono sorti alcuni dubbi inerenti ai casi d'uso dell'applicazione. Nell'incontro si sono discussi questi dubbi.\\
In particolare:
\begin{itemize}
	\item come distinguere le classificazioni della SVM;
	\item come ottenere un valore dalla RL;
	\item salvataggio e composizione dei file JSON;
	\item ruolo del database di Grafana InfluxDB;
	\item corretto utilizzo dei predittori;
	\item salvataggio dei file di addestramento;
	\item etichettamento dei dati di addestramento per suddividerli in dati "buoni" e dati "cattivi" (true positive, false positive, true negative, false negative);
	\item come può essere applicato l'apprendimento continuo al machine learning.
\end{itemize}
Il proponente ha così fornito delucidazioni sottolineando che:
\begin{itemize}
\item \textbf{Support Vector Machine}: a seconda del segno del risultato ottenuto dall'applicazione della SVM, il predittore apparterrà ad una classe piuttosto che un altra;
\item \textbf{Regressione Lineare}: l'applicazione della regressione lineare produce un risultato singolo;
\item \textbf{File JSON}: questi file devono essere strutturati in modo tale da contenere una sezione comune tra i file JSON usati dall'applicativo, e altra sezione dedicata a RL o SVM;
\item \textbf{InfluxDB}: essendo il database su cui si appoggia l'applicazione, deve essere utilizzato, tramite l'interfaccia di Grafana, per la lettura dei dati su cui fare previsioni, e per la storicizzazioni delle stesse;
\item \textbf{Addestramento}: i dati di addestramento devono essere simulati e salvati in file di estensione .csv\glo , un plug in effettua l'addestramento, viene stabilito il preditore, e riaddestrato il sistema;
\item \textbf{Predittore}: un flusso di dati può avere più di un predittore, ma un predittore è associabile ad un solo flusso;
\item \textbf{Auto apprendimento}: per questa operazione, nel caso venisse utilizzata la Regressione Lineare, è consigliato applicarla in modo continuo, invece nel caso dell'uso di SVM, quest'ultima deve essere completamente riaddestrata.
\end{itemize}