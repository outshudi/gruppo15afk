\section{Informazioni generali}
\subsection{Informazioni incontro}
\begin{itemize}
\item \textbf{Luogo}: Skype;
\item \textbf{Data}: 2020-03-31;
\item \textbf{Ora di inizio}: 14:30;
\item \textbf{Ora di fine}: 15:00;
\item \textbf{Partecipanti}:
	\begin{itemize}
		\item tutti i membri;
		\item Gregorio Piccoli (proponente)
	\end{itemize}
\end{itemize}

\subsection{Topic}
Durante la stesura del documento \textit{Analisi dei Requisiti} sono sorti alcuni dubbi inerenti ai casi d'uso dell'applicazione. Nell'incontro si sono discussi questi dubbi.\\
In particolare:
\begin{itemize}
	\item come distinguere le classificazioni della SVM;
	\item come ottenere un valore dalla RL;
	\item salvataggio e composizione dei file JSON;
	\item ruolo del database di Grafana InfluxDB;
	\item corretto utilizzo dei predittori;
	\item salvataggio dei file di addestramento;
	\item etichettamento dei dati di addestramento per suddividerli in dati buoni e dati cattivi (true positive, false positive, true negative, false negative);
	\item come può essere applicato l'apprendimento continuo al machine learning.
\end{itemize}