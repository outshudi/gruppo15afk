\documentclass[a4paper, oneside, openany, dvipsnames, table, 12pt]{article}
\usepackage{../../Template/AFKstyle}
\usepackage{hyperref}
\usepackage{verbatim} %per commenti di più righe \begin{comment} \end{comment}
\usepackage{amsmath}
\newcommand{\Titolo}{Verbale esterno 2020-03-31}

\newcommand{\Gruppo}{TeamAFK}

\newcommand{\Redattori}{Simone Meneghin}

\newcommand{\Verificatori}{z}

\newcommand{\pathimg}{../../Template/img/logoAFK.png}

\newcommand{\Approvatore}{a}

\newcommand{\Distribuzione}{Prof. Vardanega Tullio \newline Prof. Cardin Riccardo \newline TeamAFK}

\newcommand{\Uso}{Esterno}

\newcommand{\NomeProgetto}{"Predire in Grafana"}

\newcommand{\Mail}{gruppoafk15@gmail.com}

\newcommand{\Versionedoc}{1.0.0}

\newcommand{\DescrizioneDoc}{Riassunto dell'incontro del gruppo \textit{TeamAFK} con il proponente tenutosi il 2020-03-31.}


\makeindex

\begin{document}
\copertina{}

%------------------ COLORI TABELLE 
\definecolor{pari}{RGB}{255, 207, 158} %{HTML}{E1F5FE} %azzurrino
\definecolor{dispari}{HTML}{FAFAFA} %bianco/grigetto 

%definizione colori per tabelle (tranne copertina)
\definecolor{redafk}{RGB}{255, 133, 51}
\definecolor{grey2}{RGB}{204, 204, 204}
\definecolor{greyRowafk}{RGB}{234, 234, 234}
\definecolor{lastrowcolor}{RGB}{176, 196, 222} %steel blue %{255,165,0} orange %{RGB}{255, 207, 158}
\rowcolors{2}{pari}{dispari}
\renewcommand{\arraystretch}{1.5}

%------------------

\newpage
\section*{Registro delle modifiche}
{
	\centering
	\begin{longtable}{ c c  C{4cm}  c  c }
		\rowcolor{redafk}
		\textcolor{white}{\textbf{Versione}} & \textcolor{white}{\textbf{Data}} & \textcolor{white}{\textbf{Descrizione}} & \textcolor{white}{\textbf{Nominativo}} & \textcolor{white}{\textbf{Ruolo}}\\		
		0.0.1 & 2020-03-20 & Stesura documento & Davide Zilio &\reda{}\\		
		
	\end{longtable}

}


%Didascalia tabelle/immagini (prendono come riferimento la subsection)
\counterwithin{table}{subsection}
\counterwithin{figure}{subsection}
\newpage

%indice, indice figure e indice tabelle
\tableofcontents
\newpage
\begin{comment}
\listoffigures
\newpage
\listoftables
\newpage
\end{comment}

\section{Informazioni generali}
\subsection{Informazioni primo incontro}
\begin{itemize}
\item \textbf{Luogo}: Zoom\glo;
\item \textbf{Data}: 2020-03-11;
\item \textbf{Ora di inizio}: 10:30;
\item \textbf{Ora di fine}: 12:15;
\item \textbf{Partecipanti}: tutti i membri.
\end{itemize}

\subsection{Topic}
Gli argomenti trattati in questo incontro sono:
\begin{itemize}
	\item presentazione di ciascun membro del gruppo;
	\item scelta di un nome per il team;
	\item creazione dell'email di gruppo;
	\item presa visione di tutti i capitolati disponibili, esponendo le proprie preferenze per la scelta.
\end{itemize}

\subsection{Note}
Questo incontro è stato il primo incontro del gruppo, essendo tale non avevamo ancora deciso i ruoli dei membri e le norme da applicare per cui questo documento è stato scritto in una data successiva.
\pagebreak

\end{document}