\section{Standard di qualità}
Lo standard che definisce la qualità del software è \textbf{ISO/IEC 9126}. \\
Con la sigla ISO/IEC 9126 si individua una serie di normative e linee guida, sviluppate dall'ISO (\textit{Organizzazione internazionale per la normazione}) in collaborazione con l'IEC (\textit{Commissione Elettrotecnica Internazionale}), preposte a descrivere un modello di qualità del software. \\
La norma tecnica relativa alla qualità del software si compone in quattro parti:
\begin{itemize}
	\item Modello della qualità del software
	\item Metriche per la qualità esterna
	\item Metriche per la qualità interna
	\item Metriche per la qualità in uso
\end{itemize}
\subsection{Modello della qualità del software}
Il modello di qualità è classificato da sei caratteristiche generali e da varie sottocaratteristiche misurabili attraverso delle metriche.\\
Di seguito vengono definite tali caratteristiche.

\subsubsection{Funzionalità}
\'E la capacità del software di soddisfare determinate esigenze stabilite nell' \textit{Analisi dei Requisiti}, necessarie per operare sotto condizioni specifiche. Nello specifico il software deve soddisfare le seguenti caratteristiche:
\begin{itemize}
	\item \textbf{Appropriatezza}: le funzioni fornite dal software sono appropriate per i compiti ed obiettivi prefissati;
	\item \textbf{Accuratezza}: il software deve fornire risposte concordanti o i precisi effetti richiesti;
	\item \textbf{Interoperabilità}: capacità di interagire ed operare con uno o più sistemi specificati;
	\item \textbf{Conformità}: il prodotto deve essere in grado di aderire a standard, convenzioni e regolamentazioni;
	\item \textbf{Sicurezza}: è la capacità del software di proteggere informazioni e dati, negandone l'accesso a persone o sistemi non autorizzati e invece fornirlo a chi ne è effettivamente autorizzato.
\end{itemize}
\subsubsection{Affidabilità}
\'E la capacità del prodotto software di mantenere uno specificato livello di prestazioni quando usato in date condizioni. Nello specifico il software deve soddisfare le seguenti caratteristiche:
\begin{itemize}
	\item \textbf{Maturità}: capacità di un prodotto software di evitare che si verificano errori, malfunzionamenti o siano prodotti risultati non corretti;
	\item \textbf{Tolleranza agli errori}: il software deve mantenere dei predeterminati livelli di prestazioni anche in caso di un uso scorretto o di malfunzionamenti;
	\item \textbf{Recuperabilità}: capacità di ripristinare il livello appropriato di prestazioni a seguito di un malfunzionamento o di un errore. La caratteristica di recuperabilità è valutata dall'arco di tempo nel quale il software può risultare non accessibile;
	\item \textbf{Aderenza}: è la capacità di aderire a standard, regole e convenzioni riguardanti all'affidabilità.
\end{itemize}
\subsubsection{Efficienza}
Capacità del prodotto software di eseguire le proprie funzioni minimizzando il tempo necessario e sfruttando al meglio le risorse che necessita.
Nello specifico il software deve soddisfare le seguenti caratteristiche:
\begin{itemize}
	\item \textbf{Comportamento rispetto al tempo}: è la capacità di fornire adeguati tempi di risposta, elaborazione e velocità di attraversamento, sotto condizioni determinate;
	\item \textbf{Utilizzo delle risorse}: è la capacità di utilizzo di quantità e tipo di risorse in maniera adeguata;
	\item \textbf{Conformità}: è la capacità di aderire a standard e specifiche sull'efficienza.
\end{itemize}
\subsubsection{Usabilità}
\'E la capacità del prodotto software di essere capito, appreso, usato e ben accettato dall'utente, quando usato sotto condizioni specificate.
Nello specifico il software deve soddisfare le seguenti caratteristiche:
\begin{itemize}
	\item \textbf{Comprensibilità}: esprime la facilità di comprensione dei concetti, funzionalità e utilizzo da parte dell'utente;
	\item \textbf{Apprendibilità}: è la capacità di ridurre l'impegno richiesto agli utenti per imparare ad usare l'applicazione;
	\item \textbf{Operabilità}: capacità di mettere in condizione gli utenti di farne uso per i propri scopi e controllarne l'uso;
	\item \textbf{Attrattivà:} è la capacità del software di essere piacevole all'uso per l'utente;
	\item \textbf{Conformità}: è la capacità del software di aderire a standard o convenzioni relativi all'usabilità.

\end{itemize}
\subsubsection{Manutenibilità}
La manutenibilità è la capacità del software di essere modificato, includendo correzioni, miglioramenti o adattamenti.
Nello specifico il software deve soddisfare le seguenti caratteristiche:
\begin{itemize}
	\item \textbf{Analizzabilità}: rappresenta la facilità con la quale è possibile analizzare il codice per localizzare un errore nello stesso;
	\item \textbf{Modificabilità}: la capacità del prodotto software di permettere l'implementazione di una modifica;
	\item \textbf{Stabilità}: capacità del software di evitare effetti inaspettati a seguito di modifiche errate;
	\item \textbf{Testabilità}: la capacità di essere facilmente testato per validare le modifiche apportate al software.
\end{itemize}
\subsubsection{Portabilità}
La portabilità è la capacità del software di essere trasportato da un ambiente di lavoro ad un altro. Ambiente che può variare dall'hardware al software.
Nello specifico il software deve soddisfare le seguenti caratteristiche:
\begin{itemize}
	\item \textbf{Adattabilità}: la capacità del software di essere adattato per differenti ambienti senza dover applicare modifiche;
	\item \textbf{Installabilità}: capacità del software di essere installato in un diverso ambiente;
	\item \textbf{Conformità}: la capacità del prodotto software di aderire a standard e convenzioni relative alla portabilità;
	\item \textbf{Sostituibilità}: capacità di essere utilizzato al posto di un altro software per svolgere gli stessi compiti nello stesso ambiente.
\end{itemize}
\subsection{Metriche per la qualità esterna}
Servono a misurare i comportamenti del software sulla base dei test, in funzione degli obiettivi stabiliti. Viene rilevata tramite analisi dinamica.

\subsection{Metriche per la qualità interna}
Metriche che si applicano al software non eseguibile durante le fasi di progettazione e codifica. Permettono di individuare eventuali problemi che potrebbero influire sulla qualità finale del prodotto prima che venga realizzato un eseguibile. Grazie alle misure effettuate tramite le metriche interne è possibile prevedere il livello di qualità esterna e di qualità in uso del prodotto finale, poiché entrambe vengono influenzate dalla qualità interna.\\
Viene rilevata tramite analisi statica.

\subsection{Metriche per la qualità in uso}
La qualità in uso rappresenta il punto di vista dell'utente sul software. Il livello di qualità viene raggiunto dopo che sono stati raggiunti i livelli nella qualità esterna che interna. La qualità in uso, quindi, permette di abilitare specificati utenti ad ottenere specificati obiettivi con efficacia, produttività, sicurezza e soddisfazione.
