\section{Introduzione}
\subsection{Premessa}
Il \textit{Piano di Qualifica} è un documento su cui si prevede di lavorare l'intera durata del progetto. Molti dei contenuti del documento sono di natura instabile. Ad esempio, molte delle metriche scelte non sono applicabili nella fase iniziale, e solo con il loro utilizzo pratico si può valutarne l'effettiva utilità. Anche i processi selezionati possono essere soggetti a cambiamenti, rivelandosi insufficienti o inadeguati agli scopi del progetto e al modo di lavorare del team. Il documento è stato scritto in diversi periodi in quanto alcune delle cose non le si potevano conoscere a priori. \\
Per tutte queste ragioni, il documento è prodotto in maniera incrementale, e i suoi contenuti iniziali sono da considerarsi incompleti: subiranno significative aggiunte e modifiche nel tempo.

\subsection{Scopo del documento}
Questo documento ha lo scopo di mostrare le strategie di verifica\glo e validazione\glo adottate al fine di garantire la qualità di prodotto e di processo. Per raggiungere questo obiettivo viene applicato un sistema di verifica continua sui processi in corso e sulle attività svolte. In questo modo è quindi possibile rilevare e correggere all'istante eventuali anomalie, riducendo al minimo lo spreco delle risorse.

\subsection{Scopo del prodotto}
Lo scopo del prodotto è quello di realizzare due plug-in per il software Grafana\glo, che permettano di monitorare e predire lo stato di un sistema in analisi. Grazie alle predizioni sarà possibile attivare degli allarmi così da poter gestire preventivamente eventuali situazioni di rischio. \\
I due plug-in\glo utilizzeranno la Support Vector Machine\glo (SVM) per poter effettuare regressione lineare o categorizzazione sui dati forniti.

\subsection{Glossario}
Per evitare ambiguità nei documenti formali, viene fornito il documento \textbf{Glossario},
contenente tutti i termini considerati di difficile comprensione. Perciò nella documentazione fornita, ogni vocabolo contenuto in Glossario è contrassegnato dalla lettera G a pedice.

\subsection{Riferimenti}
\subsubsection{Riferimenti normativi}
\begin{itemize}
	\item Norme di Progetto: \textit{Norme_di_Progetto_v1.1.0};
	\item ISO/IEC 9126: \url{https://en.wikipedia.org/wiki/ISO/IEC_9126};
	\item ISO/IEC 15504: \url{https://en.wikipedia.org/wiki/ISO/IEC_15504}.
\end{itemize}
\subsubsection{Riferimenti informativi}
\begin{itemize}
	\item Capitolato d'appalto C4: \url{https://www.math.unipd.it/~tullio/IS-1/2019/Progetto/C4.pdf}.
\end{itemize}