\section{Specifica dei test}
Per verificare la qualità del prodotto software, il gruppo fornitore ha deciso di adottare il \textbf{Modello di Sviluppo a V}\glo, sviluppando così una serie di test. Questi hanno lo scopo di controllare che tutte le unità di cui è composto il sistema, siano state implementate correttamente, rispettando tutti gli aspetti del progetto.
Per seplificare la loro consultazione, i test saranno suddivisi in categorie, per mezzo di tabelle, mostrando l'output prodotto, e sottolineando se è un risultato atteso o non atteso.
\subsection{Stato dei test}
Per definire lo stato dei test, si usano le seguenti sigle:
\begin{itemize}
\item \textbf{I}: test implementato;
\item \textbf{NI}: test non implementato.
\end{itemize}
\subsection{Tipologie di test}
I test implementati si possono istinguere in:
\begin{itemize}
\item \textbf{Test di Accettazione[TA]}: test eseguiti per verificare che il prodotto soddisfi le aspettatve del proponente;
\item \textbf{Test di Sistema[TS]}: test effettuato controllare se il sistema, una volta installato su una piattaforma, rispetti le richieste e gli obiettivi  preventivati;
\item \textbf{Test Integrazione[TI]}: questi test si accertano che sottoinsiemi dei moduli di cui è formato il sotware interagiscano correttamente tra di loro. Si effettuano prima dei Test di Sistema e dopo i Test di unità;
\item \textbf{Test di Unità[TU]}: verificano che i singoli programmi del sistema siano in grado di garantire un funzinamento autonomo. Da notare che il superamento del test da parte di un solo componente non garantisce il corretto funzionamento del sistema.
\end{itemize}
