\section{Qualifica di prodotto}
\subsection{Scopo}
Per stabilire la Qualità di prodotto, il team di Quality Management\glo adotta come standard di riferimento \textbf{ISO/IEC 9126} per stabilire il modello della qualità del software. Per stabilire il raggiungimento di un determinato obiettivo di qualità, ogni voce trattata è accompagnata da un apposito parametro.
\subsection{Obiettivi}
Gli obiettivi di qualità che il team di Quality Management vuole raggiungere sono:
\begin{itemize}
\item Affidabilità;
\item Usabilità.
\end{itemize}
\subsection{Metriche della documentazione}
In relazione agli obiettivi prestabiliti, il team adotta i diversi strumenti per misurare la qualità del prodotto, riportati di seguito.
\subsubsection{MD01 - Indice di Gulpease}
L'Indice di Gulpease registra la leggibilità di un documento. \\ \\ 
\textbf{Parametri adottati:} 
\begin{itemize}
\item Range accettabile: [60, 80);
\item Range ottimale: [80, 100].
\end{itemize}
\subsection{Metriche del codice sorgente}
\subsubsection{MS01 - Linee di Codice}
\'E la metrica che registra la dimensione di tutto il codice sorgente.
\subsubsection{MS02 - Numero dei Metodi}
Questa metrica conteggia il numero di metodi presenti nella classe di un oggetto.\\ \\ 
\textbf{Parametri adottati:} 
\begin{itemize}
\item Range accettabile: [0, 15];
\item Range ottimale: [0, 8].
\end{itemize}
\subsubsection{MS03 - Numero di Parametri}
Questo strumento tiene conto del numero di parametri formali di un metodo.\\ \\ 
\textbf{Parametri adottati:} 
\begin{itemize}
\item Range accettabile: [0, 6];
\item Range ottimale: [0, 3].
\end{itemize}
\subsubsection{MS04 - Commenti per Linee di Codice}
\'E il rapporto tra le rige di commento e il codice effettivo.\\ \\ 
\textbf{Parametri adottati:} 
\begin{itemize}
\item Range accettabile: [0.05, 0.10);
\item Range ottimale: [0.10, 0.20].
\end{itemize}
\subsubsection{MS05 - Code Coverage}
\'E la metrica con il compito di misurare l'indice di copertura del codice da parte dei test in termini percentuali.\\ \\ 
\textbf{Parametri adottati:} 
\begin{itemize}
\item Range accettabile: [70, 80)\%;
\item Range ottimale: [80, 100]\%.
\end{itemize}
Seppur l'obiettivo del team di sviluppo sia quello di avere una Code Coverage del 100\%, tale traguardo potrebbe non essere raggiunto, in quanto comporterrebbe un aumento dei costi di progetto troppo elevati.