\section{Qualifica di prodotto}
\subsection{Scopo}
Per stabilire la Qualità di prodotto, il team di Quality Management\glo adotta come standard di riferimento \textbf{ISO/IEC 9126} per stabilire il modello della qualità del software. Per stabilire il raggiungimento di un determinato obiettivo di qualità, ogni voce trattata è accompagnata da un apposito parametro.
\subsection{Obiettivi}
Gli obiettivi di qualità che il team di Quality Management vuole raggiungere sono:
\begin{itemize}
\item Affidabilità;
\item Usabilità.
\end{itemize}
\subsection{Metriche}
In relazione agli obiettivi prestabiliti, il team adotta i seguenti strumenti per misurare la qualità del prodotto 
\begin{itemize}
\item \textbf{Linee di Codice}: riportata con la sigla \textbf{MS01}, è la metrica che registra la dimensione di tutto il codice sorgente;
\item \textbf{Numero dei Metodi}: con la sigla \textbf{MS02}, conteggia il numero di metodi presenti nella classe di un oggetto:
	\begin{itemize}
	\item valore ottimale: $\leq$ 8;
	\item valore accettabile: $\leq$ 15;
	\end{itemize}
\item \textbf{Numero parametri}: riporta il numero di parametri di un metodo con il codice \textbf{MS03}
	\begin{itemize}
	\item valore ottimale: $\leq$ 3;
	\item valore accettabile: $\leq$ 6.
	\end{itemize}
\item \textbf{Commenti per Linee di Codice}: riportato con la sigla \textbf{MS04}, è lo strumento che rileva il rapporto tra le rige di commento e il codice effettivo, sottolineando che:
	\begin{itemize}
		\item valore ottimale: $\geq$ 0.20;
		\item valore accettabile: $\geq$ 0.10.
	\end{itemize}
\item \textbf{Code Coverage}: è la metrica con il codice \textbf{MS05} ed ha il compito di misurare l'indice di copertura del codice da parte dei test.

\end{itemize}

\begin{comment}
\subsection{Metriche}
In relazione agli obiettivi prestabiliti, il team adotta le seguenti metriche per misurare la qualità del prodotto.

\subsubsection{MS01 - Linee di Codice}
\subsubsection{MS02 - Numero dei Metodi}
\begin{itemize}
\item Range accettabile: [0, 15];
\item Range ottimale: [0, 8].
\end{itemize}
\subsubsection{MS03 - Numero di parametri}
\begin{itemize}
\item Range accettabile: [0, 6];
\item Range ottimale: [0, 3].
\end{itemize}
\subsubsection{MS04 - Commenti per Linee di Codice}
\begin{itemize}
\item Range accettabile: $\geq 0.20$;
\item Range ottimale: $\geq 0.10$.
\end{itemize}
\subsubsection{MS05 - Code Coverage}
\begin{itemize}
\item Range accettabile: [85-95]\%;
\item Range ottimale: 100\%.
\end{itemize}

\end{comment}
