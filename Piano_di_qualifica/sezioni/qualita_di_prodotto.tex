\section{Qualifica di prodotto}
Per stabilire la Qualità di prodotto, il team di Quality Management\glo adotta come standard di riferimento \textbf{ISO/IEC 9126} per stabilire il modello della qualità del software. Per stabilire il raggiungimento di un determinato obiettivo di qualità, ogni voce trattata è accompagnata da un apposito parametro. Nello specifico:
\subsection{Funzionalità}
Con funzionalità si intendono le capacità del prodotto di fornire funzioni che soddisfano le richieste stabilite, e possono essere analizzate secondo diversi aspetti. 
\paragraph{Obiettivi}
\begin{itemize}
	\item \textbf{Appropriatezza}: abilià del software nel fornire un insieme di funzioni conveniente svolgere le operazioni richieste;
	\item \textbf{Accuratezza}: le azioni svolte dal software riportano i risultati attesi;
	\item \textbf{Conformità}: il prodotto aderisce agli standard del settore presi come riferimento;
	\item \textbf{Sicurezza}: capacità dell'applicativo nel proteggere le informazioni, impedendo che sistemi esterni possano entrarne in possesso.
\end{itemize}
\paragraph{Metriche}
\begin{itemize}
	\item ;
\end{itemize}
\subsection{Affidabilità}
In questa sottosezione vengono presi in considerazione tutti gli aspetti che permettono di stabilire il grado di fiducia che si può affidare al software in sutuazioni difficili che richiedono alte prestazioni.
\paragraph{Obiettivi}
\begin{itemize}
	\item \textbf{Maturità}: capacità del software nell'evitare la produzione di errori;
	\item \textbf{Tolleranza agli errori}: abilità del prodotto nel sopportare e matenere i carichi prestazionali aspettati in presenza di malfunzionamenti;
	\item \textbf{Recuperabilità}: capacità nel ristabilire le prestazioni e nel recuperare 	informazioni rilevanti;
	\item \textbf{Aderenza}: ovvero aderenza agli standard.
\end{itemize}
\paragraph{Metriche}
\subparagraph{Code Coverage}
Questo parametro ha il compito di misurare l'indice di copertura del codice da parte dei test: \\
\[ CC = \frac{\#righe\_codice\_testato}{\#tot\_righe\_codice}\] \\
\begin{itemize}
\end{itemize}
\subsection{Usabilità}
L'usabilità indica la acapacità dell'utente di comprendere le funzionalita del prodotto e di poterle applicare.
\paragraph{Metriche}
\subparagraph{Facilità di comprensione}
Per poter misurare questo aspetto, viene utilizzato l'indicatore Commenti per Linee di Codice, rilevando  il rapporto tra le rige di commento e il codice effettivo.
\[ CLC = \frac{\#righe\_commento}{\#tot\_righe}\] \\
Un codice sufficentemente commentato ha alte probabilità di essere facilmente compreso dall'utente, per questo si prendano come riferimento:
\begin{itemize}
\item valore desiderabile: $\geq$ 0.20;
\item valore accettabile: $\geq$ 0.10.
\end{itemize}

\subsection{Manutenibilità}
\paragraph{Metriche}
\subparagraph{Linee di codice}
\subparagraph{Numero dei metodi}
\subparagraph{Numero dei parametri}
