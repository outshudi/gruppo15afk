\documentclass[a4paper, oneside, openany, dvipsnames, table, 12pt]{article}
\usepackage{../Template/AFKstyle}
\usepackage{hyperref}
\usepackage{verbatim} %per commenti di più righe \begin{comment} \end{comment}
\usepackage{amsmath}
\newcommand{\Titolo}{Verbale esterno 2020-03-31}

\newcommand{\Gruppo}{TeamAFK}

\newcommand{\Redattori}{Simone Meneghin}

\newcommand{\Verificatori}{z}

\newcommand{\pathimg}{../../Template/img/logoAFK.png}

\newcommand{\Approvatore}{a}

\newcommand{\Distribuzione}{Prof. Vardanega Tullio \newline Prof. Cardin Riccardo \newline TeamAFK}

\newcommand{\Uso}{Esterno}

\newcommand{\NomeProgetto}{"Predire in Grafana"}

\newcommand{\Mail}{gruppoafk15@gmail.com}

\newcommand{\Versionedoc}{1.0.0}

\newcommand{\DescrizioneDoc}{Riassunto dell'incontro del gruppo \textit{TeamAFK} con il proponente tenutosi il 2020-03-31.}


\makeindex

\begin{document}
\copertina{}

%definizione colori per tabelle (tranne copertina)
\definecolor{redafk}{RGB}{255, 71, 87}
\definecolor{grey2}{RGB}{204, 204, 204}
\definecolor{greyRowafk}{RGB}{234, 234, 234}
\rowcolors{2}{grey2}{greyRowafk}
\renewcommand{\arraystretch}{1.5}

\newpage
\section*{Registro delle modifiche}
{
	\centering
	\begin{longtable}{ c c  C{4cm}  c  c }
		\rowcolor{redafk}
		\textcolor{white}{\textbf{Versione}} & \textcolor{white}{\textbf{Data}} & \textcolor{white}{\textbf{Descrizione}} & \textcolor{white}{\textbf{Nominativo}} & \textcolor{white}{\textbf{Ruolo}}\\		
		0.0.1 & 2020-03-20 & Stesura documento & Davide Zilio &\reda{}\\		
		
	\end{longtable}

}

%Didascalia tabelle/immagini (prendono come riferimento la subsection)
\counterwithin{table}{subsection}
\counterwithin{figure}{subsection}
\newpage

%indice, indice figure e indice tabelle
\tableofcontents
\newpage
\listoffigures
\newpage
\listoftables
\newpage

\section{Introduzzione}
\subsection{Premessa}
Il \textit{Piano di Qualifica} è un documento su cui si prevede di lavorare l'intera durata del progetto. Molti dei contenuti del documento sono di natura instabile. Ad esempio, molte delle metriche scelte non sono applicabili nella fase iniziale, e solo con il loro utilizzo pratico si può valutarne l'effettiva utilità. Anche i processi selezionati possono essere soggetti a cambiamenti, rivelandosi insufficienti o inadeguati agli scopi del progetto e al modo di lavorare del team. Il documento è stato scritto in diversi periodi in quanto alcune delle cose non le si potevano conoscere a priori. \\
Per tutte queste ragioni, il documento è prodotto in maniera incrementale, e i suoi contenuti iniziali sono da considerarsi incompleti: subiranno significative aggiunte e modifiche nel tempo.

\subsection{Scopo del documento}
Questo documento ha lo scopo di mostrare le strategie di verifica\glo e validazione\glo adottate al fine di garantire la qualità di prodotto e di processo. Per raggiungere questo obiettivo viene applicato un sistema di verifica continua sui processi in corso e sulle attività svolte. In questo modo è quindi possibile rilevare e correggere all'istante eventuali anomalie, riducendo al minimo lo spreco delle risorse.

\subsection{Scopo del prodotto}
Lo scopo del prodotto è quello di realizzare due plug-in per il software Grafana\glo, che permettano di monitorare e predire lo stato di un sistema in analisi. Grazie alle predizioni sarà possibile attivare degli allarmi così da poter gestire preventivamente eventuali situazioni di rischio. \\
I due plug-in\glo utilizzeranno la Support Vector Machine\glo (SVM) per poter effettuare regressione lineare o categorizzazione sui dati forniti.

\subsection{Glossario}
Per evitare ambiguità nei documenti formali, viene fornito il documento \textbf{Glossario},
contenente tutti i termini considerati di difficile comprensione. Perciò nella documentazione fornita, ogni vocabolo contenuto in Glossario è contrassegnato dalla lettera G a pedice.

\subsection{Riferimenti}
\subsubsection{Riferimenti normativi}
\begin{itemize}
	\item \textit{Norme di Progetto};
	\item ISO/IEC 9126: \\
	https://en.wikipedia.org/wiki/ISO/IEC_9126
	\item ISO/IEC 15504: \\
	https://en.wikipedia.org/wiki/ISO/IEC_15504
\end{itemize}
\subsubsection{Riferimenti informativi}
\begin{itemize}
	\item Capitolato d'appalto C4: \\
	 https://www.math.unipd.it/~tullio/IS-1/2019/Progetto/C4.pdf
\end{itemize}
\pagebreak

\section{Qualità di processo}
Al fine di garantire la qualità del prodotto è necessario perseguire in primis la qualità dei processi che lo definiscono. Si è deciso dunque di aderire allo standard ISO/IEC 15504: quest'ultimo permette di valutare il livello di maturità e capacità (capability) dei processi, al fine di apportare modifiche migliorative. \\
Sono fissati i seguenti obiettivi: \begin{itemize}
\item rispetto di tempi e costi descritti nel \textit{Piano\_di\_Progetto\_v1.0.0};
\item continuo miglioramento dei processi;
\item misurabilità dello stato dei processi.
\end{itemize}
Per misurare la qualità, sono state scelte delle speifiche metriche; ognuna di queste fa uso di scale differenti e fissate a priori. Ogni metriche conterrà:
\begin{itemize}
\item \textbf{Nome};
\item \textbf{Descrizione};
\item \textbf{Parametri}: range di valori su cui confrontare le misure ottenute. Sono definiti i seguenti intervalli: \begin{itemize}
\item accettabile;
\item ottimale.
\end{itemize}
Essi possono essere: \begin{itemize}
\item \textbf{aperti}, se gli estremi non sono compresi. Esempio: (a, b) = $a < x < b$; 
\item \textbf{chiusi}, se gli estremi sono compresi. Esempio: [a, b] = $a \leq x \leq b$;
\item \textbf{limitati}, se gli estremi sono numeri finiti;
\item \textbf{illimitati}, se almeno uno degli estremi è infinito.
\end{itemize}
\end{itemize}
Non saranno trattati in questo documento la descrizione e gli strumenti per il calcolo delle metriche, che sono reperibili nelle \textit{Norme\_di\_Progetto\_v1.0.0}.

\subsubsection{Metriche dei processi}
Le metriche presentate in questa sezione monitorano lo stato dei processi del progetto analizzando l’uso
che essi fanno di tempo e denaro. Sono particolarmente utili per il \textit{Responsabile}, che può quindi decidere di apportare modifiche alla pianificazione quando necessario.

\paragraph{MP01 - Schedule Variance} \mbox{} \\ \mbox{} \\
La Schedule Variance indica se una certa attività o processo è in anticipo, in pari, o in ritardo rispetto alla data di scadenza prevista. \\ \\ 
\textbf{Parametri adottati:} 
\begin{itemize}
\item Range accettabile: ($ -\infty $, 2];
\item Range ottimale: ($ -\infty $, 0].
\end{itemize}

\paragraph{MP02 - Budget Variance} \mbox{} \\ \mbox{} \\
Permette di controllare i costi sostenuti alla data corrente rispetto al budget preventivato. \\ \\ 
\textbf{Parametri adottati:}  
\begin{itemize}
\item Range accettabile: [$-15\%$, $0\%$]; 
\item Range ottimale: $ > 0\%$.
\end{itemize}

\paragraph{MP03 - Produttività} \mbox{} \\ \mbox{} \\
Rappresenta la produttività media delle risorse impiegate, cioè delle persone coinvolte, nelle diverse fasi del progetto. \'E misurata in termini di numero di linee di codice (LOC) sviluppate da una persona
nell’unità di tempo stabilita (settimana).\\ \\ 
\textbf{Parametri adottati:} 
\begin{itemize}
	\item Range accettabile: [50, 100];
	\item Range ottimale: $ > 100$.
\end{itemize}



\pagebreak

\section{Qualifica di prodotto}
\subsection{Scopo}
Per stabilire la Qualità di prodotto, il team di Quality Management\glo adotta come standard di riferimento \textbf{ISO/IEC 9126} per stabilire il modello della qualità del software. Per stabilire il raggiungimento di un determinato obiettivo di qualità, ogni voce trattata è accompagnata da un apposito parametro.
\subsection{Obiettivi}
Gli obiettivi di qualità che il team di Quality Management vuole raggiungere sono:
\begin{itemize}
\item Affidabilità;
\item Usabilità.
\end{itemize}
\subsection{Metriche della documentazione}
In relazione agli obiettivi prestabiliti, il team adotta i diversi strumenti per misurare la qualità del prodotto, riportati di seguito.
\subsubsection{MD01 - Indice di Gulpease}
L'Indice di Gulpease registra la leggibilità di un documento. \\ \\ 
\textbf{Parametri adottati:} 
\begin{itemize}
\item Range accettabile: [60, 80);
\item Range ottimale: [80, 100].
\end{itemize}
\subsection{Metriche del codice sorgente}
\subsubsection{MS01 - Linee di Codice}
\'E la metrica che registra la dimensione di tutto il codice sorgente.
\subsubsection{MS02 - Numero dei Metodi}
Questa metrica conteggia il numero di metodi presenti nella classe di un oggetto.\\ \\ 
\textbf{Parametri adottati:} 
\begin{itemize}
\item Range accettabile: [0, 15];
\item Range ottimale: [0, 8].
\end{itemize}
\subsubsection{MS03 - Numero di Parametri}
Questo strumento tiene conto del numero di parametri formali di un metodo.\\ \\ 
\textbf{Parametri adottati:} 
\begin{itemize}
\item Range accettabile: [0, 6];
\item Range ottimale: [0, 3].
\end{itemize}
\subsubsection{MS04 - Commenti per Linee di Codice}
\'E il rapporto tra le rige di commento e il codice effettivo.\\ \\ 
\textbf{Parametri adottati:} 
\begin{itemize}
\item Range accettabile: [0.05, 0.10);
\item Range ottimale: [0.10, 0.20].
\end{itemize}
\subsubsection{MS05 - Code Coverage}
\'E la metrica con il compito di misurare l'indice di copertura del codice da parte dei test in termini percentuali.\\ \\ 
\textbf{Parametri adottati:} 
\begin{itemize}
\item Range accettabile: [70, 80)\%;
\item Range ottimale: [80, 100]\%.
\end{itemize}
Seppur l'obiettivo del team di sviluppo sia quello di avere una Code Coverage del 100\%, tale traguardo potrebbe non essere raggiunto, in quanto comporterrebbe un aumento dei costi di progetto troppo elevati.
\pagebreak

\section{Specifica dei test}
Per verificare la qualità del prodotto software, il gruppo fornitore ha deciso di adottare il \textbf{Modello di Sviluppo a V}\glo, sviluppando così una serie di test. Questi hanno lo scopo di controllare che tutte le unità di cui è composto il sistema, siano state implementate correttamente, rispettando tutti gli aspetti del progetto.
Per seplificare la loro consultazione, i test saranno suddivisi in categorie, per mezzo di tabelle, mostrando l'output prodotto, e sottolineando se è un risultato atteso o non atteso.
\subsection{Stato dei test}
Per definire lo stato dei test, si usano le seguenti sigle:
\begin{itemize}
\item \textbf{I}: test implementato;
\item \textbf{NI}: test non implementato.
\end{itemize}

\begin{longtable}{C{2.5cm} C{2.5cm} L{8cm} C{2cm}}
\rowcolor{white}\caption{Tabella dei test} \\
		\rowcolor{redafk}
\textcolor{white}{\textbf{Requisito}} &
\textcolor{white}{\textbf{Caso d'uso}} &
\textcolor{white}{\textbf{Descrizione}} &
\textcolor{white}{\textbf{Esito}} \\
		\endfirsthead
		\rowcolor{white}\caption[]{(continua)} \\
		\rowcolor{redafk}
\textcolor{white}{\textbf{Requisito}} &
\textcolor{white}{\textbf{Caso d'uso}} &
\textcolor{white}{\textbf{Descrizione}} &
\textcolor{white}{\textbf{Esito}} \\
		\endhead
%-------------------------------------------- Simo		
TSOF1 & UC1 &
L'utente deve poter creare il file JSON contenente i predittori. \newline
All'utente viene chiesto di:
\begin{itemize}
	\item cliccare il pulsante “Carica Dati di Addestramento”;
	\item scegliere i dati di addestramento da caricare;
	\item selezionare l’algoritmo di previsione;
	\item conferma delle operazioni;
	\item salvataggio file JSON contenente i predittori.
\end{itemize} & NI \\

TSOF1.1 & UC1.1 &
L'utente deve poter caricare i dati di addestramento. \newline All'utente viene chiesto di:
\begin{itemize}
 	\item cliccare il pulsante "Carica Dati di Addestramento";
 	\item verificare che si apra la finestra che visualizza il file system.
\end{itemize} & NI	\\


TSOF1.2 & UC1.2 &
L'utente deve poter scegliere i dati di addestramento. \newline All'utente viene chiesto di:
\begin{itemize}
 	\item verificare che dalla finestra di dialogo siano visibili solo file CSV;
	\item selezionare i dati di addestramento.
\end{itemize} 
& NI \\
 
TSOF1.3 & UC1.3 & 
L'utente deve poter scegliere l'algoritmo di predizione. \newline All'utente viene chiesto di:
\begin{itemize}
	\item cliccare sulla Combo Box con etichetta "Seleziona Algoritmo";
	\item scegliere uno degli algoritmi proposti (RL o SMV).
\end{itemize} & NI \\
 
TSOF1.4 & UC1.4 & 
L'utente deve poter confermare la scelta dell'algoritmo. \newline All'utente viene chiesto di:
\begin{itemize}
	\item cliccare sul pulsante "Conferma".
\end{itemize} & NI \\

TSOF1.4.1 & UC9 & 
L'utente deve poter visualizzare un messaggio d'errore se la scelta dell'algoritmo non è compatibile con i dati di addestramento. \newline All'utente viene chiesto di:
\begin{itemize}
	\item verificare la visualizzazione dell'errore;
	\item verificare di essere rimandati al TSOF1.2.
\end{itemize} & NI \\

TSOF1.5 & UC1.5 & 
L'utente deve poter denominare il file JSON e scegliere dove salvarlo. \newline All'utente viene chiesto di:
\begin{itemize}
	\item scegliere un nome per il file JSON;
	\item scegliere dove salvare il file JSON.
\end{itemize} & NI \\

TSOF1.5.1 & UC16 &
L'utente deve poter vedere la conferma dell'avvenuto salvataggio. \newline All'utente viene chiesto di:
\begin{itemize}
	\item visualizzare il messaggio di notifica "Avvenuto Successo Salvataggio File JSON";
	\item cliccare su "Conferma" per chiudere la notifica.
\end{itemize} & NI	\\

TSOF2 & UC2 &
L'utente deve poter caricare il file JSON nel plug-in. \newline All'utente viene chiesto di:
\begin{itemize}
	\item cliccare il pulsante per caricare il file JSON;
	\item selezionare il file JSON;
	\item confermare il caricamento del file.
\end{itemize} & NI	\\


TSOF2.1 & UC2.1 &
L'utente deve selezionare di caricare file JSON. \newline All'utente viene chiesto di:
\begin{itemize}
	\item cliccare su "Carica JSON";
	\item verificare la visualizzazione della finestra di selezione file.
\end{itemize} & NI	\\

TSOF2.1.1 & UC10 &
L'utente deve poter visualizzare il messaggio di alert del caricamento già avvenuto e caricare nuovamente il file. \newline All'utente viene chiesto di:
\begin{itemize}
	\item visualizzare il messaggio di alert "File JSON già caricato";
	\item cliccare su "Conferma" per sovrascrivere il file.
\end{itemize} & NI	\\

TSOF2.1.2 & UC10 &
L'utente deve poter visualizzare il messaggio di alert del caricamento già avvenuto e annullare il caricamento \newline All'utente viene chiesto di:
\begin{itemize}
	\item visualizzare il messaggio di alert "File JSON già caricato";
	\item cliccare su "Annulla" per tornare alla sezione di caricamento.
\end{itemize} & NI	\\

TSOF2.2 & UC2.2 &
L'utente deve poter selezionare il file JSON. \newline All'utente viene chiesto di:
\begin{itemize}
	\item verificare che siano visibili solo file JSON;
	\item selezionare il file dalla finestra di dialogo.
\end{itemize} & NI	\\

TSOF2.3 & UC2.3 &
L'utente deve poter confermare il caricamento del file. \newline All'utente viene chiesto di:
\begin{itemize}
	\item cliccare sul pulsante "Conferma".
\end{itemize} & NI	\\


TSOF2.3.1 & UC11 &
L'utente deve poter visualizzare un messaggio d'errore in caso di problemi con il caricamento. \newline All'utente viene chiesto di:
\begin{itemize}
	\item visualizzare il messaggio d'errore "Struttura del file JSON non Supportata";
	\item cliccare il pulsante "Conferma";
	\item verificare di essere ritornato alla selezione del file.
\end{itemize} & NI	\\

TSOF2.3.2 & UC17 &
L'utente deve poter visualizzare un messaggio di notifica di caricamento avvenuto con successo. \newline All'utente viene chiesto di:
\begin{itemize}
	\item visualizzare il messaggio di notifica "Avvenuto Successo Caricamento File JSON";
	\item cliccare il pulsante "Continua".
\end{itemize} & NI	\\
%-------------------------------------------- Olly
TSO3 & 
UC3 &
L'utente monitoratore deve poter collegare un predittore ad un flusso. In particolare l'utente deve:
\begin{itemize}
	\item poter visualizzare la schermata di collegamento;
	\item poter selezionale il server di Grafana a cui collegarsi.
\end{itemize} &
NI \\ 

TSO3.1 &
UC3.1 &
L'utente monitoratore deve poter selezionare il Database. All'utente viene chiesto di:
\begin{itemize}
	\item verificare l'effettiva connessione al server;
	\item visualizzare la lista di Database disponibili;
	\item verificare di poter selezionare il Database desiderato.
\end{itemize}&
NI \\

TSO3.2 &
UC3.2 &
L'utente monitoratore deve poter selezionare almeno un flusso di dati. All'utente viene chiesto di:
\begin{itemize}
	\item visuarizzare le tabelle del Database;
	\item verificare che le tabelle contengano i dati di un corrispondente flusso;
	\item verificare di poter selezionare il flusso desiderato;
	\item verificare di poter utilizzare i dati del flusso selezionato.
\end{itemize}&
NI \\

TSO3.3 &
UC3.3 &
L'utente monitoratore deve poter selezionare il predittore da associare al flusso. All'utente viene chiesto di:
\begin{itemize}
	\item visuarizzare l'elencoo dei predittori;
	\item verificare di poter selezionare il/i predittore/i desiderati;
	\item verificare la buona riuscita adel collegamento.
\end{itemize}&
NI \\

TSO3.4 &
UC3.4 &
L'utente monitoratore deve poter selezionare un nodo del flusso. All'utente viene chiesto di:
\begin{itemize}
	\item verificare di poter selezionare il nodo desiderato;
	\item verificare di aver a disposizione il nodo desiderato.
\end{itemize}&
NI \\

TSO3.5 &
UC3.5 &
L'utente monitoratore deve poter stabilire una o più soglie al predittore. All'utente viene chiesto di:
\begin{itemize}
	\item verificare di poter stabilire il nodo desiderato;
	\item verificare la coerenza tra la soglia impostata e quella desiderata.
\end{itemize}&
NI \\

TSO3.5.1 &
UC12 &
L'utene monitoratore deve poter visualizzare il messaggio d'errore sulla soglia stabilita. All'utente viene chiesto di:
\begin{itemize}
	\item poter visualizzare il messaggio "Errore Impostazione Soglia Non Valida";
	\item poter cliccare il pulsante "Conferma";
	\item verificare che dopo il click sul pulsante "Conferma", sia possibile impostare la soglia.
\end{itemize} &
NI \\

TSO3.6 &
UC3.6 &
L'utente monitoratore deve poter confermare il collegamento e vedere la lista dei collegamenti. All'utente viene chiesto di:
\begin{itemize}
	\item poter visualizzare e cliccare il pulsante etichettato "Conferma Collegamento";
	\item verificare l'effettiva conferma del collegamento;
	\item verificare la possibilità di effettuare un altro collegamento.
\end{itemize}&
NI \\

TSO3.6.1 &
UC13 &
L'utente monitoratore deve poter confermare il collegamento e vedere la lista dei collegamenti. All'utente viene chiesto di:
\begin{itemize}
	\item poter visualizzare e cliccare il pulsante etichettato "Conferma Collegamento";
	\item verificare l'effettiva conferma del collegamento;
	\item verificare la possibilità di effettuare un altro collegamento.
\end{itemize}&
NI \\
TSO3.6.2 &
UC18 &
L'utente deve poter visualizzare il messaggio di notifica per la buona riuscita del collegamento. All'utente viene chiesto di:
\begin{itemize}
	\item visualizzate il messaggio "Collegamento Avvenuto con Successo";
	\item poter vvisualizzare e cliccare il pulsante "Conferma".
\end{itemize} &
NI \\

TSO3.6.3 &
UC19 &
L'utente monitoratore deve poter visualizzare l'elenco dei collegamenti. All'utente viene chiesto di:
\begin{itemize}
	\item poter visualizzare, per ogni collegamento, il predittore/i, il nodo del flusso dati e la soglia;
	\item poter visualizzare i pulsanti "Scollega Collegamento" e "Modifica Collegamento".
\end{itemize}&
NI \\

% ------------- FINE TEST 3 ------------------
% ------------- INIZIO TEST 4 ----------------

TSO4 &
UC4 &
L'utente monitoratore deve poter scollegare il predittore. All'utente viene chiesto di:
\begin{itemize}
	\item poter visualizzare e cliccare il pulsante "Scollega Predittore";
	\item verificare l'effettiva e corretta esecuzione dello scollegamento.
\end{itemize}&
NI \\


TSO4.1 &
UC20 &
L'utente monitoratore deve poter visualizzare il messaggio di alert in caso di scollegamento. All'utente viene chiesto di:
\begin{itemize}
	\item poter visualizzare e cliccare il pulsante "Scollega Predittore";
	\item verificare l'effettiva e corretta esecuzione dello scollegamento.
\end{itemize}&
NI \\

% ------------- FINE TEST 4 ------------------

%-------------------------------------------- Davide		
TSOF5 & UC5 & L'utente deve poter modificare un collegamento. \newline All'utente viene chiesto di: \begin{itemize}
\item cliccare il pulsante "Modifica collegamento".
\end{itemize} & NI \\
TSOF6 & UC6 & L'utente deve poter effettuare le operazioni di calcolo delle previsioni. \newline
All'utente viene chiesto di: \begin{itemize}
\item inserire la politica temporale da applicare;
\item avviare il monitoraggio sul flusso di dati.
\end{itemize}& NI \\
TSOF6.1 & UC6.1 & L'utente deve poter inserire la politica temporale.\newline All'utente viene chiesto di inserire: \begin{itemize}
\item il campo "Secondi";
\item il campo "Minuti";
\item il campo "Ore".
\end{itemize} & NI \\
TSOF6.1.1 & UC14 & L'utente deve poter visualizzare il messaggio d’errore nel caso in cui la politica temporale non sia stata definita. & NI \\
TSOF6.2 & UC6.2 & L'utente deve poter avviare il monitoraggio sul flusso di dati. \newline All'utente viene chiesto di: \begin{itemize}
\item selezionare il pulsante "Avvia Monitoraggio".
\end{itemize} & NI \\
TSOF6.2.1 & UC15 & L'utente deve poter visualizzare il messaggio d’errore nel caso in cui nessun predittore sia stato collegato. & NI \\
TSOF6.2.2 & UC21 & L'utente deve poter visualizzare il messaggio di notifica del corretto avvio del monitoraggio. & NI \\
TSOF6.3 & UC6.3 & L'utente deve poter salvare la previsione. \newline All'utente viene chiesto di: \begin{itemize}
\item cliccare il pulsante "Invia previsioni".
\end{itemize} & NI \\
TSOF6.3.1 & UC23 & L'utente deve poter visualizzare il messaggio di notifica del corretto invio, e salvataggio, della previsione. & NI \\
TSOF7 & UC7 & L'utente deve poter interrompere il monitoraggio. \newline All'utente viene chiesto di: \begin{itemize}
\item cliccare il pulsante "Interrompi Monitoraggio".
\end{itemize} & NI \\
TSOF7.1 & UC22 & L'utente deve poter visualizzare il messaggio di notifica dell'interruzione del monitoraggio. \newline All'utente viene chiesto di: \begin{itemize}
\item cliccare il pulsante "Conferma", per confermare l'avvenuta interruzione.
\end{itemize} & NI \\
TSOF8 & UC8 & L'utente visualizza le previsioni nella dashboard. & NI \\
TSFF8.1 & UC24 & L'utente deve poter visualizzare il messaggio di alert di avvenuto raggiungimento della soglia critica. \newline Per poter proseguire, all'utente viene chiesto di: \begin{itemize}
\item cliccare il pulsante "Conferma".
\end{itemize}& NI \\


\end{longtable}
\pagebreak

\appendix %tutte le \section da questo documento in poi saranno numerate con lettere

\section{Standard di qualità}

\subsection{ISO/IEC 9126}
Con la sigla ISO/IEC 9126 si individua una serie di normative e linee guida preposte a descrivere un modello di qualità del software. \\
La norma tecnica relativa alla qualità del software si compone in quattro parti:
\begin{itemize}
	\item modello della qualità del software;
	\item metriche per la qualità esterna;
	\item metriche per la qualità interna;
	\item metriche per la qualità in uso.
\end{itemize}
\subsubsection{Modello della qualità del software}
Il modello di qualità è classificato da sei caratteristiche generali e da varie sottocaratteristiche misurabili attraverso delle metriche.\\
Di seguito vengono definite tali caratteristiche.

\paragraph{Funzionalità} \mbox{} \\ \mbox{} \\
È la capacità del software di soddisfare determinate esigenze, stabilite nell' \textit{Analisi dei Requisiti}, necessarie per operare sotto condizioni specifiche. \\
In dettaglio il software deve soddisfare le seguenti caratteristiche:
\begin{itemize}
	\item \textbf{Appropriatezza}: le funzioni fornite dal software sono appropriate per i compiti ed obiettivi prefissati;
	\item \textbf{Accuratezza}: il software deve fornire risposte concordanti o i precisi effetti richiesti;
	\item \textbf{Interoperabilità}: capacità di interagire ed operare con uno o più sistemi specificati;
	\item \textbf{Conformità}: il prodotto deve essere in grado di aderire a standard, convenzioni e regolamentazioni;
	\item \textbf{Sicurezza}: è la capacità del software di proteggere informazioni e dati, negandone l'accesso a persone o sistemi non autorizzati e invece fornirlo a chi ne è effettivamente autorizzato.
\end{itemize}
\paragraph{Affidabilità}\mbox{} \\ \mbox{} \\
È la capacità del prodotto software di mantenere uno specificato livello di prestazioni quando usato in date condizioni. \\
Nello specifico il software deve soddisfare le seguenti caratteristiche:
\begin{itemize}
	\item \textbf{Maturità}: capacità di un prodotto software di evitare che si verificano errori, malfunzionamenti o siano prodotti risultati non corretti;
	\item \textbf{Tolleranza agli errori}: il software deve mantenere dei predeterminati livelli di prestazioni anche in caso di un uso scorretto o di malfunzionamenti;
	\item \textbf{Recuperabilità}: capacità di ripristinare il livello appropriato di prestazioni a seguito di un malfunzionamento o di un errore. La caratteristica di recuperabilità è valutata dall'arco di tempo nel quale il software può risultare non accessibile;
	\item \textbf{Aderenza}: è la capacità di aderire a standard, regole e convenzioni riguardanti all'affidabilità.
\end{itemize}
\paragraph{Efficienza}\mbox{} \\ \mbox{} \\
Capacità del prodotto software di eseguire le proprie funzioni minimizzando il tempo necessario e sfruttando al meglio le risorse che necessita. \\
Nello specifico il software deve soddisfare le seguenti caratteristiche:
\begin{itemize}
	\item \textbf{Comportamento rispetto al tempo}: è la capacità di fornire adeguati tempi di risposta, elaborazione e velocità di attraversamento, sotto condizioni determinate;
	\item \textbf{Utilizzo delle risorse}: è la capacità di utilizzo di quantità e tipo di risorse in maniera adeguata;
	\item \textbf{Conformità}: è la capacità di aderire a standard e specifiche sull'efficienza\glo.
\end{itemize}
\paragraph{Usabilità}\mbox{} \\ \mbox{} \\
È la capacità del prodotto software di essere capito, appreso, usato e ben accettato dall'utente, anche quando usato sotto condizioni specificate. \\
Nello specifico il software deve soddisfare le seguenti caratteristiche:
\begin{itemize}
	\item \textbf{Comprensibilità}: esprime la facilità di comprensione dei concetti, funzionalità e utilizzo da parte dell'utente;
	\item \textbf{Apprendibilità}: è la capacità di ridurre l'impegno richiesto agli utenti per imparare ad usare l'applicazione;
	\item \textbf{Operabilità}: capacità di mettere in condizione gli utenti di farne uso per i propri scopi e controllarne l'uso;
	\item \textbf{Attrattivà:} è la capacità del software di essere piacevole all'uso per l'utente;
	\item \textbf{Conformità}: è la capacità del software di aderire a standard o convenzioni relativi all'usabilità.

\end{itemize}
\paragraph{Manutenibilità}\mbox{} \\ \mbox{} \\
La manutenibilità è la capacità del software di essere modificato, includendo correzioni, miglioramenti o adattamenti.
Nello specifico il software deve soddisfare le seguenti caratteristiche:
\begin{itemize}
	\item \textbf{Analizzabilità}: rappresenta la facilità con la quale è possibile analizzare il codice per localizzare un errore nello stesso;
	\item \textbf{Modificabilità}: la capacità del prodotto software di permettere l'implementazione di una modifica;
	\item \textbf{Stabilità}: capacità del software di evitare effetti inaspettati a seguito di modifiche errate;
	\item \textbf{Testabilità}: la capacità di essere facilmente testato per validare le modifiche apportate al software.
\end{itemize}
\paragraph{Portabilità}\mbox{} \\ \mbox{} \\
La portabilità è la capacità del software di essere trasportato da un ambiente\glo di lavoro ad un altro. Ambiente che può variare dall'hardware al software.
Nello specifico il software deve soddisfare le seguenti caratteristiche:
\begin{itemize}
	\item \textbf{Adattabilità}: la capacità del software di essere adattato per differenti ambienti senza dover applicare modifiche;
	\item \textbf{Installabilità}: capacità del software di essere installato in un diverso ambiente;
	\item \textbf{Conformità}: la capacità del prodotto software di aderire a standard e convenzioni relative alla portabilità;
	\item \textbf{Sostituibilità}: capacità di essere utilizzato al posto di un altro software per svolgere gli stessi compiti nello stesso ambiente.
\end{itemize}
\subsubsection{Metriche per la qualità esterna}
Servono a misurare i comportamenti del software sulla base dei test, in funzione degli obiettivi stabiliti. Viene rilevata tramite analisi dinamica\glo.

\subsubsection{Metriche per la qualità interna}
Metriche che si applicano al software non eseguibile durante le fasi di progettazione e codifica. Permettono di individuare eventuali problemi che potrebbero influire sulla qualità finale del prodotto prima che venga realizzato un eseguibile. Grazie alle misure effettuate tramite le metriche interne è possibile prevedere il livello di qualità esterna e di qualità in uso del prodotto finale, poiché entrambe vengono influenzate dalla qualità interna.\\
Viene rilevata tramite analisi statica\glo.

\subsubsection{Metriche per la qualità in uso}
La qualità in uso rappresenta il punto di vista dell'utente sul software. Il livello di qualità viene raggiunto dopo che sono stati raggiunti i livelli nella qualità esterna e interna. La qualità in uso, quindi, permette di abilitare specificati utenti ad ottenere specificati obiettivi con efficacia\glo, produttività, sicurezza e soddisfazione.


\subsection{ISO/IEC 15504}
Il modello ISO/IEC 15504, meglio conosciuto come SPICE (\textit{Software Process Improvement and Capability Determination}), è lo standard di riferimento per valutare in modo oggettivo la qualità dei processi nello sviluppo del software. \\
SPICE mette a disposizione una metrica per valutare diversi attributi per ogni processo ed
assegna un valore quantificabile ad ognuno di questi in modo tale da rendere esplicito come poter migliorare tale processo.

\subsubsection{Classificazione dei processi}
SPICE definisce una serie di sei livelli utilizzati per classificare la capacità\glo e la maturità\glo del processo software. Ogni livello è caratterizzato dal soddisfacimento degli attributi associati:
\begin{enumerate}
	\item \textbf{Incompleto}: il processo non è stato implementato o non ha raggiunto il successo desiderato;
	\item \textbf{Eseguito}: il processo è implementato e ha realizzato il suo obiettivo (conformità);\\
	Attributi:
	\begin{itemize}
		\item Process Performance.
	\end{itemize}
	\item \textbf{Gestito}: il processo è gestito e il prodotto finale è verificato, controllato e manutenuto (affidabilità);\\
	Attributi:
	\begin{itemize}
		\item  Performance Management;
		\item  Work Product Management.
	\end{itemize}
	\item \textbf{Stabilito}: il processo è basato sullo standard di processo (standardizzazione);\\
	Attributi:
	\begin{itemize}
		\item  Process Definition;
		\item  Process Deployment.
	\end{itemize}
	\item \textbf{Predicibile}: il processo è consistente e rispetta limiti definiti (strategico);\\
	Attributi:
	\begin{itemize}
		\item  Process Measurement;
		\item  Process Control.
	\end{itemize}
	\item \textbf{Ottimizzato}: il processo segue un miglioramento continuo per rispettare tutti gli obiettivi di
progetto;\\
Attributi:
	\begin{itemize}
		\item  Process Innovation;
		\item  Process Optimization.
	\end{itemize}
\end{enumerate}

Ogni attributo riceve una valutazione nella seguente scala, andando a definire il rispettivo livello di
capacità del processo:
\begin{itemize}
	\item N: non raggiunto (0 - 15\%);
	\item P: parzialmente raggiunto (>15\% - 50\%);
	\item L: largamente raggiunto (>50\%- 85\%);
	\item F: pienamente raggiunto (>85\% - 100\%).
\end{itemize}

\subsubsection{Linee guida}
Lo standard definisce delle linee guida per effettuare delle stime, tali linee guida sono:
\begin{itemize}
	\item processi di misurazione, descritti nel \textit{Piano di Progetto} ;
	\item modello di misurazione, descritto in questo documento;
	\item strumenti di misurazione, descritti nelle \textit{Norme di Progetto}.
\end{itemize}

\subsubsection{Competenze}
Per effettuare le misurazioni necessarie per determinare la qualità raggiunta, sono necessarie delle competenze. La mancanza di esperienza degli elementi del \textit{TeamAFK}, fa sì che nessun membro possieda queste abilità, rendendo così impossibile la piena adesione allo standard. Tuttavia, ogni componente è chiamato a studiare SPICE e ad applicare al meglio le indicazioni descritte in questo documento e nelle \textit{Norme di Progetto}, al fine di perseguire un livello di qualità accettabile.







\pagebreak

\section{Resoconto attività di verifica}
In questa sezione sono descritte le attività di verifica svolte sui documenti che vengono presentati alle revisioni di avanzamento. Qualora una verifica riscontrasse un problema su un documento, nella sezione \S C si discuterà di quali sono i possibili miglioramenti.

\subsection{Analisi dei documenti}
\paragraph*{Analisi statica} \mbox{} \\ \mbox{} \\
L'analisi dei documenti mediante Walkthrough (vedi \textit{Norme di Progetto}) ha portato all'individuazione di alcuni errori frequenti a partire dai quali è stata stilata una check list. In questo modo sarà possibile applicare l’Inspection (vedi \textit{Norme di Progetto}) per le future attività di verifica.

\subsection{Esiti Indice di Gulpease}
\begin{longtable}{c c c}
\rowcolor{white}\caption{Esiti verifica documenti con Indice di Gulpease} \\
		\rowcolor{redafk}
\textcolor{white}{\textbf{Documento}} &
\textcolor{white}{\textbf{Indice Gulpease}} &
\textcolor{white}{\textbf{Esito}} \\
		\endfirsthead
		\rowcolor{white}\caption[]{(continua)} \\
		\rowcolor{redafk}
\textcolor{white}{\textbf{Documento}} &
\textcolor{white}{\textbf{Indice Gulpease}} &
\textcolor{white}{\textbf{Esito}} \\
		\endhead
		\textit{analisi\_dei\_requisiti\_v1.0.0} & 70 & Superato \\
		\textit{glossario\_v1.0.0} & 74 & Superato \\
		\textit{norme\_di\_progetto\_v1.0.0} & 67 & Superato \\
		\textit{piano\_di\_progetto\_v1.0.0} & 69 & Superato \\
		\textit{piano\_di\_qualifica\_v1.0.0} & 72 & Superato \\
		\textit{studio\_di\_fattibilità\_v1.0.0} & 70 & Superato \\
		\textit{VI\_2020-03-11} & 72 & Superato \\
		\textit{VI\_2020-03-18} & 76 & Superato \\
		\textit{VI\_2020-03-20} & 70 & Superato \\
		\textit{VI\_2020-03-22} & 71 & Superato \\
		\textit{VI\_2020-03-31} & 71 & Superato \\
		\textit{VE\_2020-03-16} & 70 & Superato \\
		\textit{VE\_2020-03-31} & 72 & Superato \\
		\textit{VE\_2020-04-03} & 67 & Superato \\
\end{longtable}

\pagebreak

\section{Valutazioni per il miglioramento}
In questa sezione viene riportata la valutazione fatta dal gruppo riguardo il lavoro svolto finora.
Lo scopo di questa scelta è trattare i problemi sorti e procedere alla loro più efficiente risoluzione
in modo tale che non si verifichino in futuro. \\
Verrano dunque tracciati problemi riguardanti i seguenti ambiti: \begin{itemize}
\item \textbf{Organizzazione}: vengono analizzati i problemi riguardanti l'organizzazione e la comunicazione all'interno del gruppo;
\item \textbf{Ruoli}: vengono analizzati i problemi riguardanti il corretto svolgimento di un ruolo;
\item \textbf{Strumenti di lavoro}: vengono analizzati i problemi riguardanti l'uso degli strumenti scelti.
\end{itemize}
Poichè non vi è una persona esterna che possa dare una valutazione oggettiva, ogni problema viene sollevato sulla base dell'autovalutazione dei soli membri del gruppo. Nonostante sia un sistema poco efficace, il gruppo ha beneficiato di questa scelta dal punto di vista comunicativo e produttivo, migliorando progressivamente la qualità del lavoro.\\
Questa sezione verrà aggiornata con l'avanzamento del prodotto riportando nuove problematiche, qualora queste dovessero verificarsi.

\subsection{Valutazioni sull'organizzazione}

\begin{table}[H]
\caption{Problematiche relative all'organizzazione}
\begin{center}
\begin{tabular}{ C{2.5cm} L{5cm} c L{5.5cm} }
\rowcolor{redafk}
\textcolor{white}{\textbf{Problema}} & \centerline{\textcolor{white}{\textbf{Descrizione}}} & \textcolor{white}{\textbf{Gravità}} & \centerline{\textcolor{white}{\textbf{Soluzione}}}\\
Incontro tra stakeholders\glo & A causa del Covid19, gli stakeholders hanno dovuto adattarsi alle restrizioni imposte, e tuttora in corso, impiegando tecnologie di comunicazione adatte allo smart working. & Bassa & Gli stakeholders hanno quindi utilizzato le tecnologie di comunicazione riportate nelle \textit{Norme di Progetto} per proseguire il progetto senza ulteriori intoppi. \\
\end{tabular}
\end{center}
\end{table}
\pagebreak

\subsection{Valutazioni sui ruoli}

\begin{table}[H]
\caption{Problematiche relative ai ruoli}
\begin{center}
\begin{tabular}{ C{2.5cm} L{5cm} c L{5.5cm} }
\rowcolor{redafk}
\textcolor{white}{\textbf{Problema}} & \centerline{\textcolor{white}{\textbf{Descrizione}}} & \textcolor{white}{\textbf{Gravità}} & \centerline{\textcolor{white}{\textbf{Soluzione}}}\\
Ruolo di \textit{Responsabile} & A causa dell'inesperienza, chi ha lavorato come \textit{Responsabile} ha avuto difficoltà nella suddivisione bilanciata delle ore tra i membri provocando diverse ridistribuzioni delle ore. & Alta & Per evitare eventuali ritardi nelle consegne, il gruppo ha deciso di dedicare del tempo per analizzare meglio la mole di lavoro e compiere così una più accurata distribuzione delle ore. \\
\end{tabular}
\end{center}
\end{table}
\pagebreak
\subsection{Valutazioni sugli strumenti di lavoro}

\begin{table}[H]
\caption{Problematiche relative agli strumenti di lavoro}
\begin{tabular}{ C{2.5cm} L{5cm} c L{5.5cm} }
\rowcolor{redafk}
\textcolor{white}{\textbf{Problema}} & \centerline{\textcolor{white}{\textbf{Descrizione}}} & \textcolor{white}{\textbf{Gravità}} & \centerline{\textcolor{white}{\textbf{Soluzione}}}\\
GitHub & Si sono riscontrati in più occasioni
conflitti sui file in cui si stava lavorando e il tempo utilizzato per risolverli è stato sottratto dal tempo di lavoro. & Media & Il gruppo è stato istruito sull’uso di specifici branch\glo in modo tale che la modifiche di tutti i componenti si potessero integrare con il proprio lavoro senza che quest’ultimo potesse avere dei conflitti. \\
\LaTeX{} & A causa dell’inesperienza di
alcuni membri del gruppo nell’utilizzo
di questo strumento, si sono riscontrate diverse
difficoltà sopratutto nella costruzione di tabelle e nell'inserimento di formule matematiche. & Bassa & Per risolvere in breve tempo questa problematica, si è deciso di affiancare ai membri meno esperti chi sapeva già utilizzare i comandi di \LaTeX{} dando così la possibilità ai primi di imparare e permettendo ai secondi di non subire grossi rallentamenti nel lavoro.
\end{tabular}
\end{table}
\pagebreak


\end{document}