\section{Casi d'uso}

	\subsection{Introduzione}
Il numero limitato dei casi identificati è dovuto alla natura del prodotto, in quanto si tratta di un plug-in e quindi di un’estensione di una piattaforma già esistente.
Per la piattaforma in questione non è fornita alcuna documentazione in quanto essa è già disponibile al sito web del distributore della piattaforma: Grafana Labs (inserire link; vedi altri proggeti).

	\subsection{Struttura dei casi d'uso}
Viene riportata di seguito la classificazione dei casi d’uso. Verranno descritti nel documento rispettando la struttura esposta in 3.2.1..
	
		\subsubsection{Struttura dei casi d'uso}
		\begin{itemize}
			\item Titolo;
 			\item Descrizione; 
 			\item Attori; 
			\item Precondizioni; 
 			\item Postcondizioni; 
 			\item Scenario principale; 
 			\item Estensioni.
 		\end{itemize}
Verrà utilizzato il linguaggio di modellazione UML 2.0 (fornito dal prof. Cardin a link xy).
Ogni requisito può essere inserito in vari UC, che si differenziano ciascuno per una diversa profondità dei dettagli da cui possono distinguersi ulteriori requisiti. 

		\subsubsection{Classificazione dei casi d'uso}
Identificare i casi d’uso in modo univoco aiuta la tracciabilità.  I casi d’uso sono classificati nel seguente modo:  UC[ID].
\emph{ID} è il codice univoco che rappresenta il requisito del caso d’uso. Se lo UC è collegato per profondità di dettaglio ad un altro UC, l’ID verrà rappresentato in maniera gerarchica(esempio: Se un UC rappresenta un livello di dettaglio maggiore con ID x, il suo ID sarà x.y, dove y indica l’annidamento raggiunto).
	
	\subsection{Attori}
Il sistema di autenticazione e registrazione dell’utente viene gestito per intero dal sistema Grafana, in quanto il prodotto finale non disporrà di una funzionalità di autenticazione/registrazione interna.
Il numero limitato di differenti  attori che approcciano al prodotto in analisi è dovuto principalmente  al fatto che , essendo il prodotto “Predire con Grafana” un insieme di due plug-in del sistema indipendente di “Grafana”, un esiguo numero di utenti hanno effettivamente la possibilità di approcciare al prodotto finale.
\textbf{Attori primari}
	\begin{itemize}
		\item\textbf{Utente}: è un generico utente con autenticazione già effettuata nel sistema Grafana. E’ l’unico tipo di utente in grado di interagire con i prodotti, in quanto questi si tratta di plug-in.

 	\end{itemize}
\textbf{Attori secondari}
	\begin{itemize}
		\item\textbf{Piattaforma Grafana}: è un sistema di monitoraggio di stream\glo di dati, che ospiterà il plug-in prodotto. Consente agli utenti registrati di lanciare alert e realizzare grafici modellati sui dati forniti in ingresso al plug-in.
 	\end{itemize}