\section{Casi d'uso}

	\subsection{Introduzione}
Il numero limitato dei casi identificati è dovuto alla natura del prodotto, in quanto si tratta di un plug-in e quindi di un’estensione di una piattaforma già esistente.
Per la piattaforma in questione non è fornita alcuna documentazione in quanto essa è già disponibile al sito web del distributore della piattaforma: Grafana Labs (inserire link; vedi altri proggeti).

	\subsection{Struttura dei casi d'uso}
Viene riportata di seguito la classificazione dei casi d’uso. Verranno descritti nel documento rispettando la struttura esposta in 3.2.1..
	
		\subsubsection{Struttura dei casi d'uso}
		\begin{itemize}
			\item Titolo;
 			\item Descrizione; 
 			\item Attori; 
			\item Precondizioni; 
 			\item Postcondizioni; 
 			\item Scenario principale; 
 			\item Estensioni.
 		\end{itemize}
Verrà utilizzato il linguaggio di modellazione UML 2.0 (fornito dal prof. Cardin a link xy).
Ogni requisito può essere inserito in vari UC, che si differenziano ciascuno per una diversa profondità dei dettagli da cui possono distinguersi ulteriori requisiti. 

		\subsubsection{Classificazione dei casi d'uso}
Identificare i casi d’uso in modo univoco aiuta la tracciabilità.  I casi d’uso sono classificati nel seguente modo:  UC[ID].
\emph{ID} è il codice univoco che rappresenta il requisito del caso d’uso. Se lo UC è collegato per profondità di dettaglio ad un altro UC, l’ID verrà rappresentato in maniera gerarchica(esempio: Se un UC rappresenta un livello di dettaglio maggiore con ID x, il suo ID sarà x.y, dove y indica l’annidamento raggiunto).
	
	\subsection{Attori}
Il sistema di autenticazione e registrazione dell’utente viene gestito per intero dal sistema Grafana, in quanto il prodotto finale non disporrà di una funzionalità di autenticazione/registrazione interna.
Il numero limitato di differenti  attori che approcciano al prodotto in analisi è dovuto principalmente  al fatto che , essendo il prodotto “Predire con Grafana” un insieme di due plug-in del sistema indipendente di “Grafana”, un esiguo numero di utenti hanno effettivamente la possibilità di approcciare al prodotto finale.
\textbf{Attori primari}
	\begin{itemize}
		\item\textbf{Utente}: è un generico utente con autenticazione già effettuata nel sistema Grafana. E’ l’unico tipo di utente in grado di interagire con i prodotti, in quanto questi si tratta di plug-in.

 	\end{itemize}
\textbf{Attori secondari}
	\begin{itemize}
		\item\textbf{Piattaforma Grafana}: è un sistema di monitoraggio di stream\glo di dati, che ospiterà il plug-in prodotto. Consente agli utenti registrati di lanciare alert e realizzare grafici modellati sui dati forniti in ingresso al plug-in.
 	\end{itemize}

%%%%%%%%%%%%%%%%%
	\subsection{UC1 - Creazione file JSON dai dati di addestramento}
		\begin{itemize}
			\item\textbf{Attore Primario}: Utente;
			\item\textbf{Precondizioni}: 
				\begin{enumerate}
					\item L’utente deve possedere dei dati di addestramento in un file formato CSV;
					\item L’utente si trova sul tool di addestramento.
				\end{enumerate}
			\item\textbf{Postcondizioni}:
				\begin{enumerate}
					\item L’utente ha prodotto il file JSON contenente i predittori;
					\item L’utente ha salvato il file JSON in locale.
				\end{enumerate}
			\item\textbf{Scenario Principale}:
				\begin{enumerate}
					\item (\hyperref[par:UC1.1]{UC1.1}) L’utente seleziona il pulsante “Carica Dati di Addestramento”;
					\item (\label{par:UC1.2}) L’utente seleziona i dati di addestramento da caricare;
					\item (\ref{UC1.3}) L’utente selezione l’algoritmo di previsione scelto attraverso la Combo Box “Scegli algoritmo”; 
					\item (\ref{UC1.4}) Conferma delle operazioni; 
					\item (\ref{UC1.5}) L’utente riceve in output il file JSON contenente i predittori per SVM/RL e decide dove salvarlo (localmente).  

				\end{enumerate}
			\item\textbf{Estensione}:
				\begin{enumerate}
					\item UC estende \ref{UC1.4}: Viene visualizzato un messaggio d’errore nel caso in cui l’operazione di caricamento del file non sia andata a buon fine;
					\item  UC estende \ref{UC1.5}: Viene visualizzato un messaggio d’errore se il file di addestramento è incompatibile con l’algoritmo scelto. 
				\end{enumerate}
			\item\textbf{Inclusione}: UC include \ref{UC1.5}:  Viene visualizzato a schermo un messaggio di avvenuto successo della procedura.
		\end{itemize}
		
		\label{par:UC1.1}
		\subsection{UC1.1 - Selezione pulsante di caricamento dati di addestramento}
		\begin{itemize}
			\item\textbf{Attore Primario}: Utente;
			\item\textbf{Precondizioni}: L’utente si trova sul tool di addestramento. L’utente deve possedere il file CSV.
			\item\textbf{Postcondizioni}:
				\begin{enumerate}
					\item L’utente ha cliccato il pulsante di caricamento dati;
					\item Viene aperto la finestra che visualizza il file system, saranno visibili solo i file CSV.
				\end{enumerate}
			\item\textbf{Scenario Principale}: L’utente clicca il pulsante con etichetta “Carica Dati Addestramento”.
		\end{itemize}

		\subsection{UC1.2 - Selezione dati di addestramento }
		\begin{itemize}
			\item\textbf{Attore Primario}: Utente;
			\item\textbf{Precondizioni}: L’utente deve aver cliccato il pulsante di caricamento dati di addestramento (\ref{UC1.1}).
			\item\textbf{Postcondizioni}: L’utente ha selezionato il file dei dati di addestramento.
			\item\textbf{Scenario Principale}: L’utente seleziona il file CSV contenente i dati di addestramento dal file system; è visibile solo il formato CSV.
		\end{itemize}
		
		\subsection{UC1.3 Selezione dell’algoritmo di previsione}
		\begin{itemize}
			\item\textbf{Attore Primario}: Utente;
			\item\textbf{Precondizioni}:L’utente deve aver selezionato i dati di addestramento (\ref{UC1.2}); 
			\item\textbf{Postcondizioni}:L’utente ha scelto l’algoritmo di previsione selezionando il pulsante corrispondente;
			\item\textbf{Scenario Principale}: L’utente clicca la Combo Box con etichetta "Seleziona Algoritmo" e sceglie l’algoritmo (SVM o RL);
		\end{itemize}

	\subsection{UC1.4 Conferma procedura addestramento}
		\begin{itemize}
			\item\textbf{Attore Primario}: Utente;
			\item\textbf{Precondizioni}:
				\begin{enumerate}
					\item L’utente deve aver caricato i dati di addestramento(\ref{UC1.2});
					\item  L’utente deve aver selezionato l’algoritmo di previsione. (\ref{UC1.3}); 
				\end{enumerate}
			\item\textbf{Postcondizioni}:
				\begin{enumerate}
					\item L’utente ha confermato la scelta dell’algoritmo e l’inserimento dei dati di addestramento;
					\item L’utente ha salvato il file JSON.
				\end{enumerate}
			\item\textbf{Scenario Principale}: L’utente clicca il pulsante con etichetta “Conferma”;
		\end{itemize}

%%%%%%%%%%%%%%%


	\subsection{UC2 - Caricamento del file JSON nel plugin}
		\begin{itemize}
			\item\textbf{Attore Primario}: Utente;
			\item\textbf{Precondizioni}: 
				\begin{enumerate}
					\item L’utente ha effettuato l’accesso a grafana;
					\item L’utente si trova sul pannello del plugin “Predire in Grafana”;
					\item L’utente dispone del file JSON contenente i predittori (UC 1). 
				\end{enumerate}
			\item\textbf{Postcondizioni}:
				\begin{enumerate}
					\item L’utente ha caricato il file JSON con predittori associati nel plug-in;
					\item Viene letta la definizione del predittore dal file in formato JSON. 
				\end{enumerate}
			\item\textbf{Scenario Principale}:
				\begin{enumerate}
					\item (\ref{UC2.1}) L’utente seleziona sul pannello l’opzione di caricamento del file JSON; 
					\item (\ref{UC2.2}) L’utente seleziona il file JSON contenente i predittori da locale;
					\item (\ref{UC2.3}) Conferma caricamento.
				\end{enumerate}
			\item\textbf{Estensione}:
					\item UC estende \label{UC2.3}: Viene visualizzato un messaggio d’errore nel caso in cui l’operazione di caricamento del file non sia andata a buon fine;
			\item\textbf{Inclusione}: UC include \label{UC2.3}:  Viene visualizzato a schermo un messaggio di avvenuto successo della procedura.
		\end{itemize}
		
		\subsection{UC 2.1 -  Selezione opzione di Caricamento file JSON}
		\begin{itemize}
			\item\textbf{Attore Primario}: Utente;
			\item\textbf{Precondizioni}: L’utente visualizza il pannello “Predire in Grafana” nella dashboard;
			\item\textbf{Postcondizioni}: L’utente ha cliccato il pulsante di caricamento del file JSON e visualizza il pannello per la selezione del file;
			\item\textbf{Scenario Principale}: L’utente clicca il pulsante con etichetta  “Carica JSON”.
		\end{itemize}		

		\subsection{UC 2.2 - Selezione del file JSON}
		\begin{itemize}
			\item\textbf{Attore Primario}: Utente;
			\item\textbf{Precondizioni}: L’utente ha cliccato il pulsante "Carica JSON";
			\item\textbf{Postcondizioni}: L’utente ha selezionato il file JSON;
			\item\textbf{Scenario Principale}: L’utente seleziona dalla finestra di selezione il file JSON da importare.
		\end{itemize}

		\subsection{UC 2.3 - Conferma di caricamento}
		\begin{itemize}
			\item\textbf{Attore Primario}: Utente;
			\item\textbf{Precondizioni}: L’utente ha selezionato il file JSON da caricare;
			\item\textbf{Postcondizioni}: L’utente ha caricato il file; 
			\item\textbf{Scenario Principale}: L’utente clicca il pulsante etichettato con “Conferma” e il file viene caricato.
		\end{itemize}
		

	
	

