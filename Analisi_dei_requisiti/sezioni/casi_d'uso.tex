\section{Casi d'uso}

	\subsection{Introduzione}
Il numero limitato dei casi identificati è dovuto alla natura del prodotto, in quanto si tratta di un plug-in e quindi di un’estensione di una piattaforma già esistente.
Per la piattaforma in questione non è fornita alcuna documentazione in quanto essa è già disponibile al sito web del distributore della piattaforma: Grafana Labs (inserire link; vedi altri proggeti).

	\subsection{Struttura dei casi d'uso}
Viene riportata di seguito la classificazione dei casi d’uso. Verranno descritti nel documento rispettando la struttura esposta in 3.2.1..
	
		\subsubsection{Struttura dei casi d'uso}
		\begin{itemize}
			\item Titolo;
 			\item Descrizione; 
 			\item Attori; 
			\item Precondizioni; 
 			\item Postcondizioni; 
 			\item Scenario principale; 
 			\item Estensioni.
 		\end{itemize}
Verrà utilizzato il linguaggio di modellazione UML 2.0 (fornito dal prof. Cardin a link xy).
Ogni requisito può essere inserito in vari UC, che si differenziano ciascuno per una diversa profondità dei dettagli da cui possono distinguersi ulteriori requisiti. 

		\subsubsection{Classificazione dei casi d'uso}
Identificare i casi d’uso in modo univoco aiuta la tracciabilità.  I casi d’uso sono classificati nel seguente modo:  UC[ID].
\emph{ID} è il codice univoco che rappresenta il requisito del caso d’uso. Se lo UC è collegato per profondità di dettaglio ad un altro UC, l’ID verrà rappresentato in maniera gerarchica(esempio: Se un UC rappresenta un livello di dettaglio maggiore con ID x, il suo ID sarà x.y, dove y indica l’annidamento raggiunto).
	
	\subsection{Attori}
Il sistema di autenticazione e registrazione dell’utente viene gestito per intero dal sistema Grafana, in quanto il prodotto finale non disporrà di una funzionalità di autenticazione/registrazione interna.
Il numero limitato di differenti  attori che approcciano al prodotto in analisi è dovuto principalmente  al fatto che , essendo il prodotto “Predire con Grafana” un insieme di due plug-in del sistema indipendente di “Grafana”, un esiguo numero di utenti hanno effettivamente la possibilità di approcciare al prodotto finale.
\textbf{Attori primari}
	\begin{itemize}
		\item\textbf{Utente}: è un generico utente con autenticazione già effettuata nel sistema Grafana. E’ l’unico tipo di utente in grado di interagire con i prodotti, in quanto questi si tratta di plug-in.

 	\end{itemize}
\textbf{Attori secondari}
	\begin{itemize}
		\item\textbf{Piattaforma Grafana}: è un sistema di monitoraggio di stream\glo di dati, che ospiterà il plug-in prodotto. Consente agli utenti registrati di lanciare alert e realizzare grafici modellati sui dati forniti in ingresso al plug-in.
 	\end{itemize}

%%%%%%%%%%%%%%%%%
	\subsection{UC1 - Creazione file JSON dai Dati di Addestramento}
		\begin{itemize}
			\item\textbf{Attore Primario}: utente;
			\item\textbf{Precondizioni}: 
				\begin{enumerate}
					\item L’utente deve possedere dei dati di addestramento in un file formato CSV;
					\item L’utente si trova sul tool di addestramento.
				\end{enumerate}
			\item\textbf{Postcondizioni}:
				\begin{enumerate}
					\item L’utente ha prodotto il file JSON contenente i predittori;
					\item L’utente ha salvato il file JSON in locale.
				\end{enumerate}
			\item\textbf{Scenario Principale}:
				\begin{enumerate}
					\item (\hyperref[par:UC1.1]{UC1.1}) L’utente seleziona il pulsante “Carica Dati di Addestramento”;
					\item (\hyperref[par:UC1.2]{UC1.2}) L’utente seleziona i dati di addestramento da caricare;
					\item (\hyperref[par:UC1.3]{UC1.3}) L’utente selezione l’algoritmo di previsione scelto attraverso la Combo Box “Scegli Algoritmo”; 
					\item (\hyperref[par:UC1.4]{UC1.4}) Conferma delle operazioni; 
					\item (\hyperref[par:UC1.5]{UC1.5}) L’utente riceve in output il file JSON contenente i predittori per SVM/RL e decide dove salvarlo (localmente).  
				\end{enumerate}
			\item\textbf{Estensione}: \hyperref[par:UC2]{UC2} estende \hyperref[par:UC1.4]{UC1.4}: viene visualizzato un messaggio d’errore se il file di addestramento è incompatibile con l’algoritmo scelto. 
			\item\textbf{Inclusione}: \hyperref[par:UC4]{UC4} include \hyperref[par:UC1.5]{UC1.5}: viene visualizzato a schermo un messaggio di notifica di avvenuto successo della procedura.
		\end{itemize}
		
		\label{par:UC1.1}
		\subsubsection{UC1.1 - Selezione Pulsante di Caricamento Dati di Addestramento}
		\begin{itemize}
			\item\textbf{Attore Primario}: utente;
			\item\textbf{Precondizioni}: 
				\begin{enumerate}
					\item L’utente si trova sul tool di addestramento.
					\item L’utente deve possedere il file CSV.
				\end{enumerate}
			\item\textbf{Postcondizioni}:
				\begin{enumerate}
					\item L’utente ha cliccato il pulsante di caricamento dati;
					\item Viene aperto la finestra che visualizza il file system, saranno visibili solo i file CSV.
				\end{enumerate}
			\item\textbf{Scenario Principale}: l’utente clicca il pulsante con etichetta “Carica Dati Addestramento”.
		\end{itemize}
		
		\label{par:UC1.2}
		\subsubsection{UC1.2 - Selezione Dati di Addestramento }
		\begin{itemize}
			\item\textbf{Attore Primario}: utente;
			\item\textbf{Precondizioni}: l’utente deve aver cliccato il pulsante di caricamento dati di addestramento (\hyperref[par:UC1.1]{UC1.1});
			\item\textbf{Postcondizioni}: l’utente ha selezionato il file dei dati di addestramento.
			\item\textbf{Scenario Principale}: l’utente seleziona il file CSV contenente i dati di addestramento dal file system; è visibile solo il formato CSV.
		\end{itemize}
		
		\label{par:UC1.3}
		\subsubsection{UC1.3 - Selezione dell’Algoritmo di Previsione}
		\begin{itemize}
			\item\textbf{Attore Primario}: utente;
			\item\textbf{Precondizioni}: l’utente deve aver selezionato i dati di addestramento (\hyperref[par:UC1.2]{UC1.2}); 
			\item\textbf{Postcondizioni}: l’utente ha scelto l’algoritmo di previsione selezionando il pulsante corrispondente;
			\item\textbf{Scenario Principale}: l’utente clicca la Combo Box con etichetta "Seleziona Algoritmo" e sceglie l’algoritmo (SVM o RL);
		\end{itemize}
	
	\label{par:UC1.4}
	\subsubsection{UC1.4 - Conferma Procedura Addestramento}
		\begin{itemize}
			\item\textbf{Attore Primario}: utente;
			\item\textbf{Precondizioni}:
				\begin{enumerate}
					\item L’utente deve aver caricato i dati di addestramento(\hyperref[par:UC1.2]{UC1.2});
					\item  L’utente deve aver selezionato l’algoritmo di previsione (\hyperref[par:UC1.3]{UC1.3}); 
				\end{enumerate}
			\item\textbf{Postcondizioni}:
				\begin{enumerate}
					\item L’utente ha confermato la scelta dell’algoritmo e l’inserimento dei dati di addestramento;
					\item L’utente ha salvato il file JSON.
				\end{enumerate}
			\item\textbf{Scenario Principale}: l’utente clicca il pulsante con etichetta “Conferma”;
			\item\textbf{Estensione}:  \hyperref[par:UC2]{UC2}: viene visualizzato un messaggio d’errore se il file di addestramento è incompatibile con l’algoritmo scelto. 	
		\end{itemize}

	\label{par:UC1.5}
	\subsubsection{UC1.5 - Salvataggio File JSON}
		\begin{itemize}
			\item\textbf{Attore Primario}: utente;
			\item\textbf{Precondizioni}: l’utente ha cliccato il pulsante di conferma (\hyperref[par:UC1.4]{UC1.4});
			\item\textbf{Postcondizioni}: l’utente ha salvato il file JSON in locale;
			\item\textbf{Scenario Principale}: l’utente decide come denominare il file JSON e in che cartella del file system salvarlo;  
			\item\textbf{Inclusione}: \hyperref[par:UC4]{UC4} : viene visualizzato a schermo un messaggio di notifica di avvenuto successo della procedura di salvataggio del file JSON.			
		\end{itemize}

%%%%%%%%%%%%%%%

	\label{par:UC2}
	\subsection{UC2 - Visualizzazione Messaggio d'Errore file CSV Incompatbile}
		\begin{itemize}
			\item\textbf{Attore Primario}: utente;
			\item\textbf{Precondizioni}: l’utente ha selezionato il file CSV che intende utilizzare per l'addestramento e ha cliccato il pulsante di conferma . Il file selezionato è strutturalmente errato e non c'è compatibilità con l'algoritmo selezionato;
			\item\textbf{Postcondizioni}: l'utente visualizza l'errore, viene quindi riportato alla finestra di selezione del file CSV (\hyperref[par:UC1.2]{UC1.2});
			\item\textbf{Scenario Principale}: l’utente visualizza il messaggio d'errore "File Incompatibile" in cui viene segnalato il fatto che il file CSV da lui selezionato (\hyperref[par:UC1.2]{UC1.2}) non è adatto per l'addetramento;  		
		\end{itemize}


%%%%%%%%%%%%%%%

	\label{par:UC3}
	\subsection{UC3 - Visualizzazione Notifica Avvenuto Successo Salvataggio file JSON}
		\begin{itemize}
			\item\textbf{Attore Primario}: utente;
			\item\textbf{Precondizioni}: l'utente ha salvato il file JSON generato a partire dai dati di addestramento (\hyperref[par:UC1.5]{UC1.5}); 
			\item\textbf{Postcondizioni}: l'utente visualizza la notifica di avvenuto salvataggio del file JSON;					\item\textbf{Scenario Principale}: 
				\begin{enumerate} 
					\item L’utente visualizza il messaggio di notifica "Avvenuto Successo Salvataggio File JSON" in cui viene notificato che il salvataggio (\hyperref[par:UC1.5]{UC1.5}) è avvenuto correttamente;
					\item L'utente clicca il pulsante "Ok" per proseguire o il pulsante contrassegnato con una "X" per uscire dalla finestra di notifica.		
				\end{enumerate}		
		\end{itemize}


%%%%%%%%%%%%%

	label{par:UC4}
	\subsection{UC4 - Caricamento del file JSON nel plug-in}
		\begin{itemize}
			\item\textbf{Attore Primario}: utente;
			\item\textbf{Precondizioni}: 
				\begin{enumerate}
					\item L’utente ha effettuato l’accesso alla piattaforma Grafana;
					\item L’utente ha selezionato una dashboard e ha aggiunto il plug-in “Predire in Grafana”,  che verrà visualizzato con il rispettivo pannello;
					\item L’utente dispone del file JSON contenente i predittori (\hyperref[par:UC1]{UC1}). 
				\end{enumerate}
			\item\textbf{Postcondizioni}:
				\begin{enumerate}
					\item L’utente ha caricato il file JSON con predittori associati nel plug-in;
					\item Viene letta la definizione del predittore dal file in formato JSON. 
				\end{enumerate}
			\item\textbf{Scenario Principale}:
				\begin{enumerate}
					\item (\hyperref[par:UC4.1]{UC4.1}) L’utente seleziona sul pannello l’opzione di caricamento del file JSON; 
					\item (\hyperref[par:UC4.2]{UC4.2}) L’utente seleziona il file JSON contenente i predittori da locale;
					\item (\hyperref[par:UC4.3]{UC4.3}) L'utente conferma il caricamento.
				\end{enumerate}
			\item\textbf{Estensione}:
				\begin{enumerate} 
					\item\hyperref[par:UC5]{UC5} estende \hyperref[par:UC4.1]{UC4.1}: viene visualizzato un messaggio di allert nel caso in cui sia già presente un file JSON caricato nel plug-in "Predire in Grafana".
					\item\hyperref[par:UC6]{UC6} estende \hyperref[par:UC4.3]{UC4.3}: viene visualizzato un messaggio d’errore nel caso in cui l’operazione di caricamento del file non sia andata a buon fine;
				\end{enumerate}	
			\item\textbf{Inclusione}: \hyperref[par:UC7]{UC7} include \hyperref[par:UC4.3]{UC4.3}: viene visualizzato a schermo un messaggio di notifica di avvenuto successo della procedura di caricamenteo JSON.
		\end{itemize}
		
		label{par:UC4.1}
		\subsubsection{UC4.1 - Selezione Opzione di Caricamento file JSON}
		\begin{itemize}
			\item\textbf{Attore Primario}: utente;
			\item\textbf{Precondizioni}: l’utente visualizza il pannello “Predire in Grafana” nella dashboard;
			\item\textbf{Postcondizioni}: l’utente ha cliccato il pulsante di caricamento del file JSON e visualizza la finestra per la selezione del file;
			\item\textbf{Scenario Principale}: l’utente clicca il pulsante con etichetta  “Carica JSON”.
			\item\textbf{Estensioni}: \hyperref[par:UC5]{UC5}: viene visualizzato un messaggio di allert nel caso in cui sia già presente un file JSON caricato nel plug-in "Predire in Grafana".
		\end{itemize}		
		
		label{par:UC4.2}
		\subsubsection{UC4.2 - Selezione del file JSON}
		\begin{itemize}
			\item\textbf{Attore Primario}: utente;
			\item\textbf{Precondizioni}: l’utente ha cliccato il pulsante "Carica JSON" (\hyperref[par:UC4.1]{UC4.1}) e visualizza il pannello di selezione del file;
			\item\textbf{Postcondizioni}: l’utente ha selezionato il file JSON;
			\item\textbf{Scenario Principale}: l’utente seleziona dalla finestra di selezione il file JSON da importare tra quelli disponibili; sono visibili solo i file con formato .json.
		\end{itemize}
		
		label{par:UC4.3}
		\subsubsection{UC4.3 - Conferma di Caricamento}
		\begin{itemize}
			\item\textbf{Attore Primario}: utente;
			\item\textbf{Precondizioni}: l’utente ha selezionato il file JSON da caricare;
			\item\textbf{Postcondizioni}: l’utente ha caricato il file nel plug-in; 
			\item\textbf{Scenario Principale}: L’utente clicca il pulsante etichettato con “Conferma” e il file viene caricato.
			\item\textbf{Estensione}: \hyperref[par:UC6]{UC6}: viene visualizzato un messaggio d’errore nel caso in cui l’operazione di caricamento del file non sia andata a buon fine;				
			\item\textbf{Inclusione}: \hyperref[par:UC7]{UC7}: viene visualizzato a schermo un messaggio di notifica di avvenuto successo della procedura di caricamenteo JSON.	
		\end{itemize}

%%%%%%%%%%%%%%
	
	label{par:UC5}
	\subsection{UC5 - Visualizzazione Messaggio di Allert file JSON già Caricato}
		\begin{itemize}
			\item\textbf{Attore Primario}: utente;
			\item\textbf{Precondizioni}: l’utente ha selezionato l'opzione di caricaemento del file JSON tramite click del pulsante "Carica JSON" (\hyperref[par:UC4.1]{UC4.1});
			\item\textbf{Postcondizioni}: l’utente visualizza un messaggio di allert in cui viene segnalato che è già presente un file JSON caricato nel plug-in; 
			\item\textbf{Scenario Principale}: 
				\begin{enumerate} 
					\item L’utente visualizza il messaggio di allert "File JSON già Caricato" in cui viene segnalato che è già stato caricato un file JSON in precedenza e la conferma procederà alla sua sovrascrizione;
					\item L'utente clicca il pulsante "Ok" per proseguire e caricare il nuovo file JSON oppure clicca il pulsante "Annulla" per ritornare alla sezione di caricamento.
				\end{enumerate}
		\end{itemize}	

%%%%%%%%%%%%%%
	
	label{par:UC6}
	\subsection{UC6 - Visualizzazione Messaggio d'Errore Caricamento file JSON }
		\begin{itemize}
			\item\textbf{Attore Primario}: utente;
			\item\textbf{Precondizioni}: l’utente ha selezionato il file JSON da caricare (\hyperref[par:UC4.2]{UC4.2}) e ha cliccato il pulsante di conferma (\hyperref[par:UC4.3]{UC4.3});
			\item\textbf{Postcondizioni}: l’utente visualizza un messaggio d'errore in cui viene segnalato che la struttura del file selezionato non è supportata una SVM o per una RL, di conseguenza viene riportato alla selezione del fiel JSON; 
			\item\textbf{Scenario Principale}: 
				\begin{enumerate} 
					\item L’utente visualizza il messaggio d'errore "Struttura del file JSON non Supportata" in cui viene segnalato che la definizione dei predittori del file JSON selezionato (\hyperref[par:UC4.2]{UC4.2}) non è sintatticamente corretta;
					\item L'utente clicca il pulsante "Ok" per proseguire e viene riportato alla sezione di selezione.
				\end{enumerate}
		\end{itemize}	


%%%%%%%%%%%%%%

	
	label{par:UC7}
	\subsection{UC7 - Visualizzazione Notifica Avvenuto Successo Caricamento JSON}
		\begin{itemize}
			\item\textbf{Attore Primario}: utente;
			\item\textbf{Precondizioni}: l’utente ha confermato il caricamento del file JSON selezionato  (\hyperref[par:UC4.3]{UC4.3});
			\item\textbf{Postcondizioni}: l’utente visualizza la notifica di avvenuto caricamento del file JSON nel plug-in; 
			\item\textbf{Scenario Principale}: 
				\begin{enumerate} 
					\item L’utente visualizza il messaggio di notifica "Avvenuto Successo Caricamento File JSON" in cui viene notificato che il caricamento (\hyperref[par:UC4]{UC4}) è avvenuto correttamente;
					\item L'utente clicca il pulsante "Ok" per proseguire o il pulsante contrassegnato con una "X" per uscire dalla finestra di notifica.		
				\end{enumerate}		
		\end{itemize}

%%%%%%%%%5%%%%
	
	label{par:UC8}
	\subsection{UC8 - Collegamento del Predittore al Flusso Dati}
		\begin{itemize}
			\item\textbf{Attore Primario}: utente;
			\item\textbf{Precondizioni}: 
				\begin{enumerate}
					\item L’utente ha caricato e letto con successo i predittori contenuti nel file JSON (\hyperref[par:UC4]{UC4});
					\item L’utente dispone di una serie di predittori da poter collegare al flusso di dati voluto;
					\item L’utente deve aver configurato la connessione al server tramite Grafana.	
				\end{enumerate}
			\item\textbf{Postcondizioni}: l’utente ha collegato correttamente i predittori ad un flusso dati, definendone le soglie per i rispettivi stati di ogni nodo e visualizza i collegamenti in un pannello dedicato. 

			\item\textbf{Scenario Principale}:
				\begin{enumerate}
					\item (\hyperref[par:UC8.1]{UC8.1}) L’utente seleziona il database che verrà scelto tra quelli disponibili memorizzati in Grafana;
					\item (\hyperref[par:UC8.2]{UC8.2}) l’utente seleziona il flusso dati contenuto in una tabella del DB;
					\item (\hyperref[par:UC8.3]{UC8.3}) l'utente selezione uno o più predittori scegliendoli tra quelli disponibili in una lista che verrà visualizzata una volta caricato il file JSON; 
					\item (\hyperref[par:UC8.4]{UC8.4}) l'utente seleziona il nodo del flusso dati da associare al predittore;
					\item (\hyperref[par:UC8.5]{UC8.5}) impostazione delle soglie;
					\item (\hyperref[par:UC8.6]{UC8.6}) l'utente conferma le impostazioni di collegamento selezionate.	
				\end{enumerate}
			\item\textbf{Estensione}: 
				\begin{enumerate}
					\item\hyperref[par:UC9]{UC9} estende \hyperref[par:UC8.5]{UC8.5}: viene visualizzato un messaggio di errore soglia non valida nel caso venga inserito un valore non consentito;
					\item\hyperref[par:UC10]{UC10} estende \hyperref[par:UC8.6]{UC8.6}: viene visualizzato un messaggio d’errore nel caso il collegamento non sia stato possibile;
				\end{enumerate}
			\item\textbf{Inclusione}:
				\begin{enumerate}
					\item\hyperref[par:UC11]{UC11} include \hyperref[par:UC8.6]{UC8.6}: viene visualizzata una notifica avvenuto collegamento al nodo del flusso dati se la procedura è andata a buon fine.
					\item\hyperref[par:UC12]{UC12} include \hyperref[par:UC8.6]{UC8.6}:  l'utente visualizza una lista dei collegamenti con le rispettive impostazioni di collegamento selezionate;
				\end{enumerate}
		\end{itemize}
		
		\label{par:UC8.1}
		\subsubsection{UC8.1 - Selezione Database}
		\begin{itemize}
			\item\textbf{Attore Primario}: utente;
			\item\textbf{Precondizioni}: l’utente ha configurato correttamente la connessione al server tramite Grafana e visualizza la lista di quelli disponibili;
			\item\textbf{Postcondizioni}: è stato selezionato il database contenente il flusso dati desiderato;
			\item\textbf{Scenario Principale}: l’utente seleziona il database da usare come sorgente dati tra quelli disponibili;
		\end{itemize}
		
		\label{par:UC8.2}
		\subsubsection{UC8.2 - Selezione del Flusso Dati}
		\begin{itemize}
			\item\textbf{Attore Primario}: utente;
			\item\textbf{Precondizioni}: l’utente ha selezionato il DB presente in Grafana (\hyperref[par:UC8.1]{UC8.1}) da cui selezionare il flusso dati interessato e visualizza la lista delle tabelle disponibili;
			\item\textbf{Postcondizioni}: l’utente ha selezionato la tabella e dispone di un flusso dati su cui fare il collegamento;
			\item\textbf{Scenario Principale}: l’utente seleziona la tabella che vuole utilizzare per effettuare il collegamento ai predittori scelti.
		\end{itemize}
		
		\label{par:UC8.3}
		\subsubsection{UC8.3 - Selezione del Predittore}
		\begin{itemize}
			\item\textbf{Attore Primario}: utente;
			\item\textbf{Precondizioni}: 
				\begin{enumerate}
					\item E' disponibile un flusso di dati da monitorare (\hyperref[par:UC8.2]{UC8.2}); 
					\item E' stato letto il file JSON con i predittori e l'utente visualizza una lista con i predittori disponibili.
				\end{enumerate}
			\item\textbf{Postcondizioni}: uno o più predittori sono stati selezionati dalla lista disponibile.
			\item\textbf{Scenario Principale}: l’utente seleziona il predittore che intende associare ad un aspetto del flusso.
		\end{itemize}
	
	\label{par:UC8.4}
	\subsubsection{UC8.4 - Selezione Nodo del Flusso Dati}
		\begin{itemize}
			\item\textbf{Attore Primario}: utente;
			\item\textbf{Precondizioni}:
				\begin{enumerate}
					\item Il predittore da collegare è stato selezionato dalla lista dei predittori (\hyperref[par:UC8.3]{UC8.3});
					\item  L’utente visualizza la lista di nodi del flusso di dati a cui poter associare il/i predittore/i selezionato/i; 
				\end{enumerate}
			\item\textbf{Postcondizioni}: l'utente ha selezionato il nodo dal flusso dati;
			\item\textbf{Scenario Principale}: l’utente seleziona l’aspetto interessato dal flusso di dati alla quale può essere agganciato il predittore selezionato (o i predittori).
		\end{itemize}

	\label{par:UC8.5}
	\subsubsection{UC8.5 - Impostazione delle Soglie}
		\begin{itemize}
			\item\textbf{Attore Primario}: utente;
			\item\textbf{Precondizioni}: l'utente ha selezionato almeno un predittore che vuole collegare al flusso dati (\hyperref[par:UC8.3]{UC8.3});
			\item\textbf{Postcondizioni}: l’utente ha impostato la soglia di monitoraggio sul preditore selezionato;
			\item\textbf{Scenario Principale}: l’utente dispone di almeno un predittore e aggiunge una soglia da associare; è possibile impostare più di una soglia. 
			\item\textbf{Estensioni}: \hyperref[par:UC9]{UC9} : viene visualizzato un messaggio di errore soglia non valida nel caso venga inserito un valore non consentito;	
		\end{itemize}

	\label{par:UC8.6}
	\subsubsection{UC8.6 - Conferma collegamento}
		\begin{itemize}
			\item\textbf{Attore Primario}: utente;
			\item\textbf{Precondizioni}: l’utente ha creato le basi per l'associazione del predittore al nodo del flusso dati scelto, in particolare è stato selezionato il predittore (\hyperref[par:UC8.3]{UC8.3}), il nodo del flusso dati da associare (\hyperref[par:UC8.4]{UC8.4} ed è stata impostata una soglia (\hyperref[par:UC8.5]{UC8.5};
			\item\textbf{Postcondizioni}: 
				\begin{enumerate}
					\item L’utente ha confermato il collegamento e visualizza la lista dei collegamenti;
					\item L'utente ha la possibilità di effettuare un altro collegamento tornando a \hyperref[par:UC8.1]{UC8.1}.
				\end{enumerate}

			\item\textbf{Scenario Principale}: l’utente clicca il pulsante eticchettato con "Conferma Collegamento".
			\item\textbf{Estensioni}:
				\begin{enumerate}
					\item\hyperref[par:UC10]{UC10} estende \hyperref[par:UC8.6]{UC8.6}: viene visualizzato un messaggio d’errore nel caso il collegamento non sia stato possibile;
				\end{enumerate}
			\item\textbf{Inclusione}: 
				\begin{enumerate}
					\item\hyperref[par:UC11]{UC11}: viene visualizzata una notifica avvenuto collegamento al nodo del flusso dati se la procedura è andata a buon fine.
					\item\hyperref[par:UC12]{UC12}:  l'utente visualizza una lista dei collegamenti con le rispettive impostazioni di collegamento selezionate;
				\end{enumerate}
		\end{itemize}

	
	

