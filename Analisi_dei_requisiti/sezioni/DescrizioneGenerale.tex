\section{Descrizione generale}
   \subsection{Caratteristiche del prodotto}
   Lo scopo del capitolato è la realizzazione di un plug-in(g) per Grafana(g) con la capacità di utilizzare una support vector machine(g) calibrata dall’utente e definita in formato JSON(g) in grado di lanciare allarmi e proporre miglioramenti in tempo reale(g).
Nello specifico il plug-in deve poter monitorare i dati in ingresso da un certo flusso, come per esempio percentuali di utilizzo della memoria o temperatura del processore, i quali verranno successivamente mostrati attraverso l’interfaccia grafica(g) di Grafana.
La necessità di lanciare un allarme o proporre un cambiamento verrà valutata dalla SVM in base ai predittori su cui è stata tarata dopo la fase di addestramento.
Il plug-in(g) rimane in esecuzione su grafana e riceve continuamente informazioni in ingresso da un flusso di dati: viene quindi continuamente ricalcolata la necessità di lanciare un alert o proporre un miglioramento.
La SVM sopra menzionata potrà essere sviluppata attraverso la libreria SVMJS(g) (\url{ https://github.com/karpathy/svmjs}).

	\subsection{Obiettivi del prodotto}
	L’obiettivo dell’elaborato consiste nella realizzazione di due plug-in che migliorino l’’efficienza di monitoraggio di un flusso di dati espandendo l’utilizzo di Grafana e consentendo ai clienti  interessati, ad avere un’analisi dei dati più precisa rispetto ad un utilizzo out-of-the-box(g) della piattaforma. Specificatamente il  fine del progetto consiste nella fornitura, attraverso una SVM o RL, di dati all’utente. Il beneficio derivato da una corretta applicazione dei plug-in è stato discusso in riunione esterna con la proponente: tenendo monitorato il flusso di dati con il prodotto è possibile ottenere previsioni probabilistiche. 
	\subsection{Caratteristiche degli Utenti}
	Il plug-in di Grafana che andremo a sviluppare sarà caratterizzato da un bacino di utenza e da un ambito di utilizzo relativamente ristretto.
Il software sarà rivolto all’utenza già registrata sulla piattaforma Grafana con la necessità di monitorare un flusso dati attraverso l’uso di una SVM o di una RL.

	 \subsection{Vincoli Progettuali}
	 Il prodotto finale è soggetto a vincoli progettuali obbligatori ed opzionali descritti all’interno del capitolato(G). I vincoli obbligatori richiesti dal proponente(G), sono:

	 	\begin{itemize}
	 		\item Due plug-in di Grafana, sviluppati nel linguaggio JavaScript
	 		\item  Produzione un file json dai dati di addestramento con i parametri per le previsioni
			\item Lettura della definizione del predittore dal file in formato JSON
			\item Associazione dei predittori letti dal file JSON al flusso di dati presente in Grafana 
			\item Applicazione della previsione e fornitura dei nuovi dati ottenuti al sistema di previsione di Grafana.
			\item Visualizzazione dei dati attraverso l’interfaccia grafica di Grafana.
			
	 	\end{itemize}
	 	I vincoli opzionali descritti nel capitolato potranno, a discrezione del fornitore, essere realizzati nella loro totalità o parzialmente, e sono i seguenti:
		\begin{itemize}
			\item Possibilità di definire uno stato di allarme al superamento dei livelli di soglia raggiunti dai nodi collegati alle previsioni
			\item Fornitura dei dati riguardanti l’affidabilità delle predizioni
			\item Possibilità di di applicare delle trasformazioni alle misure in ingresso dal campo per ottenere delle regressioni logaritmiche o esponenziali oltre a quelle lineari.
			\item Possibilità di addestrare la SVM o la RL direttamente in Grafana.
			\item Implementazione dei meccanismi di apprendimento di flusso, in modo da poter disporre di sistemi di previsione in costante adattamento ai dati rilevati sul campo.
			\item L’utilizzo di altri metodi di previsione, tra cui la versione delle SVM adattate alla regressione o piccole reti neurali per la classificazione.

		\end{itemize}