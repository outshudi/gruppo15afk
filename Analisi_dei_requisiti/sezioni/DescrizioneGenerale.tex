\section{Descrizione generale}
   
	\subsection{Obiettivi del prodotto}
	L’obiettivo dell’elaborato consiste nella realizzazione di due plug-in che migliorino l’efficienza di monitoraggio di un flusso di dati espandendo l’utilizzo di Grafana e consentendo ai clienti  interessati, ad avere un’analisi dei dati più precisa rispetto ad un utilizzo out-of-the-box\glo della piattaforma. Specificatamente il fine del progetto consiste nella fornitura, attraverso una SVM o RL, di dati all’utente. Il beneficio derivato da una corretta applicazione dei plug-in è stato discusso in riunione esterna con la proponente: tenendo monitorato il flusso di dati con il prodotto è possibile ottenere previsioni. 
	\subsection{Caratteristiche degli utenti}
	Il plug-in di Grafana che andremo a sviluppare sarà caratterizzato da un bacino di utenza e da un ambito di utilizzo relativamente ristretto.
Il software sarà rivolto all’utenza già registrata sulla piattaforma Grafana con la necessità di monitorare un flusso dati attraverso l’uso di una SVM o di una RL.

	 \subsection{Vincoli progettuali}
	 Il prodotto finale è soggetto a vincoli progettuali obbligatori ed opzionali descritti all’interno del capitolato. I vincoli obbligatori richiesti dal proponente, sono:

	 	\begin{itemize}
			\item un tool di addestramento\glo della SVM/RL esterno;
	 		\item un plug-in di Grafana, sviluppato nel linguaggio JavaScript;
	 		\item produzione un file JSON dai dati di addestramento con i parametri per le previsioni;
			\item lettura della definizione del predittore\glo dal file in formato JSON;
			\item associazione dei predittori letti dal file JSON al flusso di dati presente in Grafana; 
			\item applicazione della previsione\glo e fornitura dei nuovi dati ottenuti al sistema di previsione di Grafana;
			\item visualizzazione dei dati attraverso la dashboard\glo presente in Grafana.
	 	\end{itemize}

	 	I vincoli opzionali descritti nel capitolato potranno, a discrezione del fornitore, essere realizzati nella loro totalità o parzialmente, e sono i seguenti:
		\begin{itemize}
			\item possibilità di definire alert in base a livelli di soglia raggiunti dai nodi collegati alle previsioni;
			\item fornitura dei dati riguardanti l’affidabilità delle previsioni;
			\item possibilità di di applicare delle trasformazioni alle misure in ingresso dal campo per ottenere delle regressioni logaritmiche\glo o esponenziali\glo oltre a quelle lineari;
			\item possibilità di addestrare la SVM o la RL direttamente in Grafana;
			\item implementazione dei meccanismi di apprendimento di flusso, in modo da poter disporre di sistemi di previsione in costante adattamento ai dati rilevati sul campo;
			\item l’utilizzo di altri metodi di previsione, tra cui la versione delle SVM adattate alla regressione o piccole reti neurali\glo per la classificazione\glo.

		\end{itemize}