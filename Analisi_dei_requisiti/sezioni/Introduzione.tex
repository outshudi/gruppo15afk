\section{Introduzione}
	\subsection{Scopo del documento}
	L'analisi dei requisiti è un documento che rappresenta un vincolo tra il fornitore, che si impegna 		a sviluppare un software conforme ai vincoli patteggiati , e il cliente, che riconosce tali 			requisiti come quelli ricercati.
\\	Il presente documento ha lo scopo di descrivere i requisiti e le funzionalità individuate 			    analizzando il capitolato C4 e agli incontri con il proponente, Zucchetti spa .

	\subsection{Scopo del prodotto}
	Lo scopo del prodotto è lo sviluppo di due plug-in per la piattaforma Grafana. Le due estensioni     	produrranno delle previsioni di due tipi:
\\	
	\begin{itemize}
		\item \textbf {classificazioni}, attraverso l'uso della tecnica Support Vector Machine(SVM), per stimare il gruppo di appartenenza degli eventi dai dati "predittori"
		\item \textbf{regressioni} ,attraverso l'uso della tecnica Regressione Lineare, per stimare un valore numerico con campo continuo
	\end{itemize}
	
	\subsection{Glossario}
	
	\subsection{Riferimenti}
		\subsubsection{Normativi}
			\begin{itemize}
				\item \textbf{Norme di progetto}: \emph{Norme di Progetto v0.0.1;}
				\item \textbf{Capitolato d'appalto C4 - Predire con Grafana: plug-in per per piattaforma Grafana}: \url{https://www.math.unipd.it/~tullio/IS-1/2019/Progetto/C4.pdf}
				\item \textbf{Verbale esterno}: data; 
			\end{itemize}
		
		\subsubsection{Informativi}
			\begin{itemize}
				\item \textbf{Studio di Fattibilità}: \emph{Studio di Fattibilità v0.0.1;}
				\item \textbf{Capitolato d'appalto C4 - Predire con Grafana: plug-in per per piattaforma Grafana}: \url{https://www.math.unipd.it/~tullio/IS-1/2019/Progetto/C4.pdf}
				\item \textbf{Materiale didattico del corso di Ingegneria del Software:}
				\begin{itemize}
					\item \textbf{Libro del crso per esempio}
					\item \textbf{Sitografia varia penso}
					\item \textbf{altri riferimenti informativi vedremo}
				\end{itemize}
			\end{itemize}				
	