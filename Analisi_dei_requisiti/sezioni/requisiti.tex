%definizione colori per tabelle (tranne copertina)
\definecolor{redafk}{RGB}{255, 133, 51}
\definecolor{grey2}{RGB}{204, 204, 204}
\definecolor{lastrowcolor}{RGB}{156, 198, 214}
\definecolor{greyRowafk}{RGB}{234, 234, 234}
\rowcolors{2}{grey2}{greyRowafk}
\renewcommand{\arraystretch}{1.5}

\section{Requisiti} {
I requisiti seguiranno la seguente struttura riportata anche nel documento \emph{norme di progetto} :
				\begin{itemize}
					\item \textbf{codice identificativo}: è un codice univoco e conforme alla codifica: \\ \\
					\centerline{\textbf{Re[Importanza][Tipologia][Codice]}} \\ \\
					Le voci riportate nella precedente codifica significano: 
					\begin{itemize}
						\item \textbf{Importanza}: la quale può assumere come valori:
						\begin{itemize}
							\item 1: requisito obbligatorio, irrinunciablie;
							\item 2: requisito desiderabile, perciò non obbligatorio ma riconosscibile;
							\item 3: requisito opzionale, ovvero trattabile in un secondo momento o relativamente utile.
						\end{itemize}
						\item \textbf{Tipologia}: la quale può assumere come valori:
						\begin{itemize}
							\item F: funzionale;
							\item P: prestazionale;
							\item Q: qualitativo;
							\item V: vincolo.
						\end{itemize}
						\item \textbf{Codice identificativo}: il quale è un identificatore univoco del requisito, e viene espresso in forma gerarchica padre/figlio.
					\end{itemize}
					\item \textbf{Classificazione}: specifica il peso del requisito facilitando la sua lettura anche se causa ridondanza;
					\item \textbf{Descrizione}: sintesi completa di un requisito;
					\item \textbf{Fonti}: il requisito può avere le seguenti provenienze:
					\begin{itemize}
						\item capitolato;
						\item interno: requisito che gli analisti ritengono di aggiungere in base alle esigenze del team;
						\item caso d'uso: il requisito proviene da uno o più casi d'uso, dei quali è necessario riportare il codice univoco di caso d'uso;
						\item verbale: dopo un chiarimento da parte del proponente è possibile che sorga un requisito non preventivato. E le informazioni su di esso sono riportate e tracciati nei rispettivi verbali.
					\end{itemize}
				\end{itemize} 	}

%%%%%%%%%%%%%%%%%%%%%%%%%%%%%%%%%

	\subsection{Requisiti Funzionali}

\begin{table}[H]
\centering\renewcommand{\arraystretch}{1.5}
\caption{Tabella dei requisiti funzionali}
\vspace{0.2cm}
\begin{tabular}{ c  c  p{4 cm} C{2.5 cm} }
\rowcolor{redafk}
\textcolor{white}{\textbf{Requisito}} & \textcolor{white}{\textbf{Classificazione}} & 
\textcolor{white}{\textbf{Descrizione}} & \textcolor{white}{\textbf{Fonti}} \\
R1F1 & Obbligatorio & L’utente deve disporre dei dati di addestramento in formato CSV che potrà utilizzare  per creare il file JSON usando l’apposito tool & Capitolato UC1\\
R1F1.1 & Obbligatorio & L’utente potrà selezionare il pulsante di caricamento dati d'addestramento &  Interno UC1.1\\
R1F1.2 & Obbligatorio & All’utente sarà possibile selezionare i dati di addestramento &  Interno UC1.2\\
R1F1.3 & Obbligatorio & All’utente sarà possibile selezionare l’algoritmo con il quale potrà successivamente effettuare i calcoli di previsione &  Interno UC1.3\\
R1F1.4 & Obbligatorio & L’utente potrà confermare le procedure di addestramento impostate &  Interno UC1.4\\
R1F1.5 & Obbligatorio & L’utente potrà salvare in locale il file JSON prodotto & Interno UC1.5\\
R1F2 & Obbligatorio & L’utente potrà caricare il file JSON nel plugin & Capitolato UC2\\
R1F2.1 & Obbligatorio & All’utente sarà possibile selezionare l’opzione di caricamente del file JSON &  Interno UC2.1\\
\end{tabular}
\end{table}

\begin{table}[H]
\centering\renewcommand{\arraystretch}{1.5}
\caption{(continua)}
\vspace{0.2cm}
\begin{tabular}{ c  c  p{4 cm} C{2.5 cm} }
\rowcolor{redafk}
\textcolor{white}{\textbf{Requisito}} & \textcolor{white}{\textbf{Classificazione}} & 
\textcolor{white}{\textbf{Descrizione}} & \textcolor{white}{\textbf{Fonti}} \\
R1F2.2 & Obbligatorio & L’utente potrà selezionare il file JSON &  Interno UC2.2\\
R1F2.3 & Obbligatorio & L’utente potrà confermare il caricamente del file JSON &  Interno UC2.3\\
R1F3 & Obbligatorio & L’utente avrà la possibilità di collegare il predittore al flusso dati &  Capitolato UC3\\
R1F3.1 & Obbligatorio & L’utente potrà selezionare il database &  Interno UC3.1\\
R1F3.2 & Obbligatorio & All’utente sarà possibile selezionare il flusso dati &  Interno UC3.1\\
R1F3.3 & Obbligatorio & L’utente avrà la possibilità di selezionare il predittore  &  Interno UC3.3\\
R1F3.4 & Obbligatorio & L’utente potrà selezionare un nodo dal flusso dati &  Interno UC3.4\\
R1F3.5 & Obbligatorio & L’utente potrà impostare le soglie &  Capitolato UC3.3\\
R1F3.6 & Obbligatorio & All’utente sarà possibile confermare il collegamento &  Interno UC3.6\\
R1F4 & Obbligatorio & L’utente potrà effettuare lo scollegamento di un predittore &  Interno UC4\\
R1F5 & Obbligatorio & All’utente sarà possibile apportare delle modifiche ad un collegamento & Capitolato UC5\\
\end{tabular}
\end{table}


\begin{table}[H]
\centering\renewcommand{\arraystretch}{1.5}
\caption{(continua)}
\vspace{0.2cm}
\begin{tabular}{ c  c  p{4 cm} C{2.5 cm} }
\rowcolor{redafk}
\textcolor{white}{\textbf{Requisito}} & \textcolor{white}{\textbf{Classificazione}} & 
\textcolor{white}{\textbf{Descrizione}} & \textcolor{white}{\textbf{Fonti}} \\
R1F6 & Obbligatorio & L’utente potrà effettuare operazioni di calcolo di previsione &  Capitolato UC6\\
R1F6.1 & Obbligatorio & L’utente potrà inserire la politica temporale &  Interno UC6.1\\
R1F6.2 & Obbligatorio & L’utente avrà la possibilità di avviare il monitoraggio &  Interno UC6.2\\
R1F6.3 & Obbligatorio & All’utente sarà offerta la possibilità di effettuare il salvataggio delle previsione &  Verbale VI\_2020-03-31\\
R1F7 & Obbligatorio & L’utente potrà interrompere il monitoraggio & Interno UC7\\
R1F8 & Obbligatorio & L’utente potrà visualizzare le previsioni &  Capitolato UC8\\
R1F9 & Obbligatorio & All’utente sarà mostrato il messaggio di errore file CSV incompatibile &  Interno UC1.2\\
R1F10 & Obbligatorio & All’utente sarà mostrato un messaggio di avviso il quale segnalerà che il caricamento del file JSON è già avvenuto &  Interno UC2.1\\
R1F11 & Obbligatorio & L’utente vedrà un messaggio che segnalerà l’errato caricamento del file JSON &  Interno UC2.3\\
\end{tabular}
\end{table}

\begin{table}[H]
\centering\renewcommand{\arraystretch}{1.5}
\caption{(continua)}
\vspace{0.2cm}
\begin{tabular}{ c  c  p{4 cm} C{2.5 cm} }
\rowcolor{redafk}
\textcolor{white}{\textbf{Requisito}} & \textcolor{white}{\textbf{Classificazione}} & 
\textcolor{white}{\textbf{Descrizione}} & \textcolor{white}{\textbf{Fonti}} \\
R1F12 & Obbligatorio & All’utente verrà segnalato di aver selezionato una soglia non valida attraverso un messaggio di errore &  Interno
UC3.5\\
R1F13 & Obbligatorio & 
L’utente riceverà un messaggio d’errore segnalante l’errata impostazione di un collegamento & Interno UC3.6\\
R1F14 & Obbligatorio & L’utente vedrà una notifica di errore avvisandolo di non aver definito una politica temporale &  Interno UC6.2\\
R1F15 & Obbligatorio & L’utente verrà notificato attraverso un messaggio di errore di non aver collegato alcun predittore  &  Interno
UC6.2\\
R1F16 & Obbligatorio & L’utente verrà notificato attraverso un messaggio di aver salvato il file JSON con successo  & Interno UC1.5\\
R1F17 & Obbligatorio & L’utente riceverà un messaggio che confermerà che il caricamento del file JSON è avvenuto con successo &  Interno UC2.3\\
R1F18 & Obbligatorio & L’utente verrà notificato attraverso un messaggio del collegamento avvenuto con successo &  Interno
UC3.6\\
\end{tabular}
\end{table}

\begin{table}[H]
\centering\renewcommand{\arraystretch}{1.5}
\caption{(continua)}
\vspace{0.2cm}
\begin{tabular}{ c  c  p{4 cm} C{2.5 cm} }
\rowcolor{redafk}
\textcolor{white}{\textbf{Requisito}} & \textcolor{white}{\textbf{Classificazione}} & 
\textcolor{white}{\textbf{Descrizione}} & \textcolor{white}{\textbf{Fonti}} \\
R1F19 & Obbligatorio & L’utente visualizzerà il pannello con la lista dei collegamenti & Interno UC3.6\\
R1F20 & Obbligatorio & L’utente verrà notificato con un messaggio che è possibile procedere con lo scollegamento &  Interno UC4\\
R1F21 & Obbligatorio & L’utente riceverà un messaggio che lo notificherà di aver avviato il monitoraggio con successo  & Interno UC6.2\\
R1F22 & Obbligatorio & L’utente verrà notificato attraverso un messaggio di aver interrotto il monitoraggio &  Interno UC7\\
R1F23 & Obbligatorio & L’utente visualizzerà un messaggio che i dati di previsione verranno salvati all’interno del database & Interno UC6.3\\
R3F24 & Opzionale & L’utente riceverà un messaggio di alert che segnalerà il raggiungimento di soglia critica & Capitolato UC3.5\\
\end{tabular}
\end{table}

%%%%%%%%%%%%%%%%%%%%%%%%%%%%%%%%%


 	\subsection{Requisiti di Qualità}

\begin{table}[H]
\centering\renewcommand{\arraystretch}{1.5}
\caption{Tabella dei requisiti di qualità}
\vspace{0.2cm}
\begin{tabular}{ c  c  p{4 cm} C{2.5 cm} }
\rowcolor{redafk}
\textcolor{white}{\textbf{Requisito}} & \textcolor{white}{\textbf{Classificazione}} & 
\textcolor{white}{\textbf{Descrizione}} & \textcolor{white}{\textbf{Fonti}} \\
R1Q1 & Obbligatorio & La codifica e la progettazione devono rispettare le norme definite nel documento \emph{Piano\_di\_Qualifica v1.0.0} & Interno\\
R1Q2 & Obbligatorio & E’ necessario rendere disponibile un manuale utente per l’utilizzo del prodotto &  Capitolato\\
R1Q2.1 & Obbligatorio & Il manuale utente deve essere disponibile in lingua inglese  & Interno\\
R2Q2.2 & Desiderabile & Il manuale utente deve essere disponibile in lingua italiana &  Interno\\
R1Q3 & Obbligatorio & E’ necessario rendere disponibile un manuale per la manutenzione ed estensione del prodotto & Capitolato\\
R1Q4 & Obbligatorio & Il prodotto deve essere sviluppato in modo concorde a quanto stabilito nelle \emph{Norme\_di\_Progetto\_v1.0.0} & Capitolato\\
R2Q5 & Desiderabile & Il codice sorgente deve essere disponibile in una repository pubblica su GitHub\glo o su altre piattaforme & Capitolato\\
R2Q6 & Desiderabile & Il plug-in deve essere caricato nella sezione \href{https:// grafana.com/plugins}{Grafana Labs Plugins} & Interno\\
\end{tabular}
\end{table}

%%%%%%%%%%%%%%%%%%%%%%%%%%

	\subsection{Requisiti di Vincolo}

\begin{table}[H]
\centering\renewcommand{\arraystretch}{1.5}
\caption{Tabella dei requisiti di vincolo}
\vspace{0.2cm}
\begin{tabular}{ c  c  p{4 cm} C{2.5 cm} }
\rowcolor{redafk}
\textcolor{white}{\textbf{Requisito}} & \textcolor{white}{\textbf{Classificazione}} & 
\textcolor{white}{\textbf{Descrizione}} & \textcolor{white}{\textbf{Fonti}} \\
R1V1 & Obbligatorio & Il Sistema deve essere supportato su browser differenti & Supporto al linguaggio ECMAScript6\\
R1V1.1 & Obbligatorio & Il Sistema deve essere supportato sul browser Microsoft Edge dalla versione 14 & Supporto al linguaggio ECMAScript6\\
R1V1.2 & Obbligatorio & Il Sistema deve essere supportato sul browser Chrome dalla versione 58 &  Supporto al linguaggio ECMAScript6\\
R1V1.3 & Obblifgatorio & Il Sistema deve essere supportato sul browser Firefox dalla versione 54 &   Supporto al linguaggio ECMAScript6\\
R1V1.4 & Obbligatorio & Il Sistema deve essere supportato sul browser Safari dalla versione 10 &  Supporto al linguaggio ECMAScript6\\
R1V2 & Obbligatorio & Il file contenente i dati di addestramento deve essere in formato CSV &  Verbale VI\_2020-03-31\\
R1V3 & Obbligatorio & L’applicazione deve essere sviluppata utilizzando JavaScript 6 (ES6) & Capitolato\\
\end{tabular}
\end{table}


\begin{table}[H]
\centering\renewcommand{\arraystretch}{1.5}
\caption{(continua)}
\vspace{0.2cm}
\begin{tabular}{ c  c  p{4 cm} C{2.5 cm} }
\rowcolor{redafk}
\textcolor{white}{\textbf{Requisito}} & \textcolor{white}{\textbf{Classificazione}} & 
\textcolor{white}{\textbf{Descrizione}} & \textcolor{white}{\textbf{Fonti}} \\
R1V4 & Desiderabile & Il tool di addestramento deve essere sviluppato utilizzando il framework React\glo & Interno\\
R1V5 & Obbligatorio & Il codice sorgente del plug-in deve essere open source\glo & Capitolato\\
R1V6 & Obbligatorio & Lo sviluppo dell’interfaccia del plug-in è realizzato attraverso l’uso di tecnologie web\glo & Interno\\
\end{tabular}
\end{table}

%%%%%%%%%%%%%%%%%%%%%%%%%%%%%%

	\subsection{Requisiti prestazionali}{
Non sono stati individuati requisiti prestazionali in quanto il progetto sarà costituito da un tool di addestramento e un plug-in. Come database di supporto verrà utilizzato InfluxDB che renderà più efficiente la gestione e la reperibilità dei dati temporali. Essendo l’esecuzione del plug-in affidata alla piattaforma “Grafana”  le prestazioni dipenderanno dalla condizione dei server della piattaforma stessa.}

%%%%%%%%%%%%%%%%%%%%%%%%%%%%%%


	\subsection{Tracciamento}
		
		\subsubsection{Fonte - Requisiti}

\begin{table}[H]
\centering\renewcommand{\arraystretch}{1.5}
\caption{Tabella di tracciamento fonte-requisiti}
\vspace{0.2cm}
\begin{tabular}{ C{4 cm} C{4 cm}}
\rowcolor{redafk}
\textcolor{white}{\textbf{Fonte}} & \textcolor{white}{\textbf{Requisiti}}\\
Capitolato & R1F1\newline R1F2 \newline R1F3 \newline R1F3.5 \newline R1F5 \newline R1F6 \newline R1F8 \newline R3F24\newline R1Q2\newline R1Q3\newline R1Q4\newline R1Q5\newline R1V3\\
Interno & R1F1.1\newline R1F1.2\newline R1F1.3\newline R1F1.4\newline R1F1.5\newline R1F2.1\newline R1F2.2\newline R1F2.3\newline R1F3.1\newline R1F3.2\newline R1F3.3\newline R1F3.4\newline R1F3.6\newline R1F4\newline R1F6.1\newline  R1F6.2\newline R1F7\newline R1F9\newline R1F10\newline R1F11\newline R1F12\newline  R1F13\newline  R1F14\newline  R1F15\\
\end{tabular}
\end{table}

	
\begin{table}[H]
\centering\renewcommand{\arraystretch}{1.5}
\caption{(continua)}
\vspace{0.2cm}
\begin{tabular}{ C{4 cm} C{4 cm}}
\rowcolor{redafk}
\textcolor{white}{\textbf{Fonte}} & \textcolor{white}{\textbf{Requisiti}}\\

Interno & R1F16 \newline R1F17\newline  R1F18\newline R1F19\newline  R1F20\newline  R1F21\newline  R1F22\newline  R1F23\newline  R1Q1\newline  R1Q2.1\newline  R2Q2.2\newline  R2Q6\newline  R1V4
\newline  R1V6\\
UC1 & R1F1\\
UC1.1 & R1F1.1\\
UC1.2 & R1F1.2\newline R1F9\\
UC1.3 & R1F1.3\\
UC1.4 & R1F1.4\\
UC1.5 & R1F1.5\newline R1F16\\
UC2 & R1F2\\
UC2.1 & R1F2.1\newline R1F10\\
UC2.2 & R1F2.2\\
UC2.3 & R1F2.3\newline R1F11\newline R1F17\\
UC3 & R1F3\\
UC3.1 & R1F3.1\newline R1F3.2\\
UC3.3 & R1F3.3\newline R1F3.5\\
UC3.4 & R1F3.4\\
\end{tabular}
\end{table}


\begin{table}[H]
\centering\renewcommand{\arraystretch}{1.5}
\caption{(continua)}
\vspace{0.2cm}
\begin{tabular}{ C{4 cm} C{4 cm}}
\rowcolor{redafk}
\textcolor{white}{\textbf{Requisito}} & \textcolor{white}{\textbf{Fonte}}\\
UC3.5 & R1F12\newline R1F24\\
UC3.6 & R1F3.6\newline R1F13\newline R1F18\newline R1F19\\
UC4 & R1F4\newline R1F20\\
UC5 & R1F5\\
UC6 & R1F6\\
UC6.1 & R1F6.1\\
UC6.2 & R1F6.2\newline R1F14\newline R1F15\newline R1F21\\
UC6.3 & R1F23\\
UC7 & R1F7\newline R1F22\\
UC8 & R1F8\\
Verbale VI\_2020-03-31 & R1F6.3\newline R1V2\\
Supporto al linguaggio ECMAScript6 & R1V1\newline R1V1.1\newline R1V1.2\newline R1V1.3\newline R1V1.4\\
\end{tabular}
\end{table}	
		\subsubsection{Requisito - Fonti}

\begin{table}[H]
\centering\renewcommand{\arraystretch}{1.5}
\caption{Tabella di tracciamento requisito-fonti}
\vspace{0.2cm}
\begin{tabular}{ C{4 cm} C{4 cm}}
\rowcolor{redafk}
\textcolor{white}{\textbf{Requisito}} & \textcolor{white}{\textbf{Fonte}}\\
R1F1 & Capitolato\newline UC1\\
R1F1.1 & Interno\newline UC1.1\\
R1F1.2 & Interno\newline UC1.2\\
R1F1.3 & Interno\newline UC1.3\\
R1F1.4 & Interno\newline UC1.4\\
R1F1.5 & Interno\newline UC1.5\\
R1F2 & Capitolato\newline UC2\\
R1F2.1 & Interno\newline UC2.1\\
R1F2.2 & Interno\newline UC2.2\\
R1F2.3 & Interno\newline UC2.3\\
R1F3 & Capitolato\newline UC3\\
R1F3.1 & Interno\newline UC3.1\\
R1F3.2 & Interno\newline UC3.1\\
R1F3.3 & Interno\newline UC3.3\\
R1F3.4 & Interno\newline UC3.4\\
R1F3.5 & Capitolato\newline UC3.3\\
\end{tabular}
\end{table}
	
	
\begin{table}[H]
\centering\renewcommand{\arraystretch}{1.5}
\caption{(continua)}
\vspace{0.2cm}
\begin{tabular}{ C{4 cm} C{4 cm}}
\rowcolor{redafk}
\textcolor{white}{\textbf{Requisito}} & \textcolor{white}{\textbf{Fonte}}\\
R1F3.6 & Interno\newline UC3.6\\
R1F4 & Interno\newline UC4\\
R1F5 & Capitolato\newline UC5\\
R1F6 & Capitolato\newline UC6\\
R1F6.1 & Interno\newline UC6.1\\
R1F6.2 & Interno\newline UC6.2\\
R1F6.3 & Verbale VI\_2020-03-31\\
R1F7 & Interno\newline UC7\\
R1F8 & Capitolato\newline UC8\\
R1F9 & Interno\newline UC1.2\\
R1F10 & Interno\newline UC2.1\\
R1F11 & Interno\newline UC2.3\\
R1F12 & Capitolato\newline UC3.5\\
R1F13 & Interno\newline UC3.6\\
R1F14 & Interno\newline UC6.2\\
R1F15 & Interno\newline UC6.2\\
R1F16 & Interno\newline UC1.5\\
\end{tabular}
\end{table}


\begin{table}[H]
\centering\renewcommand{\arraystretch}{1.5}
\caption{(continua)}
\vspace{0.2cm}
\begin{tabular}{ C{4 cm} C{4 cm}}
\rowcolor{redafk}
\textcolor{white}{\textbf{Requisito}} & \textcolor{white}{\textbf{Fonte}}\\
R1F17 & Interno\newline UC2.3\\
R1F18 & Interno\newline UC3.6\\
R1F19 & Interno\newline UC3.6\\
R1F20 & Interno\newline UC4\\
R1F21 & Interno\newline UC6.2\\
R1F22 & Interno\newline UC7\\
R1F23 & Interno\newline UC6.3\\
R3F24 & Capitolato\newline UC3.5\\
R1Q1 & Interno\\
R1Q2 & Capitolato\\
R1Q2.1 & Interno\\
R2Q2.2 & Interno\\
R1Q3 & Capitolato\\
R1Q4 & Capitolato\\
R2Q5 & Capitolato\\
R2Q6 & Interno\\
R1V1 & Supporto al linguaggio ECMAScript6\\
R1V1.1 & Supporto al linguaggio ECMAScript6\\
R1V1.2 & Supporto al linguaggio ECMAScript6\\
\end{tabular}
\end{table}


\begin{table}[H]
\centering\renewcommand{\arraystretch}{1.5}
\caption{(continua)}
\vspace{0.2cm}
\begin{tabular}{ C{4 cm} C{4 cm}}
\rowcolor{redafk}
\textcolor{white}{\textbf{Requisito}} & \textcolor{white}{\textbf{Fonte}}\\
R1V1.3 & Supporto al linguaggio ECMAScript6\\
R1V1.4 & Supporto al linguaggio ECMAScript6\\
R1V2 & Verbale VI\_2020-03-31\\
R1V3 & Capitolato\\
R1V4 & Interno\\
R1V5 & Capitolato\\
R1V6 & Interno\\
\end{tabular}
\end{table}


	
%%%%%%%%%%%%%%%%%%%%%%%%%%%%%%

	\subsection{Considerazioni}
Nel caso in cui le attività pianificate dovessero terminare in anticipo e dovessero avanzare ore di lavoro, i requisiti potrebbero subire alcune modifiche o aggiunte, per permettere la revisione delle voci presenti o delle migliorie. Dunque eventuali aggiornamenti sono lasciati a momenti
futuri.


