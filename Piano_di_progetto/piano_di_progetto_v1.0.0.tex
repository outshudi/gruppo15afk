\documentclass[a4paper, oneside, openany, dvipsnames, table, 12pt]{article}
\usepackage{../Template/AFKstyle}
\usepackage{hyperref}
\usepackage{verbatim} %per commenti di più righe \begin{comment} \end{comment}
\usepackage{amsmath}
\newcommand{\Titolo}{Analisi dei equisiti}

\newcommand{\Gruppo}{AFK}

\newcommand{\Redattori}{Simone Federico Bergamin\newline Alessandro Canesso\newline Victor Dutca\newline Fouad Farid\newline Olivier Utshudi\newline}

\newcommand{\Verificatori}{Alessandro Canesso \newline Simone Meneghin\newline Olivier Utshudi }

\newcommand{\pathimg}{../Analisi_dei_requisiti/img/logoAFK.png}

\newcommand{\Approvatore}{\newline Davide Zilio\newline}

\newcommand{\Distribuzione}{Prof. Vardanega Tullio \newline Prof. Cardin Riccardo \newline Gruppo AFK}

\newcommand{\Uso}{Esterno}

\newcommand{\NomeProgetto}{Predire in Grafana}

\newcommand{\Mail}{gruppoafk15@gmail.com}

\newcommand{\Versionedoc}{1.0.0}

\newcommand{\DescrizioneDoc}{Analisi dei requisiti del gruppo AFK nella realizzazione del progetto Predire in Grafana.}


\makeindex

\begin{document}
\copertina{}

%definizione colori per tabelle (tranne copertina)
\definecolor{redafk}{RGB}{255, 71, 87}
\definecolor{grey2}{RGB}{204, 204, 204}
\definecolor{greyRowafk}{RGB}{234, 234, 234}
\rowcolors{2}{grey2}{greyRowafk}
\renewcommand{\arraystretch}{1.5}

\newpage
\section*{Registro delle modifiche}
{
	\centering
	\begin{longtable}{ c c  C{4cm}  c  c }
		\rowcolor{redafk}
		\textcolor{white}{\textbf{Versione}} & \textcolor{white}{\textbf{Data}} & \textcolor{white}{\textbf{Descrizione}} & \textcolor{white}{\textbf{Nominativo}} & \textcolor{white}{\textbf{Ruolo}}\\		
		0.0.9 & 2020-03-31 & Completata stesura \S 3 & Olivier Utshudi &\reda{} \\
		0.0.8 & 2020-03-30 & Stesura \S A.2 & Simone Meneghin &\reda{} \\
		0.0.7 & 2020-03-29 & Iniziata stesura \S B & Simone Meneghin &\reda{} \\
		0.0.6 & 2020-03-29 & Inizio Stesura \S 3 & Olivier Utshudi &\reda{} \\
		0.0.5 & 2020-03-29 & Stesura \S A.1 & Simone Meneghin &\reda{} \\
		0.0.4 & 2020-03-28 & Iniziata stesura \S A & Simone Meneghin &\reda{} \\
		0.0.3 & 2020-03-28 & Stesura \S 1 & Simone Meneghin &\reda{} \\
		0.0.2 & 2020-03-28 & Stesura \S 2 & Davide Zilio &\reda{} \\
		0.0.1 & 2020-03-27 & Creato documento latex & Davide Zilio &\reda{}\\		
		
	\end{longtable}

}

%Didascalia tabelle/immagini (prendono come riferimento la subsection)
\counterwithin{table}{subsection}
\counterwithin{figure}{subsection}
\newpage

%indice, indice figure e indice tabelle
\tableofcontents
\newpage
\listoffigures
\newpage
\listoftables
\newpage

\begin{comment}

ATTENZIONE
1) all'inizo e fino a metà maggio tutti hanno più tempo = TUTTI fanno le ore necessarie (base)	
2) le nuove distribuzioni verranno attuate dal 18/05 per le fasi più concitanti del progetto, ossia quelle di:	
- progettazione dettaglio e codifica		
- valutazione e collaudo	
contando che ci saranno da dare tutti, o quasi, gli arretrati da giugno in poi.

NUOVE DISTRIBUZIONI
x = esami arretrati

1) x > 3: -2 ore
2) 2 <= x <= 3: ore base definite per ogni fase
2a) caso Olivier: +1 ora (perchè hai da dare 2(+1) esami (calcolo + RO(+ TW proj)) ma essi sono comunque "meno complicati" di p2 proj e uno tra P3 e algo quindi il tempo da dedicarci è "minore")
3) x < 2: +2 ore

\end{comment}


\section{Introduzione}

\subsection{Scopo del documento}
Questo documento ha lo scopo di mostrare le strategie di verifica\glo e validazione\glo adottate al fine di garantire la qualità di prodotto e di processo. Per raggiungere questo obiettivo viene applicato un sistema di verifica continua sui processi in corso e sulle attività svolte. In questo modo è quindi possibile rilevare e correggere all'istante eventuali anomalie, riducendo al minimo lo spreco delle risorse.

\subsection{Scopo del prodotto}
Lo scopo del prodotto è quello di realizzare due plug-in per il software Grafana\glo, che permettano di monitorare e predire lo stato di un sistema in analisi. Grazie alle predizioni sarà possibile attivare degli allarmi così da poter gestire preventivamente eventuali situazioni di rischio. \\
I due plug-in\glo utilizzeranno la Support Vector Machine\glo (SVM) per poter effettuare regressione lineare o categorizzazione sui dati forniti.
\begin{comment}
I due plug-in\glo utilizzeranno la Support Vector Machine\glo (SVM) o la Regressione Lineare per classificazione o regressione sui dati forniti.
\end{comment}

\subsection{Glossario}
Per evitare ambiguità nei documenti formali, viene fornito il documento \textbf{Glossario},
contenente tutti i termini considerati di difficile comprensione. Perciò nella documentazione fornita, ogni vocabolo contenuto in Glossario è contrassegnato dalla lettera G a pedice.

\subsection{Riferimenti}
\subsubsection{Riferimenti normativi}
\begin{itemize}
	\item Norme di Progetto: \textit{Norme\_di\_Progetto\_v1.0.0};
	\item ISO/IEC 9126: \url{https://en.wikipedia.org/wiki/ISO/IEC_9126};
	\item ISO/IEC 15504: \url{https://en.wikipedia.org/wiki/ISO/IEC_15504}.
\end{itemize}
\subsubsection{Riferimenti informativi}
\begin{itemize}
	\item Capitolato d'appalto C4: \url{https://www.math.unipd.it/~tullio/IS-1/2019/Progetto/C4.pdf}.
	\item \textbf{Slide L06 del corso Ingegneria del Software - Gestione di Progetto}: \\
	\url{https://www.math.unipd.it/~tullio/IS-1/2019/Dispense/L06.pdf};
	\item \textbf{Ingegneria del Software - Ian Sommerville - 10\textsuperscript{a} Edizione}.
\end{itemize}

\pagebreak

%\section{Gestione dei rischi}
I rischio viene inteso come l'evento che non vorremmo accadesse nel corso di un progetto, in quanto influienzerebbe in maniera negativa sulla qualità, o sulla riuscita stessa del prodotto. Inoltre, essendo un evento che può riguardare qualunque aspetto del progetto, la gestione dei rischi risulta fondamentale per la riuscita dello stesso. Per questo motivo il gruppo intede affrontare questo compito nel seguente modo:\\
\begin{itemize}
\item \textbf{Identificazione dei rischi}: vengono identificati i rischi, distinguendoli in rischi per iprogetto, il prodotto e l'azienda;
\item \textbf{Analisi dei rischi}: viene valutata la probabilità dell'evento e la sua pericolosità;
\item \textbf{Pianificazione dei rischi}: viene stabilito un piano per la prevenzione del rischio annullandone gli effetti, quando possibile, o per lo meno mitigarne le conseguenze;
\item \textbf{Monitoraggio dei rischi}: ad ogni ridefinizione del \textbf{Piano di Progetto}, i rischi vengono nuovamente controllati sulla base delle nuove informazioni.
\end{itemize}
Inoltre viene fatta la seguente classificazione dei rischi:
\begin{itemize}
\item \textbf{RT}: Rischio Tecnologico;
\item \textbf{RO}: Rischi Organizzativo;
\item \textbf{RI}: Rischio Interpersonale;
\end{itemize}

\begin{longtable}{C{3cm} L{4.5cm} L{4.5cm} C{3.15cm}}
\rowcolor{white}\caption{Tabella dei rischi} \\
		\rowcolor{redafk}
\textcolor{white}{\textbf{Codice-Nome}} &
\textcolor{white}{\textbf{Descrizione}} &
\textcolor{white}{\textbf{Rilevamento}} &
\textcolor{white}{\textbf{Grado}}  \\
		\endfirsthead
		\rowcolor{white}\caption[]{(continua)} \\
		\rowcolor{redafk}
\textcolor{white}{\textbf{Codice-Nome}} &
\textcolor{white}{\textbf{Descrizione}} &
\textcolor{white}{\textbf{Rilevamento}} &
\textcolor{white}{\textbf{Grado}} \\
		\endhead
		
RO40 - Emergenza sanitaria &
Un'epidemia riscontrata nel territorio, può costringere le autorità a porre restrizioni per ridurne l'espansione. &
Le ristrizioni descritte dal DCPM 2020-03-08 permettono le sole interazioni telematiche tra gli stakeholders. & 
Probabilità: 2 
Pericolosità: 2 \\

Piano di contingenza &
\multicolumn{3}{L{13cm}}{Gli stakeholders dovranno decidere di utilizzare gli strumenti di comunicazione disponibili a tutti che limitino i disagi scaturiti dalle suddette restrizioni.} \\

RT41 - Inesperienza Tecnologica &
Molte delle tecnologie adottate per lo sviluppo del progetto sono nuove per i componenti, che potrebbero usarle in modo non ottimale. &
Il \textit{responsabile} ha il compito di essere al corrente delle conoscenze dei componenti. & 
Probabilità: 
2 
Pericolosità: 
2\\ 

Piano di contingenza &
\multicolumn{3}{L{13cm}}{Il \textit{Responsabile} una volta messo al corrente delle  conoscenze dei componenti, affiderà loro i ruoli che più li competono.} \\

RO32 - Calcolo dei costi &
L'insesperienza del gruppo può portare alla sottovalutazione dei costi da sostenere. &
Il \textit{responsabile} ha il compito di essere al corrente delle conoscenze dei componenti. & 
Probabilità: 
1 
Pericolosità: 
2\\ 

Piano di contingenza &
\multicolumn{3}{L{13cm}}{È consigliato comunticare tempestivamente al committente la variazione dei costi.} \\

RO33 - Impegni accademici &
Essendo questo un progetto universitario, è probabile che in corso d'opera i componeti debbano sostenere attività accademiche che li sottrarrebbero, dagli impegi di progetto. &
Ogni componente deve saper comunicare con chiarezza quelli che sono i propri impegni accademici. & 
Probabilità: 
2
Pericolosità: 
1 \\ 

Piano di contingenza &
\multicolumn{3}{L{13cm}}{È consigliato comunticare tempestivamente al \textit{Responsabile} i propri impegni accademici.} \\

RO34 - Impegni personali &
\'E possibile che in corso d'opera i componeti debbano sostenere attività che li sottrarrebbero, dagli impegi di progetto. &
Ogni componente deve saper comunicare con chiarezza nel calendario quelli che sono i propri impegni. & 
Probabilità: 
2
Pericolosità: 
1 \\ 

Piano di contingenza &
\multicolumn{3}{L{13cm}}{È consigliato comunticare tempestivamente al \textit{Responsabile} i propri impegni.} \\


RO15 - Ritardi &
Le problematiche sopracitate possono comportare ritardi non indifferenti ai fini di progetto. &
Per questo l'incaricato dell'attività deve comunicare tempestivamente il ritardo. & 
Probabilità: 
1
Pericolosità: 
0 \\ 

Piano di contingenza &
\multicolumn{3}{L{13cm}}{È consigliato riassegnare risorse laddove ce ne sia bisogno, e quindi risolvere il motivo del ritardo.} \\

RI26 - Comunicazione interna &
Può essere che in determinati momenti un elemento del gruppo non sia raggiungibile &
I membri del gruppo devono segnalare la momentanea assenza dell'interessato/a & 
Probabilità: 
0
Pericolosità: 
2 \\ 

Piano di contingenza &
\multicolumn{3}{L{13cm}}{Il gruppo ha adottato diversi mezzi di comunicazione} \\

RI26 - Comunicazione esterna &
Se si presentano problematiche come RO40, il proponente potrebbe non sempre essere reperibile &
I membri del gruppo organizzeranno le conferenze con il proponente con più largo anticipo & 
Probabilità: 
0
Pericolosità: 
2 \\ 

Piano di contingenza &
\multicolumn{3}{L{13cm}}{Il gruppo ha adottato diversi mezzi di comunicazione per rimanere in contatto con il proponente} \\

RI37 - Contrasti interni &
Essendo l'attività di progetto un lavoro collaborativo, è possibile che i membri abbiano opinioni divergenti riguardo a determinate tematiche &
I membri del gruppo organizzeranno le conferenze con il proponente con più largo anticipo & 
Probabilità: 
0
Pericolosità: 
2 \\ 

Piano di contingenza &
\multicolumn{3}{L{13cm}}{Il gruppo ha adottato diversi mezzi di comunicazione per rimanere in contatto con il proponente} \\

\end{longtable}


%\pagebreak

\input{sezioni/modello_di_sviluppo.tex}
\pagebreak

\input{sezioni/pianificazione.tex}
\pagebreak


\section{Preventivo di periodo}
Per facilitare la lettura delle tabelle, vengono utilizzate le seguenti sigle per identificare i diversi ruoli e per ognuno di essi vengono indicati i relativi costi/h: \begin{itemize}
\item \textbf{Re}: \textit{Responsabile} 30€/h;
\item \textbf{Am}: \textit{Amministratore} 20€/h;
\item \textbf{An}: \textit{Analista} 25€/h;
\item \textbf{Pt}: \textit{Progettista} 22€/h;
\item \textbf{Pm}: \textit{Programmatore} 15€/h;
\item \textbf{Ve}: \textit{Verificatore} 15€/h.
\end{itemize}
Inoltre, se le ore ricoperte in un determinato ruolo fossero nulle, la cella presenterà il simbolo "-" per indicarne l'assenza.

\subsection{Fase di Analisi}
\subsubsection{Distribuzione oraria}
In questa fase, i ruoli sono così suddivisi:
\begin{table}[H]
\centering\renewcommand{\arraystretch}{1.5}
\caption{Distribuzione delle ore nella fase di Analisi}
\vspace{0.2cm}
\begin{tabular}{ c | c | c | c | c | c | c | c }
\rowcolor{redafk}
\textcolor{white}{\textbf{Nominativo}} & \textcolor{white}{\textbf{Re}} & 
\textcolor{white}{\textbf{Am}} & \textcolor{white}{\textbf{An}} &
\textcolor{white}{\textbf{Pt}} & \textcolor{white}{\textbf{Pm}} &
\textcolor{white}{\textbf{Ve}} & \textcolor{white}{\textbf{Totale}} \\
Simone Federico Bergamin & 6 & 7 & 20 & - & - & 9 & 42 \\
Alessandro Canesso & 8 & 6 & 16 & - & - & 12 & 42 \\
Victor Dutca & 9 & - & 15 & - & - & 16 & 40 \\
Fouad Farid	& 7 & 7 & 12 & 6 & - & 8 & 40 \\
Simone Meneghin & - & 8 & 14 & 10 & - & 10 & 42 \\
Olivier Utshudi & - & 8 & 13 & 8 & - & 13 & 42 \\
Davide Zilio & 4 & 5 & 17 & - & - & 14 & 40 \\
\rowcolor{lastrowcolor}
\textbf{Ore totali ruolo} & \textbf{34} & \textbf{41} & \textbf{107} & \textbf{24} & \textbf{0} & \textbf{82} & \textbf{288} \\
\end{tabular}
\end{table}

\pagebreak

I dati ottenuti sono riassunti nel seguente istogramma:
\begin{figure}[H]
\centering
\includegraphics[scale=0.60]{img/grafici/tabella_fase_analisi.png}
\caption{Istogramma della ripartizione delle ore per ruolo nella fase di Analisi}
\end{figure}

\subsubsection{Prospetto economico}
In questa fase il costo per ogni ruolo è il seguente:

%tabella costi
\begin{table}[H]
\centering\renewcommand{\arraystretch}{1.5}
\caption{Prospetto dei costi nella fase di Analisi}
\vspace{0.2cm}
\begin{tabular}{ c | c | c  }
\rowcolor{redafk}
\textcolor{white}{\textbf{Ruolo}} & \textcolor{white}{\textbf{Ore}} & 
\textcolor{white}{\textbf{Costo}}  \\
Responsabile & 34 & 1020€ \\
Amministratore & 41 & 820€ \\
Analista & 107 & 2675€ \\
Progettista	& 24 & 528€ \\
Programmatore & 0 & 0€  \\
Verificatore & 82 & 1230€  \\
\rowcolor{lastrowcolor}
\textbf{Totale} & \textbf{288} & \textbf{6273€}  \\
\end{tabular}
\end{table}

I dati ottenuti sono riassunti nel seguente areogramma:
\begin{figure}[H]
\centering
\includegraphics[scale=0.60]{img/grafici/torta_fase_analisi_prospetto_economico.png}
\caption{Areogramma della ripartizione dei costi per ruolo nella fase di Analisi}
\end{figure}

%--------------------------------------------------

\subsection{Fase di Progettazione architetturale}
\subsubsection{Distribuzione oraria}
In questa fase, i ruoli sono così suddivisi:
\begin{table}[H]
\centering\renewcommand{\arraystretch}{1.5}
\caption{Distribuzione delle ore nella fase di Progettazione architetturale}
\vspace{0.2cm}
\begin{tabular}{ c | c | c | c | c | c | c | c }
\rowcolor{redafk}
\textcolor{white}{\textbf{Nominativo}} & \textcolor{white}{\textbf{Re}} & 
\textcolor{white}{\textbf{Am}} & \textcolor{white}{\textbf{An}} &
\textcolor{white}{\textbf{Pt}} & \textcolor{white}{\textbf{Pm}} &
\textcolor{white}{\textbf{Ve}} & \textcolor{white}{\textbf{Totale}} \\
Simone Federico Bergamin & - & - & 10 & 7 & 5 & 8 & 30 \\
Alessandro Canesso & - & 5 & - & 10 & 9 & 8 & 32 \\
Victor Dutca & 3 & 6 & 4 & 10 & 7 & - & 30 \\
Fouad Farid	& - & 5 & - & 14 & - & 11 & 30 \\
Simone Meneghin & 6 & - & 9 & 10 & 7 & - & 32 \\
Olivier Utshudi & - & 4 & - & 10 & 6 & 12 & 32 \\
Davide Zilio & 3 & - & 13 & - & - & 14 & 30 \\
\rowcolor{lastrowcolor}
\textbf{Ore totali ruolo} & \textbf{12} & \textbf{20} & \textbf{36} & \textbf{61} & \textbf{34} & \textbf{53} & \textbf{216} \\
\end{tabular}
\end{table}

I dati ottenuti sono riassunti nel seguente istogramma:
\begin{figure}[H]
\centering
\includegraphics[scale=0.60]{img/grafici/tabella_fase_prog_architetturale.png}
\caption{Istogramma della ripartizione delle ore per ruolo nella fase di Progettazione architetturale}
\end{figure}

\subsubsection{Prospetto economico}
In questa fase il costo per ogni ruolo è il seguente:

%tabella costi
\begin{table}[H]
\centering\renewcommand{\arraystretch}{1.5}
\caption{Prospetto dei costi nella fase di Progettazione architetturale}
\vspace{0.2cm}
\begin{tabular}{ c | c | c  }
\rowcolor{redafk}
\textcolor{white}{\textbf{Ruolo}} & \textcolor{white}{\textbf{Ore}} & 
\textcolor{white}{\textbf{Costo}}  \\
Responsabile & 12 & 360€ \\
Amministratore & 20 & 400€ \\
Analista & 36 & 900€ \\
Progettista	& 61 & 1342€ \\
Programmatore & 34 & 510€  \\
Verificatore & 53 & 795€  \\
\rowcolor{lastrowcolor}
\textbf{Totale} & \textbf{216} & \textbf{4307€}  \\
\end{tabular}
\end{table}

I dati ottenuti sono riassunti nel seguente areogramma:
\begin{figure}[H]
\centering
\includegraphics[scale=0.60]{img/grafici/torta_fase_prog_architetturale.png}
\caption{Areogramma della ripartizione dei costi per ruolo nella fase di Progettazione architetturale}
\end{figure}

%--------------------------------------------------

\subsection{Fase di Progettazione di dettaglio e codifica}
\subsubsection{Distribuzione oraria}
In questa fase, i ruoli sono così suddivisi:
\begin{table}[H]
\centering\renewcommand{\arraystretch}{1.5}
\caption{Distribuzione delle ore nella fase di Progettazione di dettaglio e codifica}
\vspace{0.2cm}
\begin{tabular}{ c | c | c | c | c | c | c | c }
\rowcolor{redafk}
\textcolor{white}{\textbf{Nominativo}} & \textcolor{white}{\textbf{Re}} & 
\textcolor{white}{\textbf{Am}} & \textcolor{white}{\textbf{An}} &
\textcolor{white}{\textbf{Pt}} & \textcolor{white}{\textbf{Pm}} &
\textcolor{white}{\textbf{Ve}} & \textcolor{white}{\textbf{Totale}} \\
Simone Federico Bergamin & - & 6 & - & 12 & 18 & 12 & 48 \\
Alessandro Canesso & 4 & 3 & - & 10 & 18 & 11 & 46 \\
Victor Dutca & - & 8 & - & 10 & 20 & 10 & 48 \\
Fouad Farid	& 4 & - & - & 12 & 20 & 12 & 48 \\
Simone Meneghin & 2 & - & - & 12 & 22 & 14 & 50 \\
Olivier Utshudi & 8 & - & - & 8 & 21 & 12 & 49 \\
Davide Zilio & - & 6 & - & 10 & 20 & 12 & 48 \\
\rowcolor{lastrowcolor}
\textbf{Ore totali ruolo} & \textbf{18} & \textbf{23} & \textbf{0} & \textbf{74} & \textbf{139} & \textbf{83} & \textbf{337} \\
\end{tabular}
\end{table}

I dati ottenuti sono riassunti nel seguente istogramma:
\begin{figure}[H]
\centering
\includegraphics[scale=0.60]{img/grafici/tabella_fase_prog_cod.png}
\caption{Istogramma della ripartizione delle ore per ruolo nella fase di Progettazione di dettaglio e codifica}
\end{figure}

\subsubsection{Prospetto economico}
In questa fase il costo per ogni ruolo è il seguente:

%tabella costi
\begin{table}[H]
\centering\renewcommand{\arraystretch}{1.5}
\caption{Prospetto dei costi nella fase di Progettazione di dettaglio e codifica}
\vspace{0.2cm}
\begin{tabular}{ c | c | c  }
\rowcolor{redafk}
\textcolor{white}{\textbf{Ruolo}} & \textcolor{white}{\textbf{Ore}} & 
\textcolor{white}{\textbf{Costo}}  \\
Responsabile & 18 & 540€ \\
Amministratore & 23 & 460€ \\
Analista & 0 & 0€ \\
Progettista	& 74 & 1628€ \\
Programmatore & 139 & 2085€  \\
Verificatore & 83 & 1245€  \\
\rowcolor{lastrowcolor}
\textbf{Totale} & \textbf{337} & \textbf{5958€}  \\
\end{tabular}
\end{table}

I dati ottenuti sono riassunti nel seguente areogramma:
\begin{figure}[H]
\centering
\includegraphics[scale=0.60]{img/grafici/torta_fase_prog_cod.png}
\caption{Areogramma della ripartizione dei costi per ruolo nella fase di Progettazione di dettaglio e codifica}
\end{figure}

%--------------------------------------------------
\pagebreak

%comandi per inserimento tabelle (corrette e funzionanti)
\begin{comment}
%definizione colori per tabelle (tranne copertina)
\definecolor{redafk}{RGB}{255, 71, 87}
\definecolor{grey2}{RGB}{204, 204, 204}
\definecolor{lastrowcolor}{RGB}{156, 198, 214}
\definecolor{greyRowafk}{RGB}{234, 234, 234}
\rowcolors{2}{grey2}{greyRowafk}
\renewcommand{\arraystretch}{1.5}


%tabella ore
\begin{table}[H]
\centering\renewcommand{\arraystretch}{1.5}
\caption{Fase di Analisi}
\vspace{0.2cm}
\begin{tabular}{ c | c | c | c | c | c | c | c }
\rowcolor{redafk}
\textcolor{white}{\textbf{Nominativo}} & \textcolor{white}{\textbf{Re}} & 
\textcolor{white}{\textbf{Am}} & \textcolor{white}{\textbf{An}} &
\textcolor{white}{\textbf{Pt}} & \textcolor{white}{\textbf{Pr}} &
\textcolor{white}{\textbf{Ve}} & \textcolor{white}{\textbf{Totale}} \\
Simone Federico Bergamin & 6 & 6 & 18 & - & - & 9 & 39 \\
Alessandro Canesso & 6 & 5 & 18 & - & - & 10 & 39 \\
Victor Dutca & 10 & - & 15 & - & - & 14 & 39 \\
Fouad Farid	& 6 & 9 & 16 & - & - & 8 & 39 \\
Simone Meneghin & - & 7 & 10 & 12 & - & 10 & 39 \\
Olivier Utshudi & - & 8 & 10 & 8 & - & 13 & 39 \\
Davide Zilio & 4 & 5 & 16 & - & - & 14 & 39 \\
\rowcolor{lastrowcolor}
\textbf{Ore totali ruolo} & \textbf{32} & \textbf{40} & \textbf{103} & \textbf{20} & \textbf{0} & \textbf{78} & \textbf{273} \\
\end{tabular}
\end{table}


%tabella costi
\begin{table}[H]
\centering\renewcommand{\arraystretch}{1.5}
\caption{Fase di Analisi}
\vspace{0.2cm}
\begin{tabular}{ c | c | c  }
\rowcolor{redafk}
\textcolor{white}{\textbf{Ruolo}} & \textcolor{white}{\textbf{Ore}} & 
\textcolor{white}{\textbf{Costo}}  \\
Responsabile & 32 & 960€ \\
Amministratore & 40 & 800€ \\
Analista & 103 & 2575€ \\
Progettista	& 20 & 440€ \\
Programmatore & 0 & 0€  \\
Verificatore & 78 & 1170€  \\
\rowcolor{lastrowcolor}
\textbf{Totale} & \textbf{273} & \textbf{5945€}  \\
\end{tabular}
\end{table}

\input{sezioni/tabelle/tabellaFaseArchitetturale.tex}
\input{sezioni/tabelle/tabellaFaseDettaglioCodifica.tex}
\end{comment}

\section{Consuntivo di periodo}
Di seguito verranno indicate le spese effettivamente sostenute da ogni ruolo. Il bilancio di consuntivo potrà risultare: \begin{itemize}
\item \textbf{Positivo}: se il preventivo supera il consuntivo;
\item \textbf{Pari}: se preventivo e consuntivo sono uguali;
\item \textbf{Negativo}: se il consuntivo supera il preventivo.
\end{itemize}

\subsection{Analisi}
%tabella costi
\begin{table}[H]
\centering\renewcommand{\arraystretch}{1.5}
\caption{Consuntivo del periodo di Analisi}
\vspace{0.2cm}
\begin{tabular}{ c | c | c  }
\rowcolor{redafk}
\textcolor{white}{\textbf{Ruolo}} & \textcolor{white}{\textbf{Ore}} & 
\textcolor{white}{\textbf{Costo}}  \\
Responsabile & 34 & 1020€ \\
Amministratore & 41 (+13) & 820€ (+260€) \\
Analista & 107 (-4) & 2675€ (-100€) \\
Progettista	& 24 & 528€  \\
Programmatore & 0 & 0€  \\
Verificatore & 82 (+9) & 1230€ (+135€)  \\
\textbf{Totale preventivo} & \textbf{288} & \textbf{6273€}  \\
\textbf{Totale consuntivo} & \textbf{306} & \textbf{6568€}  \\
\textbf{Differenza} & \textbf{18} & \textbf{+295€}  \\
\end{tabular}
\end{table}
\pagebreak

\section{Organigramma}

%definizione colori per tabelle (tranne copertina)
\definecolor{redafk}{RGB}{255, 71, 87}
%\definecolor{grey2}{RGB}{204, 204, 204}
%\definecolor{greyRowafk}{RGB}{234, 234, 234}
%\definecolor{lastrowcolor}{RGB}{156, 198, 214}
\rowcolors{2}{white}{white}
\renewcommand{\arraystretch}{1.5}

\subsection{Redazione} 
\begin{table}[H]
	\begin{center}
	\begin{tabular}{ c c C{8cm} }
		\rowcolor{redafk}
		\textcolor{white}{\textbf{Nominativo}} & \textcolor{white}{\textbf{Data di redazione}} & \textcolor{white}{\textbf{Firma}} \\
		Olivier Outshudi & 2020-04-10 & \includegraphics[scale=0.3]{img/firme/outshudi.png}\\
		Simone Meneghin & 2020-04-10 & \includegraphics[scale=0.10]{img/firme/meneghin.png}\\
		Davide Zilio & 2020-04-10 & \includegraphics[scale=0.4]{img/firme/zilio.png}\\
	\end{tabular}
	\end{center}	
\end{table}

\subsection{Approvazione} 
\begin{table}[H]
	\begin{center}
	\begin{tabular}{ c c C{8cm} }
		\rowcolor{redafk}
		\textcolor{white}{\textbf{Nominativo}} & \textcolor{white}{\textbf{Data di approvazione}} & \textcolor{white}{\textbf{Firma}} \\
		 & 2020-04-11 & \\
		Tullio Vardanega &  & \\
		Riccardo Cardin &  & \\
	\end{tabular}
	\end{center}	
\end{table}

\subsection{Accettazione dei componenti}
\begin{table}[H]	
	\begin{center}
	\begin{tabular}{ c c C{6cm}}
		\rowcolor{redafk}
		\textcolor{white}{\textbf{Nominativo}} & \textcolor{white}{\textbf{Data di accettazione}} & \textcolor{white}{\textbf{Firma}} \\
		Simone Federico Bergamin & 2020-03-09 & \includegraphics[scale=0.2]{img/firme/bergamin.png}\\
		Alessandro Canesso & 2020-03-09 & \includegraphics[scale=0.3]{img/firme/canesso.png}\\
		Victor Dutca & 2020-03-09 & \includegraphics[scale=0.2]{img/firme/dutca.png}\\
		Fouad Farid & 2020-03-09 & \includegraphics[scale=0.2]{img/firme/farid.png}\\
		Simone Meneghin & 2020-03-09 & \includegraphics[scale=0.10]{img/firme/meneghin.png}\\
		Olivier Outshudi & 2020-03-09 & \includegraphics[scale=0.3]{img/firme/outshudi.png}\\
		Davide Zilio & 2020-03-09 & \includegraphics[scale=0.4]{img/firme/zilio.png}\\
	\end{tabular}
	\end{center}
\end{table}

\subsection{Componenti}
\begin{table}[H]	
	\begin{center}
	\begin{tabular}{ c c C{8cm} }
		\rowcolor{redafk}
		\textcolor{white}{\textbf{Nominativo}} & \textcolor{white}{\textbf{Matricola}} & \textcolor{white}{\textbf{Indirizzo email}} \\
		Simone Federico Bergamin & 1144724  & simonefederico.bergamin@studenti.unipd.it \\
		Alessandro Canesso & 1122701 & alessandro.canesso@studenti.unipd.it\\
		Victor Dutca & 1122137 & victor.dutca@studenti.unipd.it\\
		Fouad Farid & 1122195 & fouad.farid@studenti.unipd.it\\
		Simone Meneghin & 1174926 & simone.meneghin@studenti.unipd.it\\
		Olivier Outshudi & 1143556 & olivier.utshudi@studenti.unipd.it\\
		Davide Zilio & 1149807 & davide.zilio.3@studenti.unipd.it\\
	\end{tabular}
	\end{center}
\end{table}
\pagebreak


\end{document}