\section{Consuntivo di periodo}
Di seguito verranno indicate le spese effettivamente sostenute da ogni ruolo. Il bilancio di consuntivo potrà risultare: \begin{itemize}
\item \textbf{Positivo}: se il preventivo supera il consuntivo;
\item \textbf{Pari}; se preventivo e consuntivo sono uguali;
\item \textbf{Negativo}: se il consuntivo supera il preventivo.
\end{itemize}

\subsection{Analisi}
%tabella costi
\begin{table}[H]
\centering\renewcommand{\arraystretch}{1.5}
\caption{Consuntivo del periodo di Analisi}
\vspace{0.2cm}
\begin{tabular}{ c | c | c  }
\rowcolor{redafk}
\textcolor{white}{\textbf{Ruolo}} & \textcolor{white}{\textbf{Ore}} & 
\textcolor{white}{\textbf{Costo}}  \\
Responsabile & 34 & 1020€ \\
Amministratore & 41 (+12) & 820€ (+240€) \\
Analista & 107 (+5) & 2675€ (+125€) \\
Progettista	& 24 & 528€ \\
Programmatore & 0 & 0€  \\
Verificatore & 82 (+8) & 1230€ (+120€)  \\
\textbf{Totale preventivo} & \textbf{288} & \textbf{6273€}  \\
\textbf{Totale consuntivo} & \textbf{313} & \textbf{6758€}  \\
\textbf{Differenza} & \textbf{25} & \textbf{+485€}  \\
\end{tabular}
\end{table}