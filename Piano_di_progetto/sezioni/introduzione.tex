\section{Introduzione}

\subsection{Premessa}
Per stabilire le varie attività, il gruppo si è basato sui processi, sui bisogni, e sui vincoli di dipendenza che intervengono nel progetto. In questo modo è stato possibile stabilire per ciascuna attività, il tempo e le persone da impiegare visto che sono risorse fondamentali per la vita di qualunque progetto.

\subsection{Scopo del documento}
Lo scopo del documento è quello di definire le attività da svolgere nel progetto, e di collocarle in una linea temporale.\\
Nello specifico il documento è così strutturato:
\begin{itemize}
\item analisi dei Rischi;
\item descrizione modello di sviluppo;
\item collocazione membri nelle attività;
\item stima delle risorse per llo sviuppo del progetto;
\end{itemize}
\subsection{Scopo del prodotto}
Lo scopo del prodotto è quello di realizzare due plug-in per il software Grafana\glo, che permettano di monitorare e predire lo stato di un sistema in analisi. Grazie alle predizioni sarà possibile attivare degli allarmi così da poter gestire preventivamente eventuali situazioni di rischio. \\
I due plug-in\glo utilizzeranno la Support Vector Machine\glo (SVM) per poter effettuare regressione lineare o categorizzazione sui dati forniti.
\begin{comment}
I due plug-in\glo utilizzeranno la Support Vector Machine\glo (SVM) o la Regressione Lineare per classificazione o regressione sui dati forniti.
\end{comment}

\subsection{Glossario}
Per evitare ambiguità nei documenti formali, viene fornito il documento \textbf{Glossario},
contenente tutti i termini considerati di difficile comprensione. Perciò nella documentazione fornita, ogni vocabolo contenuto in Glossario è contrassegnato dalla lettera G a pedice.

\subsection{Riferimenti}
\subsubsection{Riferimenti normativi}
\begin{itemize}
	\item Norme di Progetto: \textit{Norme\_di\_Progetto\_v1.0.0}.
\end{itemize}
\subsubsection{Riferimenti informativi}
\begin{itemize}
	\item Capitolato d'appalto C4: \url{https://www.math.unipd.it/~tullio/IS-1/2019/Progetto/C4.pdf}.
	\item \textbf{Slide L06 del corso Ingegneria del Software - Gestione di Progetto}: \\
	\url{https://www.math.unipd.it/~tullio/IS-1/2019/Dispense/L06.pdf};
	\item Ingegneria del Software - Ian Sommerville - 10\textsuperscript{a} Edizione.
\end{itemize}
\subsection{Scadenze}
Il gruppo \textit{TeamAFK} si impegna a presentare il proprio materiale nei seguenti appuntamenti:\\
\begin{itemize}
\item \textbf{Revisione dei Requisiti}: 2020-04-20;
\item \textbf{Revisione di Progettazione}: 2020-05-18;
\item \textbf{Revisione di Qualifica}: 2020-06-18;
\item \textbf{Revisione di Accettazione}: 2020-07-13. 
\end{itemize}
